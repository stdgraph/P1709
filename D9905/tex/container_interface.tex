
\chapter{Graph Container Interface}



The Graph Container Interface defines the primitive concepts, traits, types and functions used to define and access an adacency graph, no matter its internal design and organization. Thus, it is designed to reflect all forms of adjacency graphs including a vector of lists, CSR-based graph and adjacency matrix, whether they are in the standard or external to the standard.

All algorithms in this proposal require that vertices are stored in random access containers and that \tcode{vertex_id_t<G>} is integral, and it is assumed that all future algorithm proposals will also have the same requirements. 

The Graph Container Interface is designed to support a wider scope of graph containers than required by the views and algorithms in this proposal. This enables for future growth of the graph data model (e.g. incoming edges on a vertex), or as a framework for graph implementations outside of the standard. For instance, existing implementations may have requirements that cause them to define features with looser constraints, such as sparse vertex\_ids, non-integral vertex\_ids, or storing vertices in associative bi-directional containers (e.g. std::map or std::unordered\_map). Such features require specialized implementations for views and algorithms. The performance for such algorithms will be sub-optimal, but is preferrable to run them on the existing container rather than loading the graph into a high-performance graph container and then running the algorithm on it, where the loading time can far outweigh the time to run the sub-optimal algorithm. To achieve this, care has been taken to make sure that the use of concepts chosen is appropriate for algorithm, view and container.



% \section{Naming Conventions}

% Table~\ref{tab:name_conv} shows the naming conventions used throughout this document.

% \begin{table}[h!]
%   \begin{center}
%   {\begin{tabular}{l l l p{7cm}}
%      \hline
%      \textbf{Template}  &                                   & \textbf{Variable}    &                                                                                                                                                                                                  \\
%      \textbf{Parameter} & \textbf{Type Alias}               & \textbf{Names}       & \textbf{Description}                                                                                                                                                                             \\
%      \hline
%      \tcode{G}          &                                   &                      & Graph                                                                                                                                                                                            \\
%      & \tcode{graph_reference_t<G>}      & \tcode{g}            & Graph reference                                                                                                                                                                                  \\
%      \tcode{GV}         &                                   & \tcode{val}          & Graph Value, value or reference                                                                                                                                                                  \\
%      \hline
%      \tcode{V}          & \tcode{vertex_t<G>}               &                      & Vertex                                                                                                                                                                                           \\
%      & \tcode{vertex_reference_t<G>}     & \tcode{u,v,x,y}      & Vertex reference. \tcode{u} is the source (or only) vertex. \tcode{v} is the target vertex.                                                                                                      \\
%      \tcode{VId}        & \tcode{vertex_id_t<G>}            & \tcode{uid,vid,seed} & Vertex id. \tcode{uid} is the source (or only) vertex id. \tcode{vid} is the target vertex id.                                                                                                   \\
%      \tcode{VV}         & \tcode{vertex_value_t<G>}         & \tcode{val}          & Vertex Value, value or reference. This can be either the user-defined value on a vertex, or a value returned by a function object (e.g. \tcode{VVF}) that is related to the vertex.              \\
%      \tcode{VR}         & \tcode{vertex_range_t<G>}         & \tcode{ur,vr}        & Vertex Range                                                                                                                                                                                     \\
%      \tcode{VI}         & \tcode{vertex_iterator_t<G>}      & \tcode{ui,vi}        & Vertex Iterator. \tcode{ui} is the source (or only) vertex.                                                                                                                                      \\
%      &                                   & \tcode{first,last}   & \tcode{vi} is the target vertex.                                                                                                                                                                                    \\
%      \tcode{VVF}        &                                   & \tcode{vvf}          & Vertex Value Function: vvf(u) $\rightarrow$ vertex value, or vvf(uid) $\rightarrow$ vertex value, depending on requirements of the consume algorithm or view.                                    \\
%      \tcode{VProj}      &                                   & \tcode{vproj}        & Vertex descriptor projection function: \tcode{vproj(x)} $\rightarrow$ \tcode{vertex_descriptor<VId,VV>}.                                                                                         \\
%      \hdashline
%                         & \tcode{partition_id_t<G>}         & \tcode{pid}          & Partition id.                                                                                                                                                                                    \\
%                         &                                   & \tcode{P}            & Number of partitions.                                                                                                                                                                            \\
%      \tcode{PVR}        & \tcode{partition_vertex_range_t<G>} & \tcode{pur,pvr}    & Partition vertex range.                                                                                                                                                                          \\
%      \hline
%      \tcode{E}          & \tcode{edge_t<G>}                 &                      & Edge                                                                                                                                                                                             \\
%      & \tcode{edge_reference_t<G>}       & \tcode{uv,vw}        & Edge reference. \tcode{uv} is an edge from vertices \tcode{u} to \tcode{v}. \tcode{vw} is an edge from vertices \tcode{v} to \tcode{w}.                                                                             \\
%      \tcode{EId}        & \tcode{edge_id_t<G>}              & \tcode{eid,uvid}     & Edge id, a pair of vertex\_ids.                                                                                                                                                                  \\
%      \tcode{EV}         & \tcode{edge_value_t<G>}           & \tcode{val}          & Edge Value, value or reference. This can be either the user-defined value on an edge, or a value returned by a function object (e.g. \tcode{EVF}) that is related to the edge.                   \\
%      \tcode{ER}         & \tcode{vertex_edge_range_t<G>}    &                      & Edge Range for edges of a vertex                                                                                                                                                                 \\
%      \tcode{EI}         & \tcode{vertex_edge_iterator_t<G>} & \tcode{uvi,vwi}      & Edge Iterator for an edge of a vertex. \tcode{uvi} is an iterator for an edge from vertices \tcode{u} to \tcode{v}. \tcode{vwi} is an iterator for an edge from vertices \tcode{v} to \tcode{w}. \\
%      \tcode{EVF}        &                                   & \tcode{evf}          & Edge Value Function: evf(uv) $\rightarrow$ edge value, or evf(eid) $\rightarrow$ edge value, depending on the requirements of the consuming algorithm or view.                                   \\
%      \tcode{EProj}      &                                   & \tcode{eproj}        & Edge descriptor projection function: \tcode{eproj(x)} $\rightarrow$ \tcode{edge_descriptor<VId,Sourced,EV>}.                                                                                     \\
%      \hdashline
%      \tcode{PER}        & \tcode{partition_edge_range_t<G>} &                      & Partition Edge Range for edges of a partition vertex.                                                                                                                                            \\
%      \hline
%   \end{tabular}}
%     \caption{Naming Conventions for Types and Variables}
%     \label{tab:name_conv}
%   \end{center}
% \end{table}



\section{Core Concepts}

This section describes the core concepts to describe the adjacency lists used for graphs in the Graph Library. There are a 
number of prefixes used for adjacency list are as follows.
\begin{itemize}
    \item \textbf{basic} where there is only a vertex id but no vertex object.
    \item \textbf{index} where the vertex range is random-access and the vertex id is integral.
    \item \textbf{sourced} where an edge has a source id.
\end{itemize}

\subsection{Edge Concepts}
The types of edges that can occur in a graph are described with the edges concepts. The \tcode{E} edge template parameter allows
for different types of edges, such as incoming and outgoing edge types.
{\small
     \lstinputlisting{D9905/src/concepts_edges.hpp}
}

There is a single edge range concept.

\phil{\tcode{edges(g,u)} isn't being used anywhere. We could use \tcode{basic_adjacency_list} in the View and algorihm definitions.}


{\small
     \lstinputlisting{D9905/src/concepts_target_edge_range.hpp}
}

\subsection{Vertex Concepts}
The \tcode{vertex_range} concept is the general definition used for adjacency lists while \tcode{index_vertex_range} is used for
high performance graphs where vertices typically stored in a \tcode{vector}.
{\small
     \lstinputlisting{D9905/src/concepts_vertex_range.hpp}
}

\subsection{Adjacency List Concepts}
The basic adjacency lists add concepts where there is no vertex object, only a vertex id.
{\small
     \lstinputlisting{D9905/src/concepts_basic_adj_list.hpp}
}

The adjacency list concepts bring together the vertex and edge concepts used for core graph concepts. 
All algorithms initially proposed for the Graph Library use the \tcode{index_adjacency_list}.
{\small
     \lstinputlisting{D9905/src/concepts_adj_list.hpp}
}

\section{Traits}
Table \ref{tab:graph_traits} summarizes the type traits in the Graph Container Interface, allowing views and algorithms to query the graph's characteristics.

\begin{table}[h!]
\begin{center}
%\resizebox{\textwidth}{!}
{\begin{tabular}{l l L{7.0cm}}
\hline
    \textbf{Trait} & \textbf{Type} & \textbf{Comment} \\
\hline
    \tcode{has_degree<G>} & concept & Is the \tcode{degree(g,u)} function available? \\
    \tcode{has_find_vertex<G>} & concept & Are the \tcode{find_vertex(g,_)} functions available? \\
    \tcode{has_find_vertex_edge<G>} & concept & Are the \tcode{find_vertex_edge(g,_)} functions available?\\
    \tcode{has_contains_edge<G>} & concept & Is the \tcode{contains_edge(g,uid,vid)} function available?\\
\hline
    \tcode{define_unordered_edge<G,E> : false_type} & struct & Specialize for edge implementation to derive from \tcode{true_type} for unordered edges \\
    \tcode{is_unordered_edge<G,E>} & struct & \tcode{conjunction<define_unordered_edge<E>, is_sourced_edge<G, E>>} \\
    \tcode{is_unordered_edge_v<G,E>} & type alias & \\
    \tcode{unordered_edge<G,E>} & concept & \\
\hline
    \tcode{is_ordered_edge<G,E>} & struct & \tcode{negation<is_unordered_edge<G,E>>} \\
    \tcode{is_ordered_edge_v<G,E>} & type alias & \\
    \tcode{ordered_edge<G,E>} & concept & \\
\hline
    \tcode{define_adjacency_matrix<G> : false_type} & struct & Specialize for graph implementation to derive from \tcode{true_type} for edges stored as a square 2-dimensional array \\
    \tcode{is_adjacency_matrix<G>} & struct & \\
    \tcode{is_adjacency_matrix_v<G>} & type alias & \\
    \tcode{adjacency_matrix<G>} & concept & \\
\hline
\end{tabular}}
\caption{Graph Container Interface Type Traits}
\label{tab:graph_traits}
\end{center}
\end{table}


\section{Types}
Table \ref{tab:graph_type} summarizes the type aliases in the Graph Container Interface. These are the types used to define the objects 
in a graph container, no matter its internal design and organization. Thus, it is designed to be able to reflect all forms of adjacency 
graphs including a vector of lists, compressed\_graph and adjacency matrix.

The type aliases are defined by either a function specialization for the underlying graph container, or a refinement of one of those types (e.g. an iterator of a range). Table \ref{tab:graph_func} describes the functions in more detail.

\tcode{graph_value(g)}, \tcode{vertex_value(g,u)} and \tcode{edge_value(g,uv)} can be optionally implemented, depending on whether the graph container supports values on the graph, vertex and edge types.

\begin{table}[h!]
\begin{center}
\resizebox{\textwidth}{!}
{\begin{tabular}{l l L{1.5cm}}
\hline
    \textbf{Type Alias} & \textbf{Definition} & \textbf{Comment} \\
\hline
    \tcode{graph_reference_t<G>} & \tcode{add_lvalue_reference<G>} & \\
    \tcode{graph_value_t<G>} & \tcode{decltype(graph_value(g))} & optional \\
\hline
    \tcode{vertex_range_t<G>} & \tcode{decltype(vertices(g))} & \\    
    \tcode{vertex_iterator_t<G>} & \tcode{iterator_t<vertex_range_t<G>>} & \\    
    \tcode{vertex_t<G>} & \tcode{range_value_t<vertex_range_t<G>>} & \\    
    \tcode{vertex_reference_t<G>} & \tcode{range_reference_t<vertex_range_t<G>>} & \\    
    \tcode{vertex_id_t<G>} & \tcode{decltype(vertex_id(g))} & \\    
    \tcode{vertex_value_t<G>} & \tcode{decltype(vertex_value(g))} & optional \\
\hline
    \tcode{vertex_edge_range_t<G>} & \tcode{decltype(edges(g,u))} & \\    
    \tcode{vertex_edge_iterator_t<G>} & \tcode{iterator_t<vertex_edge_range_t<G>>} & \\    
    \tcode{edge_t<G>} & \tcode{range_value_t<vertex_edge_range_t<G>>} & \\    
    \tcode{edge_reference_t<G>} & \tcode{range_reference_t<vertex_edge_range_t<G>>} & \\    
    \tcode{edge_value_t<G>} & \tcode{decltype(edge_value(g))} & optional \\
\hdashline
    \multicolumn{3}{c}{The following is only available when the optional \tcode{source_id(g,uv)} is defined for the edge} \\
\hdashline
    \tcode{edge_id_t<G>} & \tcode{edge_descriptor<vertex_id_t<G>,true,void,void>} & \\    
\hline
    \tcode{partition_id_t<G>} & \tcode{decltype(partition_id(g,u))} & optional \\
    \tcode{partition_vertex_range_t<G>} & \tcode{vertices(g,pid)} & optional \\    
    \tcode{partition_edge_range_t<G>} & \tcode{edges(g,u,pid)} & optional \\    
\hline
\end{tabular}}
\caption{Graph Container Interface Type Aliases}
\label{tab:graph_type}
\end{center}
\end{table}

There is no contiguous requirement for \tcode{vertex_id} from one partition to the next, though
in practice they will often be assigned contiguously. Gaps in \tcode{vertex_id}s between partitions should be allowed.

\section{Classes and Structs}

The \tcode{graph_error} exception class is available, inherited from \tcode{runtime_error}. While any function may use it, it 
is only anticipated to be used by the \tcode{load} functions at this time.
No additional functionality is added beyond that provided by \tcode{runtime_error}.

\section{Functions}

Table \ref{tab:graph_func} summarizes the functions in the Graph Container Interface. These are the primitive functions used to access an adacency graph, no matter its internal design and organization. Thus, it is designed to be able to reflect all forms of adjacency graphs including a vector of lists, CSR-based graph and adjacency matrix.


\begin{table}[h!]
\begin{center}
\resizebox{\textwidth}{!}
{\begin{tabular}{l l p{1.5cm} L{7.0cm}}
\hline
    \textbf{Function} & \textbf{Return Type} & \textbf{Complexity} & \textbf{Default Implementation} \\
\hline
    \tcode{graph_value(g)} & \tcode{graph_value_t<G>} & constant & n/a, optional \\
\hdashline
    \tcode{partition_count(g)} & \tcode{vertex_id_t<G>} & constant & 1 \\
\hline
    \tcode{vertices(g)} & \tcode{vertex_range_t<G>} & constant & \tcode{g} if \tcode{random_access_range<G>}, n/a otherwise \\
    \tcode{num_vertices(g)} & \tcode{integral} & constant & \tcode{size(vertices(g))} \\
    \tcode{find_vertex(g,uid)} & \tcode{vertex_iterator_t<G>} & constant & \tcode{begin(vertices(g)) + uid} \\
    & & & if \tcode{random_access_range<vertex_range_t<G>>}  \\
    \tcode{vertex_id(g,ui)} & \tcode{vetex_id_t<G>} & constant & \tcode{(size_t)(ui - begin(vertices(g)))} \\
    & & & Override to define a different \tcode{vertex_id_t<G>} type (e.g. int32\_t). \\
    \tcode{vertex_value(g,u)} & \tcode{vertex_value_t<G>} & constant & n/a, optional \\
    \tcode{vertex_value(g,uid)} & \tcode{vertex_value_t<G>} & constant & \tcode{vertex_value(g,*find_vertex(g,uid))}, optional \\
    \tcode{degree(g,u)} & \tcode{integral} & constant & \tcode{size(edges(g,u))} if \tcode{sized_range<vertex_edge_range_t<G>>} \\
    \tcode{degree(g,uid)} & \tcode{integral} & constant & \tcode{size(edges(g,uid))} if \tcode{sized_range<vertex_edge_range_t<G>>} \\
\hdashline
    \tcode{partition_id(g,u)} & \tcode{partition_id_t<G>} & constant & 0 \\
    \tcode{partition_id(g,uid)} & \tcode{partition_id_t<G>} & constant & \tcode{partition_id(g,*find_vertex(g,uid))} \\
    \tcode{vertices(g,pid)} & \tcode{partition_vertex_range_t<G>} & constant & \tcode{vertices(g)} \\
    \tcode{num_vertices(g,pid)} & \tcode{integral} & constant & \tcode{size(vertices(g))}  \\
\hline
    \tcode{edges(g,u)} & \tcode{vertex_edge_range_t<G>} & constant & \tcode{u} if \tcode{forward range<vertex_t<G>>}, n/a otherwise \\
    \tcode{edges(g,uid)} & \tcode{vertex_edge_range_t<G>} & constant & \tcode{edges(g,*find_vertex(g,uid))} \\
    \tcode{target_id(g,uv)} & \tcode{vertex_id_t<G>} & constant & (see below) \\
    \tcode{target(g,uv)} & \tcode{vertex_t<G>} & constant & \tcode{*(begin(vertices(g)) + target_id(g, uv))} if \tcode{random_access_range<vertex_range_t<G>> \&\& integral<target_id(g,uv)>} \\
    \tcode{edge_value(g,uv)} & \tcode{edge_value_t<G>} & constant & \tcode{uv} if \tcode{forward_range<vertex_t<G>>}, n/a otherwise, optional \\
    \tcode{find_vertex_edge(g,u,vid)} & \tcode{vertex_edge_t<G>} & linear & \tcode{find(edges(g,u), [](uv) {target_id(g,uv)==vid;\})}} \\
    \tcode{find_vertex_edge(g,uid,vid)} & \tcode{vertex_edge_t<G>} & linear & \tcode{find_vertex_edge(g,*find_vertex(g,uid),vid)} \\
    \tcode{contains_edge(g,uid,vid)} & \tcode{bool} & constant & \tcode{uid < size(vertices(g)) \&\& vid < size(vertices(g))} if \tcode{is_adjacency_matrix_v<G>}.\\
    & & linear & \tcode{find_vertex_edge(g,uid) != end(edges(g,uid))} otherwise. \\
\hdashline
    \tcode{edges(g,u,pid)} & \tcode{partition_edge_range_t<G>} & linear & \tcode{edges(g,u)} \\
    \tcode{edges(g,uid,pid)} & \tcode{partition_edge_range_t<G>} & linear & \tcode{edges(g,uid)} \\
\hdashline
    \multicolumn{4}{c}{The following are only available when the optional \tcode{source_id(g,uv)} is defined for the edge} \\
\hdashline
    \tcode{source_id(g,uv)} & \tcode{vertex_id_t<G>} & constant & n/a, optional \\
    \tcode{source(g,uv)} & \tcode{vertex_t<G>} & constant & \tcode{*(begin(vertices(g)) + source_id(g,uv))} if \tcode{random_access_range<vertex_range_t<G>> \&\& integral<target_id(g,uv)>} \\
    \tcode{edge_id(g,uv)} & \tcode{edge_id_t<G>} & constant & \tcode{edge_descriptor<vertex_id_t<G>,true,void,void>\{source_id(g,uv),target_id(g,uv)\}} \\
\hline
\end{tabular}}
\caption{Graph Container Interface Functions}
\label{tab:graph_func}
\end{center}
\end{table}

When the graph matches the pattern \tcode{forward_range<forward_range<integral>>} or \tcode{forward_range<forward_range<tuple<integral,...>>>},
the default implementation for \tcode{target_id(g,uv)} will return the \tcode{integral}. Additionally, if the caller does not override 
\tcode{vertex_id(g,ui)}, the \tcode{integral} value will define the \tcode{vertex_id_t<G>} type.

Functions that have n/a for their Default Implementation must be defined by the author of a Graph Container implementation. 

Value functions (\tcode{graph_value(g)}, \tcode{vertex_value(g,u)} and \tcode{edge_value(g,uv)}) can be optionally implemented, 
depending on whether the graph container supports values on the graph, vertex and edge types. They return a single value and can 
be scaler, struct, class, union, or tuple. These are abstract types used by the GVF, VVF and EVF function objects to retrieve
values used by algorithms. As such it's valid to return the "enclosing" owning class (graph, vertex or edge), or some other
embedded value in those objects.

\tcode{find_vertex(g,uid)} is constant complexity because all algorithms in this proposal require that \tcode{vertex_range_t<G>} is a random access range. 

If the concept requirements for the default implementation aren't met by the graph container the function will need to be overridden.

\subsection{Determining the Vertex Id Type}
To determine the type for \tcode{vertex_id_t<G>} the following steps are taken, in order, to determine its type.
\begin{enumerate}
    \item Use the type returned by \tcode{vertex_id(g,ui)} when overridden for a graph.
    \item When the graph matches the pattern \tcode{forward_range<forward_range<integral>>} or \tcode{forward_range<forward_range<tuple<integral,...>>>},
          use the \tcode{integral} type specified.
    \item Use \tcode{size_t} in all other cases.
\end{enumerate}

\tcode{vertex_id_t<G>} is defined by the type returned by \tcode{vertex_id(g)} and it defaults to the difference\_type of the underlying container used for vertices (e.g int64\_t for 64-bit systems). 
This is sufficient for all situations. However, there are often space and performance advantages if a smaller type is used, such as int32\_t or even int16\_t. It is recommended to consider overriding 
this function for optimal results, assuring that it is also large enough for the number of possible vertices and edges in the application. It will also need to be overridden if the implementation doesn't 
expose the vertices as a range.

\section{Unipartite, Bipartite and Multipartite Graph Representation}

\tcode{partition_count(g)} returns the number of partitions, or partiteness, of the graph. It has a range of 1 to n, where 1 identifies 
a unipartite graph, 2 is a bipartite graph, and a value of 2 or more can be considered a multipartite graph. 

If a graph data structure doesn't support partitions then it is unipartite with one partition and partite functions will reflect that. 
For instance, \tcode{partition_count(g)} returns a value of 1, and \tcode{vertices(g,0)} (vertices in the first partition) will return a range that includes all vertices in the graph.

A partition identifies a type of a vertex, where the vertex value types are assumed to be uniform in each partition. This creates a dilemma because 
the existing \tcode{vertex_value(g,u)} returns a single type based template parameter for the vertex value type. Supporting 
multiple types can be addressed in different ways using C++ features. The key to remember is that the actual value used by algorithms is done
by calling a function object that retrieves the value to be used. That function is specific to the graph data structure, using the partition to 
determine how to get the appropriate value.
\begin{itemize}
    \item
          \tcode{std::variant}: The lambda returns the appropriate variant value based on the partition.
    \item
          Base class pointer: The lambda can call a member function to return the value based on the partition.
    \item
          \tcode{void*}: The lambda can cast the pointer to a concrete type based on the partition, and
          then return the appropriate value.
\end{itemize}

\tcode{edges(g,uid,pid)} and \tcode{edges(g,ui,pid)} filter the edges where the target is in the partition \tcode{pid} passed.
This isn't needed for bipartite graphs.

\section{Loading Graph Data}

The \tcode{load} functions are used to load vertex and edge data into a graph. They may throw a \tcode{graph_error} exception.

All graph data structures need to implement \tcode{load_graph}, \tcode{load_vertices} and \tcode{load_edges}. 
Whether \tcode{load_vertices} or \tcode{load_edges} can be called multiple times, or after \tcode{load_graph} is called, is dependent on the underlying graph data structure.
\tcode{load_partition} only needs to be implemented if a graph supports partitions.

Projections are used to convert values in the input range to the expected copyable type. In the following \tcode{load_vertices} 
prototype, \tcode{vproj(ranges::range_value_t<VRng>&)} $\rightarrow$ \tcode{vertex_descriptor<vertex_id_t<G>, vertex_value_t<G>>}.
If there is no vertex value stored in the graph then \tcode{vertex_value_t<G>} will be \tcode{void} and the resulting 
\tcode{vertex_descriptor} will have a single id member. If \tcode{vproj(ranges::range_value_t<VRng>&)} is the same as
\tcode{vertex_descriptor<vertex_id_t<G>, vertex_value_t<G>>} then \tcode{VProj = identity} can be used.

\begin{lstlisting}
template <adjacency_list G, ranges::forward_range VRng, class VProj = identity>
requires copyable_vertex<invoke_result<VProj, ranges::range_value_t<VRng>>,
                         vertex_id_t<G>, vertex_value_t<G>>
constexpr void load_vertices(G&, const VRng& vrng, VProj vproj);
\end{lstlisting}

The same pattern is applied using \tcode{ERng} and \tcode{EProj} for edges.

For graphs with vertex values, \tcode{load_vertices} should be called before \tcode{load_edges}.

Whether \tcode{load_vertices} or \tcode{load_edges} can be called multiple times is graph-dependent.

For graphs with partititions, \tcode{load_partition} must be called to load vertices for each 
partition \tcode{pid}. \tcode{pid} values must be contiguous and their vertices should be loaded
contiguously. \tcode{empty(vrng)} may be empty if there are no vertices in the partition. 

\begin{table}[h!]
\begin{center}
\resizebox{\textwidth}{!}
{\begin{tabular}{l l p{1.5cm} L{7.0cm}}
\hline
    \textbf{Function} & \textbf{Return Type} & \textbf{Complexity} & \textbf{Default Implementation} \\
\hline
    \tcode{load_graph(g, erng, vrng, eproj=identity(), vproj=identity())} & void & V + E & n/a \\
    \tcode{load_vertices(g, vrng, vproj=identity())} & void & V & n/a \\
    \tcode{load_partition(g, pid, vrng, vproj=identity())} & void & V(p) & \tcode{load_vertices} is called if partitions are not supported; there will be a single partition. \\
    \tcode{load_edges(g, erng, eproj=identity(), vertex_count=0)} & void & E & n/a \\
\hline
\end{tabular}}
\caption{Graph Load Functions}
\label{tab:load_func}
\end{center}
\end{table}

\subsection{Edgelists}
An edgelist is a range of \tcode{edge_descriptor} values, used by some algorithms. Examples include the \tcode{edgelist} adjacency 
list view, or possibly a transform of an existing range that projects \tcode{edge_descriptor} values. There is no concrete data
structure in this proposal like there is for an adjacency list.

\section{Using Existing Graph Data Structures}
Reasonable defaults have been defined for the GCI functions to minimize the amount of work
needed to adapt an existing graph data structure to be used by the views and algorithms.

There are two cases supported. The first is for the use of standard containers to define the graph and the other
is for a broader set of more complicated implementations.

This is described in more detail in the paper for Graph Library Container Implementations.
