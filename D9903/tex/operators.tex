\clearpage

\phil{Can any of these functions use \tcode{adjacency_list} or \tcode{basic_adjacency_list} instead of \tcode{index_adjancy_list}?}

\section{Degree}
Degree of a vertex in graph theory refers to the number of edges that are incident to the vertex. Although the Graph Container Interface provides customization point interface to access a degree of an individual vertex, a degree operator allows one to build a range of degrees of all vertices within a graph.

\begin{table}[h]
\setcellgapes{3pt}
\makegapedcells
\centering
\begin{tabular}{|P{0.30\textwidth}|P{0.25\textwidth}|P{0.25\textwidth}|}
\hline
      \multirowcell{2}{
            \textbf{Complexity} \\
            $\mathcal{O}(|V|)$
            }
      & \textbf{Throws?} No & \textbf{Cycles?} Yes \\
      & \textbf{Multi-edge?} Yes & \textbf{Directed?} Yes\\
\hline
\end{tabular}
%\caption{Algorithm Example}
\label{tab:degree_operator}
\end{table}

{\small
      \lstinputlisting{D9903/src/degree.hpp}
}
\muhammad{Support other graph types besides \tcode{index_adjacency_list} as well?}
\begin{itemdescr}
      \pnum\preconditions
      \begin{itemize}
            \item
                  \lstinline{size(degrees_per_vertex) >= size(vertices(graph))}.
      \end{itemize}
      \pnum\effects 
      \begin{itemize}
            \item
                  Output iterator \lstinline{degrees_per_vertex} contains the degree of each vertex in the graph, accessible through 
                  \lstinline{degrees_per_vertex[uid]}, where \lstinline{uid} is the \lstinline{vertex_id}.
      \end{itemize}
      %\pnum\complexity $\mathcal{O}(|V|)$ where $V$ is the number of vertices in the graph.
\end{itemdescr}

\section{Sort}
Sort operator comes in many different variants, allows to sort the input graph based on the source vertex, target vertex, the degree of the vertex or by the weight on the edges. Also relevant, \lstinline{topological_sort}, which returns a nonunique permutation of the vertices of a directed acyclic graph such that an edge from u to v implies that u appears before v in the topological sort order.

\phil{Graph parameters should be \tcode{const&}, not \tcode{&&}, the the graph is going to change}

The following algorithm summary represents all the Sort operators.
\begin{table}[h]
\setcellgapes{3pt}
\makegapedcells
\centering
\begin{tabular}{|P{0.30\textwidth}|P{0.25\textwidth}|P{0.25\textwidth}|}
\hline
      \multirowcell{2}{
            \textbf{Complexity} \\
            $\mathcal{O}(N log|N|)$
            }
      & \textbf{Throws?} No & \textbf{Cycles?} Yes \\
      & \textbf{Multi-edge?} Yes & \textbf{Directed?} Yes\\
\hline
\end{tabular}
%\caption{Algorithm Example}
\label{tab:sort_operators}
\end{table}

\phil{Removed comments that said "The same API extends for \lstinline{edgelist}" for all Sort operators. The current 
      view is that an edgelist is the result of a std::ranges::transform() with a value type of an edge\_descriptor. 
      Since it's not an adacency\_list they won't work as-is. Are overloads needed?}

\subsection{Sort by Source Vertex}
{\small
      \lstinputlisting[firstline=1,lastline=2]{D9903/src/sort.hpp}
}
\begin{itemdescr}
      %\pnum\mandates
      %\pnum\preconditions
      \pnum\effects \lstinline{graph}, which is now sorted by the source vertices in the graph.
      %\pnum\result
      %\pnum\returns
      %\pnum\throws
      %\pnum\complexity
      %\pnum\remarks
      %\pnum\errors
\end{itemdescr}
\phil{What's being sorted? The edges? (Not the graph)}
\phil{Is the sort key the source\_id?}

\subsection{Sort by Target Vertex}
{\small
      \lstinputlisting[firstline=4,lastline=5]{D9903/src/sort.hpp}
}
\begin{itemdescr}
      %\pnum\mandates
      %\pnum\preconditions
      \pnum\effects \lstinline{graph}, which is now sorted by the target vertices in the graph.
      %\pnum\result
      %\pnum\returns
      %\pnum\throws
      %\pnum\complexity
      %\pnum\remarks
      %\pnum\errors
\end{itemdescr}

\subsection{Sort by Degree}
{\small
      \lstinputlisting[firstline=7,lastline=8]{D9903/src/sort.hpp}
}
\begin{itemdescr}
      %\pnum\mandates
      %\pnum\preconditions
      \pnum\effects \lstinline{graph}, which is now sorted by the vertex degree.
      %\pnum\result
      %\pnum\returns
      %\pnum\throws
      %\pnum\complexity
      %\pnum\remarks
      %\pnum\errors
\end{itemdescr}

\subsection{Sort by Edge Weight}
{\small
      \lstinputlisting[firstline=10,lastline=12]{D9903/src/sort.hpp}
}
\begin{itemdescr}
      %\pnum\mandates
      %\pnum\preconditions
      \pnum\effects \lstinline{graph}, which is now sorted by the edge weight.
      %\pnum\result
      %\pnum\returns
      %\pnum\throws
      %\pnum\complexity
      %\pnum\remarks
      %\pnum\errors
\end{itemdescr}

\section{Relabel}
Relabels the vertices of the graph according to a given mapping or to integers.

\begin{table}[h]
\setcellgapes{3pt}
\makegapedcells
\centering
\begin{tabular}{|P{0.30\textwidth}|P{0.25\textwidth}|P{0.25\textwidth}|}
\hline
      \multirowcell{2}{
            \textbf{Complexity} \\
            $\mathcal{O}(?)$
            }
      & \textbf{Throws?} No & \textbf{Cycles?} Yes \\
      & \textbf{Multi-edge?} Yes & \textbf{Directed?} Yes\\
\hline
\end{tabular}
%\caption{Algorithm Example}
\label{tab:relabel_map}
\end{table}

\phil{Graph parameters should be \tcode{const &g} or \tcode{&g}, depending on whether the graph will be changed or not.}

{\small
      \lstinputlisting[firstline=1,lastline=2]{D9903/src/relabel.hpp}
}
\begin{itemdescr}
      %\pnum\mandates
      %\pnum\preconditions
      \pnum\effects \lstinline{graph}: the vertices of the graph are relabeled starting based on the mapping provided. \\
      %\pnum\result
      %\pnum\returns
      %\pnum\throws
      %\pnum\complexity
      \pnum\remarks
            \begin{itemize}
                  \item
                  \lstinline{mapping} is a \lstinline{std::map}, a dictionary with old labels as keys and new labels as values. 
                  If a partial mapping is provided (i.e. \lstinline{mapping.size() < size(vertices(graph))}), then only the 
                  specified vertices are relabeled.
            \end{itemize}
      %\pnum\errors
\end{itemdescr}

\begin{table}[h]
\setcellgapes{3pt}
\makegapedcells
\centering
\begin{tabular}{|P{0.30\textwidth}|P{0.25\textwidth}|P{0.25\textwidth}|}
\hline
      \multirowcell{2}{
            \textbf{Complexity} \\
            $\mathcal{O}(?)$
            }
      & \textbf{Throws?} No & \textbf{Cycles?} Yes \\
      & \textbf{Multi-edge?} Yes & \textbf{Directed?} Yes\\
\hline
\end{tabular}
%\caption{Algorithm Example}
\label{tab:relabel_to_int}
\end{table}

{\small
      \lstinputlisting[firstline=4,lastline=5]{D9903/src/relabel.hpp}
}
\begin{itemdescr}
      %\pnum\mandates
      %\pnum\preconditions
      \pnum\effects \lstinline{graph}: the vertices of the graph are relabeled starting from 0 to \lstinline{size(vertices(graph))-1}.
      %\pnum\result
      %\pnum\returns
      %\pnum\throws
      %\pnum\complexity
      %\pnum\remarks
      %\pnum\errors
\end{itemdescr}

\section{Transpose}
Transpose returns a graph with edges reversed. For an adjacency graph, it is obtained by switching the outer range of vertices with an inner range of incidence edges on each vertex.

\begin{table}[h]
\setcellgapes{3pt}
\makegapedcells
\centering
\begin{tabular}{|P{0.30\textwidth}|P{0.25\textwidth}|P{0.25\textwidth}|}
\hline
      \multirowcell{2}{
            \textbf{Complexity} \\
            $\mathcal{O}(?)$
            }
      & \textbf{Throws?} No & \textbf{Cycles?} Yes \\
      & \textbf{Multi-edge?} Yes & \textbf{Directed?} Yes\\
\hline
\end{tabular}
%\caption{Algorithm Example}
\label{tab:transpose_operator}
\end{table}

\phil{Graph parameters should be \tcode{const &g} or \tcode{&g}, depending on whether the graph will be changed or not.}

{\small
      \lstinputlisting[firstline=1,lastline=2]{D9903/src/transpose.hpp}
}
\begin{itemdescr}
      %\pnum\mandates
      %\pnum\preconditions
      \pnum\effects \lstinline{graph}: the graph is transposed such that the edges for each vertex are reversed.
      %\pnum\result
      %\pnum\returns
      %\pnum\throws
      %\pnum\complexity
      %\pnum\remarks
      %\pnum\errors
\end{itemdescr}

\section{Join}
Combining of two graphs based on common vertices. The join operation (or Sparse General Matrix-Matrix Multiplication) is a useful operators for implementing graph algorithms. The result would be a new graph where the edges are determined by the multiplication of the adjacency containers. 

\phil{Graph parameters should be \tcode{const &g} or \tcode{&g}, depending on whether the graph will be changed or not.}

\begin{table}[h]
\setcellgapes{3pt}
\makegapedcells
\centering
\begin{tabular}{|P{0.30\textwidth}|P{0.25\textwidth}|P{0.25\textwidth}|}
\hline
      \multirowcell{2}{
            \textbf{Complexity} \\
            $\mathcal{O}(?)$
            }
      & \textbf{Throws?} No & \textbf{Cycles?} Yes \\
      & \textbf{Multi-edge?} Yes & \textbf{Directed?} Yes\\
\hline
\end{tabular}
%\caption{Algorithm Example}
\label{tab:join_operator}
\end{table}

{\small
      \lstinputlisting[firstline=1,lastline=4]{D9903/src/join.hpp}
}
\phil{Should \tcode{a} and \tcode{b} be \tcode{const&}?}
\begin{itemdescr}
      %\pnum\mandates
      %\pnum\preconditions
      \pnum\effects \lstinline{c}: the resulting graph where the entry (u, v) in the result represents the number of common neighbors between vertex u in Graph A and vertex v in Graph B.
      %\pnum\result
      %\pnum\returns
      %\pnum\throws
      %\pnum\complexity
      %\pnum\remarks
      %\pnum\errors
\end{itemdescr}
