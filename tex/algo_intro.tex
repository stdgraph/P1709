\section{Introduction}

\newcommand{\tablistcommand}{% <-- for eliminating vertical space
                             %     before and after itemize
            \leavevmode\par\vspace{-\baselineskip}
\end{itemize}
\section{Introduction}
                            }
\newcolumntype{P}[1]{p{#1-3\tabcolsep-\arrayrulewidth}}

Basic characteristics of the algorithms shown below are summarized in tables of the following form:
\begin{table}[h]
\setcellgapes{3pt}
\makegapedcells
\centering
\begin{tabular}{|P{0.30\textwidth}|P{0.25\textwidth}|P{0.25\textwidth}|}
\hline
      \multirowcell{2}{
            \textbf{Complexity} \\
            $\mathcal{O}(|E|+|V|)$
            }
      & \textbf{Throws?} No & \textbf{Cycles?} No \\
      & \textbf{Multi-edge?} No & \textbf{Directed?} Yes\\
\hline
\end{tabular}
%\caption{Algorithm Example}
\label{tab:algo_example}
\end{table}

The parts of the table have the following meaning:
\begin{itemize}
      \item \textbf{Complexity} The complexity of the algorithm based on the number of vertices (V) and edges (E).
      \item \textbf{Throws?} Will the algorithm throw at all? If so, look at the \textit{Throws} section after the function prototypes for details.
      \item \textbf{Multi-edge?} Does the algorithm act as expected if more than one edge with the same direction exists between the same two vertices?
      \item \textbf{Cycles?} Does the algorithm act act as expected if a vertex (or edge) is part of a cycle?
      \item \textbf{Directed?} Is the algorithm only for directed graphs, or can it also be used for undirected graphs that have complimentary
                               edges, with different directions, between two vertices.
\end{itemize}

\phil{The Directed? section needs work.}
