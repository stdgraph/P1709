
\section{Graph Container Interface}
The Graph Container Interface defines the primitive concepts, traits, types and functions used to define and access an adacency graph, no matter its internal design and organization. Thus, it is designed to reflect all forms of adjacency graphs including a vector of lists, CSR graph and adjacency matrix, whether they are in the standard or external to the standard.

All algorithms in this proposal require that vertices are stored in random access containers and that \tcode{vertex_id_t<G>} is integral, and it is assumed that all future algorithm proposals will also have the same requirements. 

The Graph Container Interface is designed to support a wider scope of graph containers than required by the views and algorithms in this proposal. This enables for future growth of the graph data model (e.g. incoming edges on a vertex), or as a framework for graph implementations outside of the standard. For instance, existing implementations may have requirements that cause them to define features with looser constraints, such as sparse vertex\_ids, non-integral vertex\_ids, or storing vertices in associative bi-directional containers (e.g. std::map or std::unordered\_map). Such features require specialized implementations for views and algorithms. The performance for such algorithms will be sub-optimal, but is preferrable to run them on the existing container rather than loading the graph into a high-performance graph container and then running the algorithm on it, where the loading time can far outweigh the time to run the sub-optimal algorithm. To achieve this, care has been taken to make sure that the use of concepts chosen is appropriate for algorithm, view and container.

\section{Concepts}
Table \ref{tab:graph_concepts} summarizes the concepts in the Graph Container Interface, allowing views and algorithms to verify a graph implementation has the expected requirements for an \tcode{adjacency_list} or \tcode{sourced_adjacency_list}.

Sourced edges have a source\_id on them in addition to a target\_id. A \tcode{sourced_adjacency_list} has sourced edges.

% is_sourced_edge and is_sourced_edge_v structs are not included the table for brevity. We may want to add them in the future for completeness.
\begin{table}[h!]
\begin{center}
%\resizebox{\textwidth}{!}
{\begin{tabular}{l L{12.0cm}}
\hline
    \textbf{Concept} & \textbf{Definition} \\
\hline
    \tcode{vertex_range<G>} & \tcode{vertices(g)} returns a sized, forward\_range; \tcode{vertex_id(g,ui)} exists\\
    \tcode{targeted_edge<G>} & \tcode{target_id(g,uv)} and \tcode{target(g,uv)} exist\\
    \tcode{sourced_edge<G>} & \tcode{source_id(g,uv)} and \tcode{source(g,uv)} exist\\
    \tcode{adjacency_list<G>} & \tcode{vertex_range<G>} and \tcode{targeted_edge<G,edge<G>>} and \tcode{edges(g,_)} functions return a forward\_range\\
    \tcode{sourced_adjacency_list<G>} & \tcode{adjacency_list<G>} and \tcode{sourced_edge<G, edge_t<G>>} and     \tcode{edge_id(g,uv)} exists \\
\hline
    \tcode{copyable_vertex<T,VId,VV>} & \tcode{convertible_to<T, copyable_vertex_t<VId, VV>>} \\
    \tcode{copyable_edge<T,Vid,EV>} & \tcode{convertible_to<T, copyable_edge_t<VId, EV>>} \\
\hline
\end{tabular}}
\caption{Graph Container Interface Concepts}
\label{tab:graph_concepts}
\end{center}
\end{table}

\section{Traits}
Table \ref{tab:graph_traits} summarizes the type traits in the Graph Container Interface, allowing views and algorithms to query the graph's characteristics.

\begin{table}[h!]
\begin{center}
%\resizebox{\textwidth}{!}
{\begin{tabular}{l l L{7.0cm}}
\hline
    \textbf{Trait} & \textbf{Type} & \textbf{Comment} \\
\hline
    \tcode{has_degree<G>} & concept & Is the \tcode{degree(g,u)} function available? \\
    \tcode{has_find_vertex<G>} & concept & Are the \tcode{find_vertex(g,_)} functions available? \\
    \tcode{has_find_vertex_edge<G>} & concept & Are the \tcode{find_vertex_edge(g,_)} functions available?\\
    \tcode{has_contains_edge<G>} & concept & Is the \tcode{contains_edge(g,uid,vid)} function available?\\
\hline
    \tcode{define_unordered_edge<G,E> : false_type} & struct & Specialize for edge implementation to derive from \tcode{true_type} for unordered edges \\
    \tcode{is_unordered_edge<G,E>} & struct & \tcode{conjunction<define_unordered_edge<E>, is_sourced_edge<G, E>>} \\
    \tcode{is_unordered_edge_v<G,E>} & type alias & \\
    \tcode{unordered_edge<G,E>} & concept & \\
\hline
    \tcode{is_ordered_edge<G,E>} & struct & \tcode{negation<is_unordered_edge<G,E>>} \\
    \tcode{is_ordered_edge_v<G,E>} & type alias & \\
    \tcode{ordered_edge<G,E>} & concept & \\
\hline
    \tcode{define_adjacency_matrix<G> : false_type} & struct & Specialize for graph implementation to derive from \tcode{true_type} for edges stored as a square 2-dimensional array \\
    \tcode{is_adjacency_matrix<G>} & struct & \\
    \tcode{is_adjacency_matrix_v<G>} & type alias & \\
    \tcode{adjacency_matrix<G>} & concept & \\
\hline
\end{tabular}}
\caption{Graph Container Interface Type Traits}
\label{tab:graph_traits}
\end{center}
\end{table}


\section{Types}
Table \ref{tab:graph_type} summarizes the type aliases in the Graph Container Interface. These are the types used to define the objects in a graph container, no matter its internal design and organization. Thus, it is designed to be able to reflect all forms of adjacency graphs including a vector of lists, CSR graph and adjacency matrix.

The type aliases are defined by either a function specialization for the underlying graph container, or a refinement of one of those types (e.g. an iterator of a range). Table \ref{tab:graph_func} describes the functions in more detail.

\tcode{graph_value(g)}, \tcode{vertex_value(g,u)} and \tcode{edge_value(g,uv)} can be optionally implemented, depending on whether the graph container supports values on the graph, vertex and edge types.

\begin{table}[h!]
\begin{center}
\resizebox{\textwidth}{!}
{\begin{tabular}{l l L{1.5cm}}
\hline
    \textbf{Type Alias} & \textbf{Definition} & \textbf{Comment} \\
\hline
    \tcode{graph_reference_t<G>} & \tcode{add_lvalue_reference<G>} & \\
    \tcode{graph_value_t<G>} & \tcode{decltype(graph_value(g))} & optional \\
\hline
    \tcode{vertex_range_t<G>} & \tcode{decltype(vertices(g))} & \\    
    \tcode{vertex_iterator_t<G>} & \tcode{iterator_t<vertex_range_t<G>>} & \\    
    \tcode{vertex_t<G>} & \tcode{range_value_t<vertex_range_t<G>>} & \\    
    \tcode{vertex_reference_t<G>} & \tcode{range_reference_t<vertex_range_t<G>>} & \\    
    \tcode{vertex_id_t<G>} & \tcode{decltype(vertex_id(g))} & \\    
    \tcode{vertex_value_t<G>} & \tcode{decltype(vertex_value(g))} & optional \\
\hline
    \tcode{vertex_edge_range_t<G>} & \tcode{decltype(edges(g,u))} & \\    
    \tcode{vertex_edge_iterator_t<G>} & \tcode{iterator_t<vertex_edge_range_t<G>>} & \\    
    \tcode{edge_t<G>} & \tcode{range_value_t<vertex_edge_range_t<G>>} & \\    
    \tcode{edge_reference_t<G>} & \tcode{range_reference_t<vertex_edge_range_t<G>>} & \\    
    \tcode{edge_value_t<G>} & \tcode{decltype(edge_value(g))} & optional \\
\hdashline
    \multicolumn{3}{c}{The following is only available when the optional \tcode{source_id(g,uv)} is defined for the edge} \\
\hdashline
    \tcode{edge_id_t<G>} & \tcode{decltype(pair(source_id(g,uv),target_id(g,uv)))} & \\    
\hline
\end{tabular}}
\caption{Graph Container Interface Type Aliases}
\label{tab:graph_type}
\end{center}
\end{table}


\section{Functions}

\phil{The functions in the Graph Container Interface are semi-stable. New functions are not expected, but overloads may be added or removed for different combinations of 
vertex\_id and references as we refine our use cases.}

Table \ref{tab:graph_func} summarizes the functions in the Graph Container Interface. These are the primitive functions used to access an adacency graph, no matter its internal design and organization. Thus, it is designed to be able to reflect all forms of adjacency graphs including a vector of lists, CSR graph and adjacency matrix.


\begin{table}[h!]
\begin{center}
\resizebox{\textwidth}{!}
{\begin{tabular}{l l p{1.5cm} L{7.0cm}}
\hline
    \textbf{Function} & \textbf{Return Type} & \textbf{Complexity} & \textbf{Default Implementation} \\
\hline
    \tcode{graph_value(g)} & \tcode{graph_value_t<G>} & constant & n/a, optional \\
\hline
    \tcode{vertices(g)} & \tcode{vetex_range_t<G>} & constant & n/a \\
    \tcode{vertex_id(g,ui)} & \tcode{vetex_id_t<G>} & constant & \tcode{ui - begin(vertices(g))} \\
    & & & Override to define a different \tcode{vertex_id_t<G>} type (e.g. int32\_t). \\
    \tcode{vertex_value(g,u)} & \tcode{vertex_value_t<G>} & constant & n/a, optional \\
    \tcode{degree(g,u)} & \tcode{integral} & constant & \tcode{size(edges(g,u))} if \tcode{sized_range<vertex_edge_range_t<G>>} \\
    \tcode{find_vertex(g,uid)} & \tcode{vertex_iterator_t<G>} & constant & \tcode{begin(vertices(g)) + uid} \\
    & & & if \tcode{random_access_range<vertex_range_t<G>>}  \\
\hline
    \tcode{edges(g,u)} & \tcode{vertex_edge_range_t<G>} & constant & n/a \\
    \tcode{edges(g,uid)} & \tcode{vertex_edge_range_t<G>} & constant & \tcode{edges(g,*find_vertex(g,uid))} \\
    \tcode{target_id(g,uv)} & \tcode{vertex_id_t<G>} & constant & n/a \\
    \tcode{target(g,uv)} & \tcode{vertex_t<G>} & constant & \tcode{*(begin(vertices(g)) + target_id(g, uv))} if \tcode{random_access_range<vertex_range_t<G>> \&\& integral<target_id(g,uv)>} \\
    \tcode{edge_value(g,uv)} & \tcode{edge_value_t<G>} & constant & n/a, optional \\
    \tcode{find_vertex_edge(g,u,vid)} & \tcode{vertex_edge_t<G>} & linear & \tcode{find(edges(g,u), [](uv) {target_id(g,uv)==vid;\})}} \\
    \tcode{find_vertex_edge(g,uid,vid)} & \tcode{vertex_edge_t<G>} & linear & \tcode{find_vertex_edge(g,*find_vertex(g,uid),vid)} \\
    \tcode{contains_edge(g,uid,vid)} & \tcode{bool} & constant & \tcode{uid < size(vertices(g)) \&\& vid < size(vertices(g))} if \tcode{is_adjacency_matrix_v<G>}.\\
    & & linear & \tcode{find_vertex_edge(g,uid) != end(edges(g,uid))} otherwise. \\
\hdashline
    \multicolumn{4}{c}{The following are only available when the optional \tcode{source_id(g,uv)} is defined for the edge} \\
\hdashline
    \tcode{source_id(g,uv)} & \tcode{vertex_id_t<G>} & constant & n/a, optional \\
    \tcode{source(g,uv)} & \tcode{vertex_t<G>} & constant & \tcode{*(begin(vertices(g)) + source_id(g,uv))} if \tcode{random_access_range<vertex_range_t<G>> \&\& integral<target_id(g,uv)>} \\
    \tcode{edge_id(g,uv)} & \tcode{edge_id_t<G>} & constant & \tcode{pair(source_id(g,uv),target_id(g,uv))} \\
\hline
\end{tabular}}
\caption{Graph Container Interface Functions}
\label{tab:graph_func}
\end{center}
\end{table}

Functions that have n/a for their Default Implementation must be defined by the author of a Graph Container implementation. \tcode{graph_value(g)}, \tcode{vertex_value(g,u)} and \tcode{edge_value(g,uv)} can be optionally implemented, depending on whether the graph container supports values on the graph, vertex and edge types.

\tcode{vertex_id_t<G>} is defined by the type returned by \tcode{vertex_id(g)} and it defaults to the difference\_type of the underlying container used for vertices (e.g int64\_t for 64-bit systems). This is sufficient for all situations. However, there are often space and performance advantages if a smaller type is used, such as int32\_t or even int16\_t. It is recommended to consider overriding this function for optimal results, assuring that it is also large enough for the number of possible vertices and edges in the application. It will also need to be overridden if the implementation doesn't expose the vertices as a range.

\tcode{find_vertex(g,uid)} is constant complexity because all algorithms in this proposal require that \tcode{vertex_range_t<G>} is a random access range. 

If the concept requirements for the default implementation aren't met by the graph container the function will need to be overridden.

Edgelists are assumed to be either be an edgelist view of an adjacency graph, or a standard range with source\_id and target\_id values. There is no need for additional functions when a range is used.

\section{Graph Container Implementation}

\andrew{In think we should have a different name than csr\_graph since: ``CSR'' is a linear algebra concept; a graph does not have ``rows''; the same structure also represents ``CSC''; not calling it csr may avoid some of the discussions we had with linear algebra people.  This has the potential of bogging down the entire proposal.}

\section{Relationship between Graphs and Sparse Matrices}


\section{csr\_graph Graph Container}
The csr\_graph is a high-performance graph container that uses \href{https://en.wikipedia.org/wiki/Sparse_matrix#Compressed_sparse_row_\%28CSR\%2C_CRS_or_Yale_format\%29}{Compressed Sparse Row} format to store it's vertices, edges and associated values. Once constructed, vertices and edges cannot be added or deleted but values on vertices and edges can be modified.
\\

The following listing shows the prototype for the \tcode{csr_graph}. A fuller description, including its constructors, can be found in the \tcode{<csr_graph} header section.

Only the constructors, destructor and assignment operators for \tcode{csr_graph} are public. No other member functions or types are exposed. All other types are only accessible through the types and functions in the Graph Container Interface.

When a value type template argument (EV, VV, GV) is void then no extra overhead is incurred for it. The selection of the VId template argument impacts the inter storage requirements. If you have a small graph where the number of vertices is less than 256, and the number of edges is less than 256, then a \tcode{uint8_t} would be sufficient.

\begin{lstlisting}
template <class    EV     = void,     // Edge Value type
          class    VV     = void,     // Vertex Value type
          class    GV     = void,     // Graph Value type
          integral VId    = uint32_t, // vertex id type
          integral EIndex = uint32_t, // edge index type
          class    Alloc  = allocator<uint32_t>> // for internal containers
class csr_graph;
\end{lstlisting}

\section{csr\_partite\_graph Graph Container (In Design)}
\phil{This is experimental}

The \tcode{csr_partite_graph} extends \tcode{csr_graph} to have multiple partitions, where each partition defines a different value type for the vertex and edge. The same template arguments are used, but it also expects that the VV and EV arguments are \tcode{std::variant}, and the number of types in each is the same. The number of types in the variants define the number of partitions. The edge types apply to the outgoing edges of the vertices in the same partition. \tcode{std::monostate} can be used if no value is needed for a vertex or edge in a partition.

Example usage
\begin{lstlisting}
using VV = std::variant<int,double,bool>;
using EV = std::variant<int,int,std::monostate>; // no outgoing edges in the final partition
using G  = csr_partite_graph<EV, VV>;
G g = ...; // construct g with data
for(size_t p = 0; p < partition_size(g); ++p) {
  for(auto&& [uid,u] : partition(g,p)) {
    for(auto&& [vid,uv] : incidence(g,u)) {
       // do interesting things with uv
    }
  }
}
\end{lstlisting}
