

\definecolor{light-gray}{gray}{0.95}
\definecolor{medium-gray}{gray}{0.33}

\definecolor{bs_keyword}{HTML}{990055} % {D33682} 
\definecolor{bs_name_function}{HTML}{0077AA} % {268BD2}
\definecolor{bs_comment_single}{HTML}{708090} % {2AA198}
\definecolor{bs_name}{HTML}{0077AA} % {2AA198}
\definecolor{bs_number}{HTML}{000000}
\definecolor{bs_string}{HTML}{A67F59}
\definecolor{bs_highlight_background}{HTML}{F2F2F2}
\definecolor{bs_h1_h2_h3}{HTML}{005A9C}
\definecolor{bs_a_href}{HTML}{034575}
\definecolor{xcomment}{HTML}{0000A0}


%% Customize code formatting
%% Note that P2300 uses Menlo, Consolas, "DejaVu Sans Mono", Monaco, monospace;
%% TODO (maybe)
%% To use with XeLaTeX:
%% \usepackage{fontspec}
%% \newfontfamily{\lstsansserif}[Scale=.85]{Menlo}
%% \lstset{basicstyle=\lstsansserif}
%% To handle numbers:
%% cf. https://tex.stackexchange.com/questions/34896/coloring-digits-with-the-listings-package
\usepackage[final]{listings}
\lstset
{
  language=[11]C++,
  backgroundcolor=\color{bs_highlight_background},
  basicstyle=\small\ttfamily,
  breaklines=true,
  columns=fullflexible,
  commentstyle=\itshape\color{bs_comment_single},
  frame=single,
  framerule=0pt,
  identifierstyle=\color{bs_name_function},
%  keepspaces=true,
  keywordstyle=\color{bs_keyword},
  showstringspaces=false,
  stringstyle=\color{bs_string},
  texcl=true,
  xleftmargin=1em,
}

%% Keywords for C++20
\lstset{
  morekeywords = {
    char8_t,
    concept,
    consteval,
    co_await,
    co_return,
    co_yield,
    requires,
    import,
    module
  }
}

%% \lstset{
%%   emph = { % [20]
%%     import,
%%     module
%%   }
%% }

\usepackage{array}
\newcolumntype{L}[1]{>{\raggedright\arraybackslash}p{#1}}
\newcolumntype{R}[1]{>{\raggedleft\arraybackslash}p{#1}}

\usepackage{arydshln}
\usepackage{graphicx}
\graphicspath{{figs/}}

\usepackage{subcaption}

%% Shortcut for inline code
\newcommand{\tcode}[1]{%
  \lstinline[breaklines=true,columns=fullflexible]{#1}
}

% \usepackage{underscore}   % remove special status of '_' in ordinary text
%\usepackage{parskip}

\newcommand{\xcomment}[2]{{\color{xcomment}[{\textsc{#1:}} \textsf{#2}]}}
% \newcommand{\xcomment}[2]{}
\newcommand{\phil}[1]{\xcomment{Phil}{#1}}
\newcommand{\andrew}[1]{\xcomment{Andrew}{#1}}
\newcommand{\kevin}[1]{\xcomment{Kevin}{#1}}
\newcommand{\muhammad}[1]{\xcomment{Muhammad}{#1}}
