

\definecolor{light-gray}{gray}{0.95}
\definecolor{medium-gray}{gray}{0.33}

\definecolor{bs_keyword}{HTML}{990055} % {D33682} 
\definecolor{bs_name_function}{HTML}{0077AA} % {268BD2}
\definecolor{bs_comment_single}{HTML}{708090} % {2AA198}
\definecolor{bs_name}{HTML}{0077AA} % {2AA198}
\definecolor{bs_number}{HTML}{000000}
\definecolor{bs_string}{HTML}{A67F59}
\definecolor{bs_highlight_background}{HTML}{F2F2F2}
\definecolor{bs_h1_h2_h3}{HTML}{005A9C}
\definecolor{bs_a_href}{HTML}{034575}
\definecolor{xcomment}{HTML}{0000A0}


%% Customize code formatting
%% Note that P2300 uses Menlo, Consolas, "DejaVu Sans Mono", Monaco, monospace;
%% TODO (maybe)
%% To use with XeLaTeX:
%% \usepackage{fontspec}
%% \newfontfamily{\lstsansserif}[Scale=.85]{Menlo}
%% \lstset{basicstyle=\lstsansserif}
%% To handle numbers:
%% cf. https://tex.stackexchange.com/questions/34896/coloring-digits-with-the-listings-package
\usepackage[final]{listings}
\lstset
{
  language=[11]C++,
  backgroundcolor=\color{bs_highlight_background},
  basicstyle=\small\ttfamily,
  breaklines=true,
  columns=fullflexible,
  commentstyle=\itshape\color{bs_comment_single},
  frame=single,
  framerule=0pt,
  identifierstyle=\color{bs_name_function},
%  keepspaces=true,
  keywordstyle=\color{bs_keyword},
  showstringspaces=false,
  stringstyle=\color{bs_string},
  texcl=true,
  xleftmargin=1em,
}

%% Keywords for C++20
\lstset{
  morekeywords = {
    char8_t,
    concept,
    consteval,
    co_await,
    co_return,
    co_yield,
    requires,
    import,
    module
  }
}

%% \lstset{
%%   emph = { % [20]
%%     import,
%%     module
%%   }
%% }

\usepackage{array}
\newcolumntype{L}[1]{>{\raggedright\arraybackslash}p{#1}}
\newcolumntype{R}[1]{>{\raggedleft\arraybackslash}p{#1}}

\usepackage{multirow}
\usepackage{multicol}
\usepackage{makecell}

\newcolumntype{L}[1]{>{\raggedright\arraybackslash}p{#1}}
\newcolumntype{R}[1]{>{\raggedleft\arraybackslash}p{#1}}
\newcolumntype{P}[1]{p{#1-3\tabcolsep-\arrayrulewidth}}


\usepackage{arydshln}
\usepackage{graphicx}
\graphicspath{{figs/}}

\usepackage{subcaption}

%% Shortcut for inline code
\newcommand{\tcode}[1]{%
  \lstinline[breaklines=true,columns=fullflexible]{#1}
}

% \usepackage{underscore}   % remove special status of '_' in ordinary text
%\usepackage{parskip}

% \newcommand{\xcomment}[2]{{\color{xcomment}[{\textsc{#1:}} \textsf{#2}]}}
\newcommand{\xcomment}[2]{}
\newcommand{\phil}[1]{\xcomment{Phil}{#1}}
\newcommand{\andrew}[1]{\xcomment{Andrew}{#1}}
\newcommand{\kevin}[1]{\xcomment{Kevin}{#1}}
\newcommand{\muhammad}[1]{\xcomment{Muhammad}{#1}}

\usepackage{comment}


%%--------------------------------------------------
%% Niko Macros
\usepackage{enumitem}

\newlist{indenthelper}{itemize}{1}
\setlist[itemize]{parsep=\parskip, partopsep=0pt, itemsep=0pt, topsep=0pt,
                  beginpenalty=10 }
\setlist[enumerate]{parsep=\parskip, partopsep=0pt, itemsep=0pt, topsep=0pt}
\setlist[indenthelper]{parsep=\parskip, partopsep=0pt, itemsep=0pt, topsep=0pt, label={}}
%% \setlist[bnflist]{parsep=\parskip, partopsep=0pt, itemsep=0pt, topsep=0pt, label={},
%%                   leftmargin=\bnfindentrest, listparindent=-\bnfindentinc, itemindent=\listparindent}

\newenvironment{indented}[1][]
{\begin{indenthelper}[#1]\item\relax}
{\end{indenthelper}}
\newenvironment{itemdescr}
{
 \begin{indented}[beginpenalty=3000, endpenalty=-300]}
{
 \end{indented}
}
%% Library function descriptions (per 4928)
\newcommand{\Fundescx}[1]{\textit{#1}}
\newcommand{\Fundesc}[1]{\Fundescx{#1:}\space}
\newcommand{\constraints}{\Fundesc{Constraints}}
\newcommand{\mandates}{\Fundesc{Mandates}}
\newcommand{\preconditions}{\Fundesc{Preconditions}}
\newcommand{\effects}{\Fundesc{Effects}}
\newcommand{\sync}{\Fundesc{Synchronization}}
\newcommand{\postconditions}{\Fundesc{Postconditions}}
\newcommand{\result}{\Fundesc{Result}}
\newcommand{\returns}{\Fundesc{Returns}}
\newcommand{\throws}{\Fundesc{Throws}}
\newcommand{\complexity}{\Fundesc{Complexity}}
\newcommand{\remarks}{\Fundesc{Remarks}}
\newcommand{\errors}{\Fundesc{Error conditions}}

%% \usepackage[⟨options⟩]{fancyhdr}
\usepackage{fancyhdr}
\pagestyle{fancy}
\fancyhead{} % clear all header fields
\fancyhead[LO,LE]{\copyright\,\textsc{ISO/IEC}}
\fancyhead[RO,RE]{\textbf{\docno}}
\fancyfoot{} % clear all footer fields
\fancyfoot[RO,RE]{\thepage}
\fancyfoot[LO,LE]{\S\thesubsection}

\newcounter{Paras}
\counterwithin{Paras}{section}
\counterwithin{Paras}{subsection}
\counterwithin{Paras}{subsubsection}
\counterwithin{Paras}{paragraph}
\counterwithin{Paras}{subparagraph}

\newcounter{Bullets1}[Paras]
\newcounter{Bullets2}[Bullets1]
\newcounter{Bullets3}[Bullets2]
\newcounter{Bullets4}[Bullets3]

\makeatletter
\newcommand{\parabullnum}[2]{%
\stepcounter{#1}%
\noindent\makebox[0pt][l]{\makebox[#2][r]{%
\scriptsize\raisebox{.7ex}%
{%
\ifnum \value{Paras}>0
\ifnum \value{Bullets1}>0 (\fi%
                          \arabic{Paras}%
\ifnum \value{Bullets1}>0 .\arabic{Bullets1}%
\ifnum \value{Bullets2}>0 .\arabic{Bullets2}%
\ifnum \value{Bullets3}>0 .\arabic{Bullets3}%
\fi\fi\fi%
\ifnum \value{Bullets1}>0 )\fi%
\fi%
}%
\hspace{\@totalleftmargin}\quad%
}}}
\makeatother
\def\pnum{\parabullnum{Paras}{0pt}}

\renewcommand{\labelitemi}{---\parabullnum{Bullets1}{\labelsep}}
\renewcommand{\labelitemii}{---\parabullnum{Bullets2}{\labelsep}}
\renewcommand{\labelitemiii}{---\parabullnum{Bullets3}{\labelsep}}
\renewcommand{\labelitemiv}{---\parabullnum{Bullets4}{\labelsep}}
