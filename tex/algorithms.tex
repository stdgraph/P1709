\section{Tier 1 Algorithms}

\subsection{Algorithm Concepts}
Additional concepts used by algorithms.

\begin{lstlisting}
template <class G, class F>
concept edge_weight_function = // e.g. weight(uv)
      copy_constructible<F> && is_arithmetic_v<invoke_result_t<F, edge_reference_t<G>>>;
\end{lstlisting}

\begin{lstlisting}
// queueable<Q> can represent std::queue and std::priority_queue
template <class Q>
concept queueable = requires(Q&& q, Q::value_type value) {
  Q::value_type;
  Q::size_type;
  Q::reference;

  {q.top()};
  {q.push(value)};
  {q.pop()};
  {q.empty()};
  {q.size()};
};
\end{lstlisting}



\subsection{Shortest Paths}

\andrew{To keep code snippets here consistent with real working code, we use lstinputlistings to grab code from alg\_synopsis.hpp and use prototypes.cpp to compile.}


\subsubsection{Shortest Paths Header Synopsis}

\andrew{Note that NetworkX also specifies single source single target and multiple source versions of the shortest paths algorithms.  BGL does not have these (nor NWGraph).  We should discuss whether or not to consider those and whether or not to make them Tier 1, 2, 3, or infinity.}


\andrew{This takes up alot of space -- perhaps better at the end after all -- or not at all for a proposal -- might fit better into actual final standard doc.}
%\lstinputlisting[firstline=300,lastline=500]{file.cc}
{\small
\lstinputlisting{src/alg_synopsis.hpp}
}



\subsubsection{Driver Interface}

\andrew{I am not sure we should have the unified interface.  We need to be more parsimonious in our interfaces.  Users can read the documentation for which algorithms to use.  And, if they are using graph algorithms, we should assume a certain level of knowledge about graph algorithms.  OTOH, it is only a handful of algorithms.}


\andrew{I am also not sure we should have ``shortest distance'' variants.  That doubles the number of functions in the interface.
For each function we have shortest paths, s-t paths, multi-source paths, parallel = 6X variants for each base function.  If we add shortest distances, that will make 12X.  OTOH, we could consider not having s-t paths or not having multi-source paths -- which would leave 4X for each base function.  However, I think people will want s-t and multi-source.
}

{\small
\lstinputlisting[firstline=4,lastline=43]{src/alg_synopsis.hpp}
}

\andrew{The variety of algorithms was inspired by networkx....  Which also had ``distance'' variants.}

\subsubsection{Unweighted Shortest Paths: Breadth-First Search}

\paragraph{Breadth-First Search, Single Source, Initialization}

{\small
\lstinputlisting[firstline=48,lastline=60]{src/alg_synopsis.hpp}
}

\begin{itemize}
\item[] 
Effects:
\begin{itemize}
\item[]
  Each \lstinline{predessors[i]} is initialized to \lstinline{i}.
\item[] Each 
\lstinline{distance[i]} is initialized to \lstinline{std::numeric_limits::max()}
or to \lstinline{init}.
\end{itemize}
\end{itemize}


\paragraph{Breadth-First Search, Single Source}

{\small
\lstinputlisting[firstline=65,lastline=104]{src/alg_synopsis.hpp}
}

\begin{itemize}
\item[] Preconditions:
\begin{itemize}
\item[]
\lstinline{graph} is an \lstinline{adjacency_list}, which may be directed or
undirected.
\item[]
\lstinline{0 <= source < num_vertices(graph)}.
\item[]
  The \lstinline{distance} range must be initialized so that 
    \lstinline{distance[i] == \lstinline{std::numeric_limits<range_value_t<D>>::max()} 
    for all \lstinline{i}
    such that \lstinline{0 <= i < num_vertices(graph)}.  \andrew{invalid distance?}
\item[]
  The \lstinline{predecessors} range must be initialized so that
  \lstinline{precessors[i] == i} for all \lstinline{i} such that 
  \lstinline{0 <= i <  num_vertices(graph)}.
\end{itemize}
\item[] 
Effects: Compute the breadth-first path and associated distance from vertex
\lstinline{source} to all reachable vertices in \lstinline{graph}.
\item[] 
Returns:
\begin{itemize}
\item[] For \lstinline{breadth_first_search} and \lstinline{breadth_first_distance},
  if vertex with index \lstinline{i} is reachable from vertex \lstinline{source}, then
  \lstinline{distances[i]} will contain the distance from \lstinline{source} to vertex
  \lstinline{i}.  If vertex \lstinline{i} is not reachable from vertex
  \lstinline{source} then \lstinline{distances[i]} will contain
  \lstinline{std::numeric_limits<range_value_t<D>>::max()}.
\item[]
For \lstinline{breadth_first_search} if vertex with index \lstinline{i} is reachable
from vertex \lstinline{source}, then \lstinline{predecessor[i]} will contain the
predecessor vertex of vertex \lstinline{i}.  If vertex \lstinline{i} is not reachable
from vertex \lstinline{source} then \lstinline{predecessors[i]} will contain
\lstinline{i}.
\end{itemize}
%
\item[] Complexity $\mathcal{O}(|E| + |V|)$
\item[] Throws: none. \andrew{Throw if \lstinline{source} out of range?}
\end{itemize}


\subsubsection{Weighted Shortest Paths}

\paragraph{Dijkstra Initialization}

{\small
\lstinputlisting[firstline=109,lastline=138]{src/alg_synopsis.hpp}
}



\begin{itemize}
\item[] 
Effects:
\begin{itemize}
\item[]
  Each \lstinline{predessors[i]} is initialized to \lstinline{i}.
\item[] Each 
\lstinline{distance[i]} is initialized to \lstinline{std::numeric_limits::max()}
or to \lstinline{init}.
\end{itemize}
\end{itemize}


\paragraph{Dijkstra Single Source Shortest Paths}

{\small
\lstinputlisting[firstline=136,lastline=168]{src/alg_synopsis.hpp}
}

\begin{itemize}
\item Preconditions:
\begin{itemize}
\item[]
\lstinline{graph} is an \lstinline{adjacency_list}, which may be directed or
undirected.
\item[]
\lstinline{0 <= source < num_vertices(graph)}.
\item[]
  The \lstinline{distance} range must be initialized so that 
    \lstinline{distance[i] == \lstinline{std::numeric_limits<range_value_t<D>>::max()} 
    for all \lstinline{i}
    such that \lstinline{0 <= i < num_vertices(graph)}.  \andrew{invalid distance?}
\item[]
  The \lstinline{predecessors} range must be initialized so that
  \lstinline{precessors[i] == i} for all \lstinline{i} such that 
  \lstinline{0 <= i <  num_vertices(graph)}.
\end{itemize}
\item[] Requirements: 
\begin{itemize}
\item[]
The return type of weight function \lstinline{w} must be able to
  be combined with the distance type \lstinline{D}.  
\item[]
The argument type for the weight
function \lstinline{w} must be convertible from the edge type.
\item[]
  The weight function \lstinline{w} must return a non-negative value (that is
  a value \lstinline{x} such that \lstinline{dijkstra_zero() <= x},
  for any edge in \lstinline{graph}.
\end{itemize}
\item[] 
Effects: Compute the shortest path and associated distance from vertex
\lstinline{source} to all reachable vertices in \lstinline{graph}.
\item[] 
Returns:
\begin{itemize}
\item[] For \lstinline{dijkstra_shortest_paths} and \lstinline{dijkstra_shortest_distances},
  if vertex with index \lstinline{i} is reachable from vertex \lstinline{source}, then
  \lstinline{distances[i]} will contain the distance from \lstinline{source} to vertex
  \lstinline{i}.  If vertex \lstinline{i} is not reachable from vertex
  \lstinline{source} then \lstinline{distances[i]} will contain
  \lstinline{std::numeric_limits<range_value_t<D>>::max()}.
\item[]
For \lstinline{dijkstra_shortest_paths} if vertex with index \lstinline{i} is reachable
from vertex \lstinline{source}, then \lstinline{predecessor[i]} will contain the
predecessor vertex of vertex \lstinline{i}.  If vertex \lstinline{i} is not reachable
from vertex \lstinline{source} then \lstinline{predecessors[i]} will contain
\lstinline{i}.
\end{itemize}
%
\item[] Complexity $\mathcal{O}((|E| + |V|)\log{|V|})$.  (Note: complexity may be
$\mathcal{O}(|E| + |V|\log{|V|)}$ for certain implementations.)
\item[] Throws: none. \andrew{Throw if \lstinline{source} out of range?}
\end{itemize}


\paragraph{Bellman-Ford Single Source Shortest Paths}

{\small
\lstinputlisting[firstline=173,lastline=247]{src/alg_synopsis.hpp}
}

\begin{itemize}
\item[] Preconditions:
\begin{itemize}
\item[]
\lstinline{graph} is an \lstinline{adjacency_list}, which may be directed or
undirected.
\item[]
\lstinline{0 <= source < num_vertices(graph)}.
\item[]
  The \lstinline{distance} range must be initialized so that 
  \lstinline{distance[i] ==  \lstinline{std::numeric_limits<range_value_t<D>>::max()} 
  for all \lstinline{i}
  such that \lstinline{0 <= i < num_vertices(graph)}.  \andrew{invalid distance?}
\item[]
  The \lstinline{predecessors} range must be initialized so that
  \lstinline{precessors[i] == i} for all \lstinline{i} such that 
  \lstinline{0 <= i < num_vertices(graph)}.
\end{itemize}
\item[] Requirements: 
\begin{itemize}
\item[]
The return type of weight function \lstinline{w} must be able to
  be combined with the distance type \lstinline{D}.  The argument type for the weight
  function \lstinline{w} must be convertible from the edge type.
\end{itemize}
\item[] 
Effects: Compute the shortest path and associated distance from vertex
\lstinline{source} to all reachable vertices in \lstinline{graph}.
\item[] 
Returns:
\begin{itemize}
\item[] For \lstinline{bellman_ford_shortest_paths} and \lstinline{bellman_ford_shortest_distances},
  if vertex with index \lstinline{i} is reachable from vertex \lstinline{source}, then
  \lstinline{distances[i]} will contain the distance from \lstinline{source} to vertex
  \lstinline{i}.  If vertex \lstinline{i} is not reachable from vertex
  \lstinline{source} then \lstinline{distances[i]} will contain
  \lstinline{std::numeric_limits<range_value_t<D>>::max()}.
\item[]
For \lstinline{bellman_ford_shortest_paths} if vertex with index \lstinline{i} is reachable
from vertex \lstinline{source}, then \lstinline{predecessor[i]} will contain the
predecessor vertex of vertex \lstinline{i}.  If vertex \lstinline{i} is not reachable
from vertex \lstinline{source} then \lstinline{predecessors[i]} will contain
\lstinline{i}.
\end{itemize}
%
\item[] Complexity $\mathcal{O}(|E| \cdot |V|)$ \andrew{This really is atrocious.  Suitable only for really small graphs.}
\item[] Throws: none. \andrew{Throw if \lstinline{source} out of range?}
\item[] Note:  Unlike Dijkstra's algorithm, Bellman-Ford allows negative edge weights.
\end{itemize}


\paragraph{Delta-Stepping Single Source}

{\small
\lstinputlisting[firstline=252,lastline=356]{src/alg_synopsis.hpp}
}


\begin{itemize}
\item[] Delta-stepping is a parallelizable version of Dijkstra's shortest path algorithm.
\item[] Preconditions:
\begin{itemize}
\item[]
\lstinline{graph} is an \lstinline{adjacency_list}, which may be directed or
undirected.
\item[]
\lstinline{0 <= source < num_vertices(graph)}.
\item[]
  The \lstinline{distance} range must be initialized so that 
    \lstinline{distance[i] == \lstinline{std::numeric_limits<range_value_t<D>>::max()} 
      for all \lstinline{i}
      such that \lstinline{0 <= i < num_vertices(graph)}.  \andrew{invalid distance?}
\item[]
  The \lstinline{predecessors} range must be initialized so that
  \lstinline{precessors[i] == i} for all \lstinline{i} such that 
  \lstinline{0 <= i < num_vertices(graph)}.
\end{itemize}
\item[] Requirements: 
\begin{itemize}
\item[]
The return type of weight function \lstinline{w} must be able to
  be combined with the distance type \lstinline{D}.  The argument type for the weight
  function \lstinline{w} must be convertible from the edge type.
\item[]
  The weight function \lstinline{w} must return a non-negative value (that is
  a value \lstinline{x} such that \lstinline{dijkstra_zero() <= x},
  for any edge in \lstinline{graph}.
\end{itemize}
\item[] 
Effects: Compute the shortest path and associated distance from vertex
\lstinline{source} to all reachable vertices in \lstinline{graph}.
\item[] 
Returns:
\begin{itemize}
\item[] For \lstinline{delta_stepping_shortest_paths} and \lstinline{delta_stepping_shortest_distances},
  if vertex with index \lstinline{i} is reachable from vertex \lstinline{source}, then
  \lstinline{distances[i]} will contain the distance from \lstinline{source} to vertex
  \lstinline{i}.  If vertex \lstinline{i} is not reachable from vertex
  \lstinline{source} then \lstinline{distances[i]} will contain
  \lstinline{std::numeric_limits<range_value_t<D>>::max()}.
\item[]
For \lstinline{delta_stepping_shortest_paths} if vertex with index \lstinline{i} is reachable
from vertex \lstinline{source}, then \lstinline{predecessor[i]} will contain the
predecessor vertex of vertex \lstinline{i}.  If vertex \lstinline{i} is not reachable
from vertex \lstinline{source} then \lstinline{predecessors[i]} will contain
\lstinline{i}.
\end{itemize}
%
\item[] Complexity $\mathcal{O}(|E| + |V| + d\cdot L)$, where d is maximum vertex degree in the graph and $L = \max_{v\in V} dist(v)$.
\item[] Throws: none. \andrew{Throw if \lstinline{source} out of range?}
\end{itemize}



\andrew{I've tagged the algorithms below as Tier 1, Tier 2, or Tier 3 -- denoting whether they should be done right now or done later or done much later.}

\andrew{I've used NetworkX as inspiration for organization.  Oddly, NetworkX only has DFS as an adaptor (view).}


\section{Tier 2 and Tier 3 Algorithms}


\subsection{Shortest Paths}
\subsubsection{Single Source, Single Target}
\andrew{Tier X}
\subsubsection{Multiple Source}
\andrew{Tier X}
\subsubsection{Multiple Source, Single Target}
\andrew{Tier X}

\begin{comment}
\subsubsection{BFS Single Source, Single Target}
\andrew{Tier X}
\subsubsection{Dijkstra Single Source, Single Target}
\andrew{Tier X}
\subsubsection{Bellman-Ford Single Source, Single Target}
\andrew{Tier X}
\subsubsection{Delta Stepping Single Source, Single Target}
\andrew{Tier X}

\subsubsection{BFS Multiple Source}
\andrew{Tier X}
\subsubsection{Dijkstra Multiple Source}
\andrew{Tier X}
\subsubsection{Bellman-Ford Multiple Source}
\andrew{Tier X}
\subsubsection{Delta Stepping Multiple Source}
\andrew{Tier X}

\subsubsection{BFS Multiple Source, Single Target}
\andrew{Tier X}
\subsubsection{Dijkstra Multiple Source, Single Target}
\andrew{Tier X}
\subsubsection{Bellman-Ford Multiple Source, Single Target}
\andrew{Tier X}
\subsubsection{Delta Stepping Multiple Source, Single Target}
\andrew{Tier X}

\end{comment}

\subsubsection{All Pairs}
\andrew{Tier 2?  People bring this up alot -- but it is very computationally expensive.}
\subsubsection{Floyd-Warshall}
\andrew{Tier 2 -- $\mathcal{O}(N^2)$}
\subsubsection{Johnson}
\andrew{Tier 2 -- $\mathcal{O}(N^2)$}

\subsection{Centrality}
\subsubsection{Betweenness Centrality}
\andrew{Tier 2}

\subsection{Clustering}
\subsubsection{Triangle Counting}
\andrew{Tier 1}

\subsection{Coloring}
\subsubsection{Greedy}
\andrew{Tier 2}
\subsubsection{Jones Plassman}
\andrew{Tier 3}

\subsection{Communities}
\subsubsection{Label Propagation}
\andrew{Tier 1}
\subsubsection{Louvain}
\andrew{Tier 2}

\subsection{Components}
\subsubsection{Articulation Points}
\andrew{Tier 1}
\subsubsection{BiConnected Components}
\andrew{Tier 1}
\subsubsection{Connected Components}
\andrew{Tier 1}
\subsubsection{Strongly Connected Components}
\andrew{Tier 1}

\subsection{Connectivity}
\subsubsection{Minimum Cuts}
\andrew{Tier 2}

\subsection{Cores}
\subsubsection{k-cores}
\andrew{Tier 3}
\subsubsection{k-truss}
\andrew{Tier 3}

\subsection{Directed Acyclic Graphs}
\subsubsection{Topological Sort}
\andrew{Tier 1}
\subsubsection{Transitive Closure}
\andrew{Tier 2}

\subsection{Flows}
\subsubsection{Edmunds Karp}
\andrew{Tier 2}
\subsubsection{Push Relabel}
\andrew{Tier 2}
\subsubsection{Boykov Kolmogorov}
\andrew{Tier 2}

\subsection{Isomorphism}
\subsubsection{Subgraph Isomorphism}
\andrew{Tier 3}

\subsection{Link Analysis}
\subsubsection{Page Rank}
\andrew{Tier 2}
\subsubsection{Jaccard Coefficient}
\andrew{Tier 2}

\subsection{Maximal Independent Set}
\subsubsection{Maximal Independent Set}
\andrew{Tier 1}

\subsection{Minimum Spanning Tree}
\subsubsection{Kruskal Minimum Spanning Tree}
\andrew{Tier 1}
\subsubsection{Prim Minimum Spanning Tree}
\andrew{Tier 1}

\subsection{Operators}
\subsubsection{Degree}
\andrew{Tier 1}
\subsubsection{Join}
\andrew{Tier 1}
\subsubsection{Relabel}
\andrew{Tier 1}
\subsubsection{Sort}
\andrew{Tier 1}
\andrew{Need to be able to sort edge lists along source or target column as well as to sort lists of neighbors in an adjacency list.  The former is necessary for efficiently building CSR from an edge list.  The second is necessary for preconditions on various algorithms.}
\subsubsection{Transpose}
\andrew{Tier 1}



\begin{comment}

%Algorithms with [TBD] are candidates to consider but having been decided for being in the paper
%We are targeting the paper to be under 50pp
All algorithms are customization points.

\phil{Algorithms marked [TBD] are provisional and may be moved to a separate proposal to keep the size of this proposal manageable}

\subsubsection{Dijkstra's Shortest Paths and Shortest Distances}
Dijkstra's algorithm \cite{REF_} is a single-source, shortest paths algorithm for non-negative weights. It finds the shortest paths 
and their weighted distances to all vertices connected to a single seed vertex.

\begin{lstlisting}
template <adjacency_list G, class DistanceValue>
auto dijkstra_invalid_distance() {
  return numeric_limits<DistanceValue>::max(); // exposition only
}

template <adjacency_list              G,
          ranges::random_access_range Distance,
          ranges::random_access_range Predecessor,
          class EVF   = std::function<ranges::range_value_t<Distance>(edge_reference_t<G>)>,
          queueable Q = 
              priority_queue<
                  weighted_vertex<G,invoke_result_t<EVF,edge_reference_t<G>>>,
                  vector<weighted_vertex<G,invoke_result_t<EVF,edge_reference_t<G>>>>,
                  greater<weighted_vertex<G,invoke_result_t<EVF,edge_reference_t<G>>>>>>
requires ranges::random_access_range<vertex_range_t<G>> &&                  //
      integral<vertex_id_t<G>> &&                                           //
      is_arithmetic_v<ranges::range_value_t<Distance>> &&                   //
      convertible_to<vertex_id_t<G>, ranges::range_value_t<Predecessor>> && //
      edge_weight_function<G, EVF>
constexpr void dijkstra_shortest_paths(
      G&&            g,           // graph
      vertex_id_t<G> seed,        // starting vertex_id
      Distance&      distance,    // out: distance[uid] of vertex_id uid from seed
      Predecessor&   predecessor, // out: predecessor[uid] of vertex_id uid in path
      EVF            weight_fn = [](edge_reference_t<G> uv)  // weight_fn(uv) -> 1
                                         { return ranges::range_value_t<Distance>(1); },
      Q              q = Q()
      );
\end{lstlisting}

\begin{table}[h!]
%\begin{center}
%\resizebox{\textwidth}{!}
{\begin{tabular}{l l}
\hline
    \textit{Complexity} & O(V log(V)). \\
                                & Additional time will be required to pre-extend  or initialize the \tcode{distance} range by the caller. \\
                                & Additional time may also be required to pre-extend \tcode{predecessor}. \\
\andrew{O((|E| + |V|)log(|V|) or O(|E| + |V|log(|V|).  Initializing distance and predecessor are O(|V|) so don't change the complexity.}
\hline
    \textit{Constraints} &  Values returned by \tcode{weight_fn} must be non-negative. \\
\hline
    \textit{Preconditions} &  \tcode{seed >= 0 && seed < size(vertices(g))} \\
                                   &  \tcode{size(distance) >= size(vertices(g))}; caller must pre-extend with \tcode{dikstra_invalid_distance()} \\
                                   &  \tcode{size(predecessor) >= size(vertices(g))}; caller must pre-extend \\
                                   & \tcode{distance[i] = dijkstra_invalid_distance()} for \tcode{i < size(vertices(g))} \\
\hline
    \textit{Postconditions} &  \tcode{distance[seed] == 0} \\
                                     & \tcode{predecessor[seed] == seed}. \\
\hline
    \textit{Effects} & \tcode{distance[uid]} will be the shortest, weighted distance of vertex\_id \tcode{uid} from \tcode{seed}. \\
    & If \tcode{uid} is not connected to \tcode{seed} by any edges then it will have a value of \tcode{dijkstra_invalid_distance()}. \\
    & \\
    & \tcode{predecessor[uid]} will have the preceding vertex\_id of \tcode{uid} in the weighted shortest path to \tcode{seed} \\
    & when \tcode{distance[uid] != dijkstra_invalid_distance()}. \\
\hline
\end{tabular}}
%\caption{\tcode{incidence} View Functions}
%\label{tab:incidence}
%\end{center}
\end{table}

\phil{Describe the pros and cons of different kinds of queues} \\

\andrew{Consider putting the queue template parameter on BFS and passing BFS as a parameter to Dijkstra. Kevin will think about it.} \\

The default weight function \tcode{weight_fn} returns a value of 1. When that is used, \tcode{distance[uid]} will have the shortest number of edges 
between vertex \tcode{uid} and vertex \tcode{seed}. The distance from \tcode{seed} to itself is zero.

If the caller wishes to use a different queue other than \tcode{priority_queue}, the queue will need to have elements of \tcode{weighted_vertex}
which is used internally by the algorithm.
\begin{lstlisting}
template <class G, class W>
requires is_default_constructible_v<vertex_id_t<G>> && is_default_constructible_v<W>
struct weighted_vertex {
  vertex_id_t<G> vertex_id = vertex_id_t<G>();
  W              weight    = W();
  constexpr auto operator<=>(const weighted_vertex&) const noexcept; // compare vertex_id
};
\end{lstlisting}

The \tcode{dijkstra_shortest_distances} function is the same as \tcode{dijkstra_shortest_paths} except that it doesn't include
the \tcode{predecessor} parameter.
\begin{lstlisting}
template <adjacency_list              G,
          ranges::random_access_range Distance,
          class EVF   = std::function<ranges::range_value_t<Distance>(edge_reference_t<G>)>,
          queueable Q = 
              priority_queue<
                  weighted_vertex<G,invoke_result_t<EVF,edge_reference_t<G>>>,
                  vector<weighted_vertex<G,invoke_result_t<EVF,edge_reference_t<G>>>>,
                  greater<weighted_vertex<G,invoke_result_t<EVF,edge_reference_t<G>>>>>>
requires ranges::random_access_range<vertex_range_t<G>> &&                  //
      integral<vertex_id_t<G>> &&                                           //
      is_arithmetic_v<ranges::range_value_t<Distance>> &&                   //
      edge_weight_function<G, EVF>
constexpr void dijkstra_shortest_distances(
      G&&            g,           // graph
      vertex_id_t<G> seed,        // starting vertex_id
      Distance&      distance,    // out: distance[uid] of vertex_id uid from seed
      EVF            weight_fn = [](edge_reference_t<G> uv)  // weight_fn(uv) -> 1
                                         { return ranges::range_value_t<Distance>(1); },
      Q              q = Q()
      );
\end{lstlisting}

\subsubsection{[TBD] Delta Stepping Shortest Paths}
The  algorithm \cite{REF_} ...

\subsubsection{Bellman-Ford Shortest Paths}
The Bellman-Ford algorithm \cite{REF_} ...

\subsubsection{[TBD] Shortest Paths}
(Overloads to select appropriate algorithm based on inputs)

\subsubsection{Connected Components}
Connected components \cite{REF_} ...

\subsubsection{Strongly Connected Components}
Strongly connected components \cite{REF_} ...

\subsubsection{Biconnected Components}
Biconnected components \cite{REF_} ...

\subsubsection{Articulation Points}
Articulation points \cite{REF_} ...

\subsubsection{Minimum Spanning Tree}
Minimum Spanning Tree \cite{REF_} ...

\subsubsection{[TBD] Page Rank}
\subsubsection{[TBD] Betweenness Centrality}
\subsubsection{[TBD] Triangle Count}
\subsubsection{[TBD] Subgraph Isomorphism}
\subsubsection{[TBD] Kruskell Minimum Spanning Tree}
\subsubsection{[TBD] Prim Minimum Spanning Tree}
\subsubsection{[TBD] Louvain (Community Detection)}
\subsubsection{[TBD] Label propagation (Community Detection)}

\end{comment}
