
\chapter{Edgelist Container Interface and Implementation}
The Edgelist Container Interface defines the primitive concepts, traits, types and functions used to define and access an edgelist. Note that edgelist does not refer to a list data structure, rather it is standard graph terminology for a collection of edges. An edgelist is often a useful intermediary data structure for loading and organizing unstructured relationship data before converting to a graph container. In order to avoid namespace collisions of functions relevant to the graph interface and edgelist interface, we propose everything defined in this section will be contained in namespace \tcode{std::graph::edgelist}.

\section{Concepts}
Table \ref{tab:edgelist_concepts} summarizes the concepts in the Edgelist Container Interface.
\begin{table}[h!]
\begin{center}
%\resizebox{\textwidth}{!}
{\begin{tabular}{l L{12.0cm}}
\hline
    \textbf{Concept} & \textbf{Definition} \\
\hline
    \tcode{edgelist_range<EL>} & \tcode{edges(EL)} returns a sized, forward\_range\\
\hline
\end{tabular}}
\caption{Edgelist Container Interface Concepts}
\label{tab:edgelist_concepts}
\end{center}
\end{table}

\section{Traits}
Table \ref{tab:edgelist_traits} summarizes the type traits in the Edgelist Container Interface, allowing views and algorithms to query the edgelist's characteristics.

\begin{table}[h!]
\begin{center}
%\resizebox{\textwidth}{!}
{\begin{tabular}{l l L{7.0cm}}
\hline
\textbf{Trait} & \textbf{Type} & \textbf{Comment} \\
\hline
\tcode{is_same_vertex_t<EL>} & concept & \tcode{source_id(el, uv)} and \tcode{target_id(el, uv)} are same type \\
\hline
\end{tabular}}
\caption{Edgelist Container Interface Type Traits}
\label{tab:edgelist_traits}
\end{center}
\end{table}


\section{Types}
Table \ref{tab:edgelist_type} summarizes the type aliases in the Edgelist Container Interface.

\begin{table}[h!]
\begin{center}
\resizebox{\textwidth}{!}
{\begin{tabular}{l l L{1.5cm}}
\hline
    \textbf{Type Alias} & \textbf{Definition} & \textbf{Comment} \\
\hline
    \tcode{edgelist_range_t<EL>} & \tcode{decltype(edges(el))} & \\
    \tcode{edgelist_iterator_t<EL>} & \tcode{iterator_t<edgelist_range_t<EL>>} & \\
    \tcode{edge_t<EL>} & \tcode{range_value_t<edgelist_range_t<EL>>} & \\
    \tcode{edge_reference_t<EL>} & \tcode{range_reference_t<edgelist_range_t<EL>>} & \\
    \tcode{vertex_source_id_t<EL>} & \tcode{decltype(vertex_id_source(el, uv))} & \\
    \tcode{vertex_target_id_t<EL>} & \tcode{decltype(vertex_id_target(el, uv))} & \\
    \tcode{edge_value_t<EL>} & \tcode{decltype(edge_value(el, uv))} & optional \\ 
    \hline
\end{tabular}}
\caption{Edgelist Container Interface Type Aliases}
\label{tab:edgelist_type}
\end{center}
\end{table}


\section{Functions}
Table \ref{tab:edgelist_func} summarizes the functions in the Edgelist Container Interface.

\begin{table}[h!]
\begin{center}
\resizebox{\textwidth}{!}
{\begin{tabular}{l l p{1.5cm} L{7.0cm}}
\hline
    \textbf{Function} & \textbf{Return Type} & \textbf{Complexity} & \textbf{Default Implementation} \\
\hline
    \tcode{edges(el)} & \tcode{edgelist_range_t<EL>} & constant & n/a \\
    \tcode{source_id(g,uv)} & \tcode{vertex_source_id_t<EL>} & constant & n/a \\
    \tcode{target_id(g,uv)} & \tcode{vertex_target_id_t<EL>} & constant & n/a \\
    \tcode{edge_value(g,uv)} & \tcode{edge_value_t<EL>} & constant & n/a \\
\hline
\end{tabular}}
\caption{Edgelist Container Interface Functions}
\label{tab:edgelist_func}
\end{center}
\end{table}

