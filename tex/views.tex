\chapter{Views}

The views in this section provide common ways that algorithms use to traverse graphs. They are a simple as iterating through the set of vertices, or more complex ways such as depth-first search and breadth-first search. The also provide a consistent and reliable way to access related elements using the View Return Types, and guaranteeing expected values, such as that the target is really the target on unordered edges.

\section{Return Types (Descriptors)}
Views return one of the types in this section, providing a consistent set of values. They are templated so that the view can adjust the 
actual values returned to be appropriate for its use. The three types, \tcode{vertex_descriptor, edge_descriptor} and 
\tcode{neighbor_descriptor}, define the data model used by the algorithms.

The following examples show the general design and how it's used. While it focuses 
on vertexlist to iterate over all vertices, it applies to all descriptors and view functions.

\begin{lstlisting}
// the type of uu is vertex\_descriptor<vertex\_id\_t<G>, vertex\_reference\_t<G>, void>
for(auto&& uu : vertexlist(g)) {
  vertex_id<G>          id = uu.id;
  vertex_reference_t<G> u  = uu.vertex;
  // ... do something interesting
}
\end{lstlisting}

Structured bindings make it simpler.
\begin{lstlisting}
for(auto&& [id, u] : vertexlist(g)) {
  // ... do something interesting
}
\end{lstlisting}

A function object can also be passed to return a value from the vertex. In this case, \tcode{vertexlist(g)} returns \tcode{vertex_descriptor<vertex_id_t<G>, vertex_reference_t<G>, decltype(vvf(u))>}.
\begin{lstlisting}
// the type returned by vertexlist is 
// vertex\_descriptor<vertex\_id\_t<G>, 
//                    vertex\_reference\_t<G>, 
//                    decltype(vvf(vertex\_reference\_t<G>))>
auto vvf = [&g](vertex_reference_t<G> u) { return vertex_value(g,u); };
for(auto&& [id, u, value] : vertexlist(g, vvf)) {
  // ... do something interesting
}
\end{lstlisting}

A simpler version also exists if all you need is a vertex id. The vertex value function takes a vertex id
instead of a vertex reference.
\begin{lstlisting}
for(auto&& [uid] : basic_vertexlist(g)) {
  // ... do something interesting
}

auto vvf = [&g](vertex_id_t<G> uid) { return vertex_value(g,uid); };
for(auto&& [uid, value] : basic_vertexlist(g,vvf)) {
  // ... do something interesting
}
\end{lstlisting}

\subsubsection{\tcode{struct vertex_descriptor<VId, V, VV>}}\label{vertex-view}\mbox{} \\
\tcode{vertex_descriptor} is used to return vertex information. It is used by \tcode{vertexlist(g)}, \tcode{vertices_breadth_first_search(g,u)}, 
\tcode{vertices_dfs(g,u)} and others. The \tcode{id} member always exists.

{\small
     \lstinputlisting{src/vertex_descriptor.hpp}
}

Specializations are defined with \tcode{V=void} or \tcode{VV=void} to suppress the existance of their associated member variables, 
giving the following valid combinations in Table \ref{tab:vertex-view} . For instance, the second entry, \tcode{vertex_descriptor<VId, V>} 
has two members \tcode{\{id_type id; vertex_type vertex;\}} and \tcode{value_type} is \tcode{void}.
\begin{table}[h!]
\begin{center}
%\resizebox{\textwidth}{!}
{\begin{tabular}{l |c c c}
\hline
    \multicolumn{1}{l}{\textbf{Template Arguments}}
    &
    \multicolumn{3}{c}{\textbf{Members}} \\
    %\textbf{Template Arguments} & id & vertex & value \\
\hline
    \tcode{vertex_descriptor<VId, V, VV>} & \tcode{id} & \tcode{vertex} & \tcode{value} \\
    \tcode{vertex_descriptor<VId, V, void>} & \tcode{id} & \tcode{vertex} & \\
    \tcode{vertex_descriptor<VId, void, VV>} & \tcode{id} & & \tcode{value} \\
    \tcode{vertex_descriptor<VId, void, void>} & \tcode{id} & & \\
\hline
\end{tabular}}
\caption{\tcode{vertex_descriptor} Members}
\label{tab:vertex-view}
\end{center}
\end{table}

A useful type alias for copying vertex values (excluding the vertex reference) is also available.
\phil{Duplicate; remove}
\begin{lstlisting}
template <class VId, class VV>
using copyable_vertex_t = vertex_descriptor<VId, void, VV>; // {id, value}
\end{lstlisting}

\subsubsection{\tcode{struct edge_descriptor<VId, Sourced, E, EV>}}\label{edge-view}\mbox{} \\
\tcode{edge_descriptor} is used to return edge information. It is used by 
\tcode{incidence(g,u), edgelist(g), edges_breadth_first_search(g,u), edges_dfs(g,u)} and others. 
When \tcode{Sourced=true}, the \tcode{source_id} member is included with type \tcode{VId}. The \tcode{target_id} member always exists.

{\small
     \lstinputlisting{src/edge_descriptor.hpp}
}

Specializations are defined with \tcode{Sourced=true|false}, \tcode{E=void} or \tcode{EV=void} to suppress the existance of the associated 
member variables, giving the following valid combinations in Table \ref{tab:edge-view} . For instance, the second entry, 
\tcode{edge_descriptor<VId,true,E>} has three members \tcode{\{source_id_type source_id; target_id_type target_id; edge_type edge;\}} 
and \tcode{value_type} is \tcode{void}.
\begin{table}[h!]
\begin{center}
%\resizebox{\textwidth}{!}
{\begin{tabular}{l |c c c c}
\hline
    \multicolumn{1}{l}{\textbf{Template Arguments}}
    &
    \multicolumn{4}{c}{\textbf{Members}} \\
    %\textbf{Template Arguments} & id & edge & value \\
\hline
    \tcode{edge_descriptor<VId, true, E, EV>} & \tcode{source_id} & \tcode{target_id} & \tcode{edge} & \tcode{value} \\
    \tcode{edge_descriptor<VId, true, E, void>} & \tcode{source_id} & \tcode{target_id} & \tcode{edge} & \\
    \tcode{edge_descriptor<VId, true, void, EV>} & \tcode{source_id} & \tcode{target_id} & & \tcode{value} \\
    \tcode{edge_descriptor<VId, true, void, void>} & \tcode{source_id} & \tcode{target_id} & & \\
    \tcode{edge_descriptor<VId, false, E, EV>} & & \tcode{target_id} & \tcode{edge} & \tcode{value} \\
    \tcode{edge_descriptor<VId, false, E, void>} & & \tcode{target_id} & \tcode{edge} & \\
    \tcode{edge_descriptor<VId, false, void, EV>} & & \tcode{target_id} & & \tcode{value} \\
    \tcode{edge_descriptor<VId, false, void, void>} & & \tcode{target_id} & & \\
\hline
\end{tabular}}
\caption{\tcode{edge_descriptor} Members}
\label{tab:edge-view}
\end{center}
\end{table}

A useful type alias for copying edge values (excluding the edge reference) is also available.
\phil{Duplicate; remove}
\begin{lstlisting}
template <class VId, class EV>
using copyable_edge_t = edge_descriptor<VId, true, void, EV>; // source\_id,target\_id[,value]
\end{lstlisting}


\subsubsection{\tcode{struct neighbor_descriptor<VId, Sourced, V, VV>}}\label{neighbor-view}\mbox{} \\
\tcode{neighbor_descriptor} is used to return information for a neighbor vertex, through an edge. It is used by \tcode{neighbors(g,u)}. 
When \tcode{Sourced=true}, the \tcode{source_id} member is included with type \tcode{source_id_type}. The \tcode{target_id} member always exists.

{\small
     \lstinputlisting{src/neighbor_descriptor.hpp}
}

Specializations are defined with \tcode{Sourced=true|false}, \tcode{E}=void or \tcode{EV}=void to suppress the existance of the 
associated member variables, giving the following valid combinations in Table \ref{tab:neighbor-view} . For instance, the second entry, 
\tcode{neighbor_descriptor<VId,true,E>} has three members \tcode{\{source_id_type source_id; target_id_type target_id; vertex_type target;\}} 
and \tcode{value_type} is \tcode{void}.
\begin{table}[h!]
\begin{center}
%\resizebox{\textwidth}{!}
{\begin{tabular}{l |c c c c}
\hline
    \multicolumn{1}{l}{\textbf{Template Arguments}}
    &
    \multicolumn{4}{c}{\textbf{Members}} \\
\hline
    \tcode{neighbor_descriptor<VId, true, E, EV>} & \tcode{source_id} & \tcode{target_id} & \tcode{target} & \tcode{value} \\
    \tcode{neighbor_descriptor<VId, true, E, void>} & \tcode{source_id} & \tcode{target_id} & \tcode{target} & \\
    \tcode{neighbor_descriptor<VId, true, void, EV>} & \tcode{source_id} & \tcode{target_id} & & \tcode{value} \\
    \tcode{neighbor_descriptor<VId, true, void, void>} & \tcode{source_id} & \tcode{target_id} & & \\
    \tcode{neighbor_descriptor<VId, false, E, EV>} & & \tcode{target_id} & \tcode{target} & \tcode{value} \\
    \tcode{neighbor_descriptor<VId, false, E, void>} & & \tcode{target_id} & \tcode{target} & \\
    \tcode{neighbor_descriptor<VId, false, void, EV>} & & \tcode{target_id} & & \tcode{value} \\
    \tcode{neighbor_descriptor<VId, false, void, void>} & & \tcode{target_id} & & \\
\hline
\end{tabular}}
\caption{\tcode{neighbor_descriptor} Members}
\label{tab:neighbor-view}
\end{center}
\end{table}

\section{Copyable Descriptors}

\subsection{Copyable Descriptor Types}
Copyable descriptors are specializations of the descriptors that can be copied. More specifically, they don't include
a vertex or edge reference. \tcode{copyable_vertex_t<G>} shows the simple definition.

\begin{lstlisting}
template <class VId, class VV>
using copyable_vertex_t = vertex_descriptor<VId, void, VV>; // {id, value}
\end{lstlisting}

\begin{table}[h!]
\begin{center}
%\resizebox{\textwidth}{!}
{\begin{tabular}{l L{12.0cm}}
\hline
    \textbf{Type} & \textbf{Definition} \\
\hline
    \tcode{copyable_vertex_t<T,VId,VV>} & \tcode{vertex_descriptor<VId, void, VV>} \\
    \tcode{copyable_edge_t<T,Vid,EV>} & \tcode{edge_descriptor<VId, true, void, EV>>} \\
    \tcode{copyable_neighbor_t<Vid,VV>} & \tcode{neighbor_descriptor<VId, true, void, VV>} \\
\hline
\end{tabular}}
\caption{Descriptor Concepts}
\label{tab:descriptor_concepts}
\end{center}
\end{table}

\subsection{Copyable Descriptor Concepts}

Given the copyable types, it's useful to have concepts to determine if a type is a desired copyable type.
\begin{table}[h!]
\begin{center}
%\resizebox{\textwidth}{!}
{\begin{tabular}{l L{12.0cm}}
\hline
    \textbf{Concept} & \textbf{Definition} \\
\hline
    \tcode{copyable_vertex<T,VId,VV>} & \tcode{convertible_to<T, copyable_vertex_t<VId, VV>>} \\
    \tcode{copyable_edge<T,Vid,EV>} & \tcode{convertible_to<T, copyable_edge_t<VId, EV>>} \\
    \tcode{copyable_neighbor<T,Vid,VV>} & \tcode{convertible_to<T, copyable_neighbor_t<VId, VV>>} \\
\hline
\end{tabular}}
\caption{Descriptor Concepts}
\label{tab:descriptor_concepts}
\end{center}
\end{table}


\section{Common Types and Functions for ``Search'' }
\phil{Do these apply to all "search" functions?}

The Depth First, Breadth First, and Topological Sort searches share a number of common types and functions. 

Here are the types and functions for cancelling a search, getting the current depth of the search, and active elements in the search (e.g. number of vertices in a stack or queue).
\begin{lstlisting}
// enum used to define how to cancel a search
enum struct cancel_search : int8_t { 
  continue_search, // no change (ignored)
  cancel_branch,   // stops searching from current vertex
  cancel_all       // stops searching and dfs will be at end()
};

// stop searching from current vertex
template<class S)
void cancel(S search, cancel_search);

// Returns distance from the seed vertex to the current vertex, 
// or to the target vertex for edge views
template<class S>
auto depth(S search) -> integral;

// Returns number of pending vertices to process
template<class S>
auto size(S search) -> integral; 
\end{lstlisting}

Of particular note, \tcode{size(dfs)} is typically the same as \tcode{depth(dfs)} and is simple to calculate. breadth\_first\_search requires extra bookkeeping to evaluate \tcode{depth(bfs)} and returns a different value than \tcode{size(bfs)}.

The following example shows how the functions  could be used, using \tcode{dfs} for one of the depth\_first\_search views. The same functions can be used for all all search views.
\begin{lstlisting}
auto&& g = ...; // graph
auto&& dfs = vertices_dfs(g,0); // start with vertex\_id=0
for(auto&& [vid,v] : dfs) {
  // No need to search deeper?
  if(depth(dfs) > 3) {
    cancel(dfs,cancel_search::cancel_branch);
    continue;
  }
  
  if(size(dfs) > 1000) {
    std::cout << "Big depth of " << size(dfs) << '\n';
  }
  
  // do useful things
}

\end{lstlisting}

\section{vertexlist Views}
\
\tcode{vertexlist} views iterate over a range of vertices, returning a \tcode{vertex_descriptor} on each iteration. 
Table \ref{tab:vertexlist} shows the vertexlist functions overloads and their return values. \tcode{first} and \tcode{last} are vertex iterators.

\phil{Change naming to \tcode{vertexlist} and \tcode{extended_vertexlist} instead of \tcode{basic_vertexlist} and \tcode{vertexlist}?}

\phil{vvf needs to accept vertex\_reference or vertex\_id}

\begin{table}[h!]
\begin{center}
%\resizebox{\textwidth}{!}
{\begin{tabular}{l l}
\hline
    \textbf{Example} & \textbf{Return} \\
\hline
    \tcode{for(auto\&\& [uid,u] : vertexlist(g))} & \tcode{vertex_descriptor<VId,V,void>} \\
    \tcode{for(auto\&\& [uid,u,val] : vertexlist(g,vvf))} & \tcode{vertex_descriptor<VId,V,VV>} \\
    \tcode{for(auto\&\& [uid,u] : vertexlist(g,first,last))} & \tcode{vertex_descriptor<VId,V,void>} \\
    \tcode{for(auto\&\& [uid,u,val] : vertexlist(g,first,last,vvf))} & \tcode{vertex_descriptor<VId,V,VV>} \\
    \tcode{for(auto\&\& [uid,u] : vertexlist(g,vr))} & \tcode{vertex_descriptor<VId,V,void>} \\
    \tcode{for(auto\&\& [uid,u,val] : vertexlist(g,vr,vvf))} & \tcode{vertex_descriptor<VId,V,VV>} \\
\hdashline
    \tcode{for(auto\&\& [uid] : basic_vertexlist(g))} & \tcode{vertex_descriptor<VId,void,void>} \\
    \tcode{for(auto\&\& [uid,val] : basic_vertexlist(g,vvf))} & \tcode{vertex_descriptor<VId,void,VV>} \\
    \tcode{for(auto\&\& [uid] : basic_vertexlist(g,first,last))} & \tcode{vertex_descriptor<VId,void,void>} \\
    \tcode{for(auto\&\& [uid,val] : basic_vertexlist(g,first,last,vvf))} & \tcode{vertex_descriptor<VId,void,VV>} \\
    \tcode{for(auto\&\& [uid] : basic_vertexlist(g,vr))} & \tcode{vertex_descriptor<VId,void,void>} \\
    \tcode{for(auto\&\& [uid,val] : basic_vertexlist(g,vr,vvf))} & \tcode{vertex_descriptor<VId,void,VV>} \\
\hline
\end{tabular}}
\caption{\tcode{vertexlist} View Functions}
\label{tab:vertexlist}
\end{center}
\end{table}

\section{incidence Views}
\tcode{incidence} views iterate over a range of adjacent edges of a vertex, returning a \tcode{edge_descriptor} on each iteration. 
Table \ref{tab:incidence} shows the \tcode{incidence} function overloads and their return values. 

Since the source vertex \tcode{u} is available when calling an \tcode{incidence} function, there's no need to include sourced versions of the function to include \tcode{source_id} in the output.

\phil{evf needs to accept edge\_reference or edges\_id (sourced required)}

\begin{table}[h!]
\begin{center}
%\resizebox{\textwidth}{!}
{\begin{tabular}{l l}
\hline
    \textbf{Example} & \textbf{Return} \\
\hline
    \tcode{for(auto\&\& [vid,uv] : incidence(g,uid))} & \tcode{edge_descriptor<VId,false,E,void>} \\
    \tcode{for(auto\&\& [vid,uv,val] : incidence(g,uid,evf))} & \tcode{edge_descriptor<VId,false,E,EV>} \\
\hdashline
    \tcode{for(auto\&\& [vid] : basic_incidence(g,uid))} & \tcode{edge_descriptor<VId,false,void,void>} \\
    \tcode{for(auto\&\& [vid,val] : basic_incidence(g,uid,evf))} & \tcode{edge_descriptor<VId,false,void,EV>} \\
\hline
\end{tabular}}
\caption{\tcode{incidence} View Functions}
\label{tab:incidence}
\end{center}
\end{table}

\section{neighbors Views}
\tcode{neighbors} views iterate over a range of edges for a vertex, returning a \tcode{vertex_descriptor} of each neighboring target vertex on each iteration. 
Table \ref{tab:neighbors} shows the \tcode{neighbors} function overloads and their return values. 

Since the source vertex \tcode{u} is available when calling a \tcode{neighbors} function, there's no need to include sourced versions of the function to include \tcode{source_id} in the output.

\phil{vvf needs to accept vertex\_reference or vertex\_id}

\begin{table}[h!]
\begin{center}
%\resizebox{\textwidth}{!}
{\begin{tabular}{l l}
\hline
    \textbf{Example} & \textbf{Return} \\
\hline
    \tcode{for(auto\&\& [vid,v] : neighbors(g,uid))} & \tcode{neighbor_descriptor<VId,false,V,void>} \\
    \tcode{for(auto\&\& [vid,v,val] : neighbors(g,uid,vvf))} & \tcode{neighbor_descriptor<VId,false,V,VV>} \\
\hdashline
    \tcode{for(auto\&\& [vid] : basic_neighbors(g,uid))} & \tcode{neighbor_descriptor<VId,false,void,void>} \\
    \tcode{for(auto\&\& [vid,val] : basic_neighbors(g,uid,vvf))} & \tcode{neighbor_descriptor<VId,false,void,VV>} \\
\hline
\end{tabular}}
\caption{\tcode{neighbors} View Functions}
\label{tab:neighbors}
\end{center}
\end{table}

\section{edgelist Views}
\tcode{edgelist} views iterate over all edges for all vertices, returning a \tcode{edge_descriptor} on each iteration. 
Table \ref{tab:edgelist} shows the \tcode{edgelist} function overloads and their return values. 

\phil{evf needs to accept edge\_reference or edges\_id (sourced required)}

\begin{table}[h!]
\begin{center}
%\resizebox{\textwidth}{!}
{\begin{tabular}{l l}
\hline
    \textbf{Example} & \textbf{Return} \\
\hline
    \tcode{for(auto\&\& [uid,vid,uv] : edgelist(g))} & \tcode{edge_descriptor<VId,true,E,void>} \\
    \tcode{for(auto\&\& [uid,vid,uv,val] : edgelist(g,evf))} & \tcode{edge_descriptor<VId,true,E,EV>} \\
\hdashline
    \tcode{for(auto\&\& [uid,uv] : basic_edgelist(g))} & \tcode{edge_descriptor<VId,true,void,void>} \\
    \tcode{for(auto\&\& [uid,uv,val] : basic_edgelist(g,evf))} & \tcode{edge_descriptor<VId,true,void,EV>} \\
\hline
\end{tabular}}
\caption{\tcode{edgelist} View Functions}
\label{tab:edgelist}
\end{center}
\end{table}

\section{Depth First Search Views}
Depth First Search views iterate over the vertices and edges from a given seed vertex, returning a \tcode{vertex_descriptor} or \tcode{edge_descriptor} on each iteration when it is first encountered, depending on the function used. 
Table \ref{tab:dfs} shows the functions and their return values.

While not shown in the examples, all functions have a final, optional allocator parameter that defaults to \tcode{std::allocator<bool>}. 
It is used for containers that are internal to the view. The \tcode{<bool>} argument has no particular meaning.

\phil{vvf needs to accept vertex\_reference or vertex\_id}

\phil{evf needs to accept edge\_reference or edges\_id (sourced required)}

\begin{table}[h!]
\begin{center}
\resizebox{\textwidth}{!}
{\begin{tabular}{l l}
\hline
    \textbf{Example} & \textbf{Return} \\
\hline
    \tcode{for(auto\&\& [vid] : basic_vertices_dfs(g,seed))} & \tcode{vertex_descriptor<VId,void,void>} \\
    \tcode{for(auto\&\& [vid,val] : basic_vertices_dfs(g,seed,vvf))} & \tcode{vertex_descriptor<VId,void,VV>} \\
\hdashline
    \tcode{for(auto\&\& [vid,v] : vertices_dfs(g,seed))} & \tcode{vertex_descriptor<VId,V,void>} \\
    \tcode{for(auto\&\& [vid,v,val] : vertices_dfs(g,seed,vvf))} & \tcode{vertex_descriptor<VId,V,VV>} \\
\hline
    \tcode{for(auto\&\& [vid] : basic_edges_dfs(g,seed))} & \tcode{edge_descriptor<VId,false,void,void>} \\
    \tcode{for(auto\&\& [vid,val] : basic_edges_dfs(g,seed,evf))} & \tcode{edge_descriptor<VId,false,void,EV>} \\
\hdashline
    \tcode{for(auto\&\& [vid,uv] : edges_dfs(g,seed))} & \tcode{edge_descriptor<VId,false,E,void>} \\
    \tcode{for(auto\&\& [vid,uv,val] : edges_dfs(g,seed,evf))} & \tcode{edge_descriptor<VId,false,E,EV>} \\
\hline
    \tcode{for(auto\&\& [uid,vid] : basic_sourced_edges_dfs(g,seed))} & \tcode{edge_descriptor<VId,true,void,void>} \\
    \tcode{for(auto\&\& [uid,vid,val] : basic_sourced_edges_dfs(g,seed,evf))} & \tcode{edge_descriptor<VId,true,void,EV>} \\
\hdashline
    \tcode{for(auto\&\& [uid,vid,uv] : sourced_edges_dfs(g,seed))} & \tcode{edge_descriptor<VId,true,E,void>} \\
    \tcode{for(auto\&\& [uid,vid,uv,val] : sourced_edges_dfs(g,seed,evf))} & \tcode{edge_descriptor<VId,true,E,EV>} \\
\hline
\end{tabular}}
\caption{depth\_first\_search View Functions}
\label{tab:dfs}
\end{center}
\end{table}

\section{Breadth First Search Views}
Breadth First Search views iterate over the vertices and edges from a given seed vertex, returning a \tcode{vertex_descriptor} 
or \tcode{edge_descriptor} on each iteration when it is first encountered, depending on the function used. 
Table \ref{tab:bfs} shows the functions and their return values.

While not shown in the examples, all functions have a final, optional allocator parameter that defaults to \tcode{std::allocator<bool>}. It is used for containers that are internal to the view. The \tcode{<bool>} argument has no particular meaning.

\phil{vvf needs to accept vertex\_reference or vertex\_id}

\phil{evf needs to accept edge\_reference or edges\_id (sourced required)}

\begin{table}[h!]
\begin{center}
\resizebox{\textwidth}{!}
{\begin{tabular}{l l}
\hline
    \textbf{Example} & \textbf{Return} \\
\hline
    \tcode{for(auto\&\& [vid] : basic_vertices_bfs(g,seed))} & \tcode{vertex_descriptor<VId,void,void>} \\
    \tcode{for(auto\&\& [vid,val] : basic_vertices_bfs(g,seed,vvf))} & \tcode{vertex_descriptor<VId,void,VV>} \\
\hdashline
    \tcode{for(auto\&\& [vid,v] : vertices_bfs(g,seed))} & \tcode{vertex_descriptor<VId,V,void>} \\
    \tcode{for(auto\&\& [vid,v,val] : vertices_bfs(g,seed,vvf))} & \tcode{vertex_descriptor<VId,V,VV>} \\
\hline
    \tcode{for(auto\&\& [vid] : basic_edges_bfs(g,seed))} & \tcode{edge_descriptor<VId,false,void,void>} \\
    \tcode{for(auto\&\& [vid,val] : basic_edges_bfs(g,seed,evf))} & \tcode{edge_descriptor<VId,false,void,EV>} \\
\hdashline
    \tcode{for(auto\&\& [vid,uv] : edges_bfs(g,seed))} & \tcode{edge_descriptor<VId,false,E,void>} \\
    \tcode{for(auto\&\& [vid,uv,val] : edges_bfs(g,seed,evf))} & \tcode{edge_descriptor<VId,false,E,EV>} \\
\hline
    \tcode{for(auto\&\& [uid,vid] : basic_sourced_edges_bfs(g,seed))} & \tcode{edge_descriptor<VId,true,void,void>} \\
    \tcode{for(auto\&\& [uid,vid,val] : basic_sourced_edges_bfs(g,seed,evf))} & \tcode{edge_descriptor<VId,true,void,EV>} \\
\hdashline
    \tcode{for(auto\&\& [uid,vid,uv] : sourced_edges_bfs(g,seed))} & \tcode{edge_descriptor<VId,true,E,void>} \\
    \tcode{for(auto\&\& [uid,vid,uv,val] : sourced_edges_bfs(g,seed,evf))} & \tcode{edge_descriptor<VId,true,E,EV>} \\
\hline
\end{tabular}}
\caption{breadth\_first\_search View Functions}
\label{tab:bfs}
\end{center}
\end{table}

\section{Topological Sort Views}
Topological Sort views iterate over the vertices and edges from a given seed vertex, returning a \tcode{vertex_descriptor} or 
\tcode{edge_descriptor} on each iteration when it is first encountered, depending on the function used. 
Table \ref{tab:topo_sort} shows the functions and their return values.

While not shown in the examples, all functions have a final, optional allocator parameter that defaults to \tcode{std::allocator<bool>}. It is used for containers that are internal to the view. The \tcode{<bool>} argument has no particular meaning.

\phil{vvf needs to accept vertex\_reference or vertex\_id}

\phil{evf needs to accept edge\_reference or edges\_id (sourced required)}

\begin{table}[h!]
\begin{center}
\resizebox{\textwidth}{!}
{\begin{tabular}{l l}
\hline
    \textbf{Example} & \textbf{Return} \\
\hline
    \tcode{for(auto\&\& [vid] : basic_vertices_topological_sort(g,seed))} & \tcode{vertex_descriptor<VId,void,void>} \\
    \tcode{for(auto\&\& [vid,val] : basic_vertices_topological_sort(g,seed,vvf))} & \tcode{vertex_descriptor<VId,void,VV>} \\
\hdashline
    \tcode{for(auto\&\& [vid,v] : vertices_topological_sort(g,seed))} & \tcode{vertex_descriptor<VId,V,void>} \\
    \tcode{for(auto\&\& [vid,v,val] : vertices_topological_sort(g,seed,vvf))} & \tcode{vertex_descriptor<VId,V,VV>} \\
\hline
    \tcode{for(auto\&\& [vid] : basic_edges_topological_sort(g,seed))} & \tcode{edge_descriptor<VId,false,void,void>} \\
    \tcode{for(auto\&\& [vid,val] : basic_edges_topological_sort(g,seed,evf))} & \tcode{edge_descriptor<VId,false,void,EV>} \\
\hdashline
    \tcode{for(auto\&\& [vid,uv] : edges_topological_sort(g,seed))} & \tcode{edge_descriptor<VId,false,E,void>} \\
    \tcode{for(auto\&\& [vid,uv,val] : edges_topological_sort(g,seed,evf))} & \tcode{edge_descriptor<VId,false,E,EV>} \\
\hline
    \tcode{for(auto\&\& [uid,vid] : basic_sourced_edges_topological_sort(g,seed))} & \tcode{edge_descriptor<VId,true,void,void>} \\
    \tcode{for(auto\&\& [uid,vid,val] : basic_sourced_edges_topological_sort(g,seed,evf))} & \tcode{edge_descriptor<VId,true,void,EV>} \\
\hdashline
    \tcode{for(auto\&\& [uid,vid,uv] : sourced_edges_topological_sort(g,seed))} & \tcode{edge_descriptor<VId,true,E,void>} \\
    \tcode{for(auto\&\& [uid,vid,uv,val] : sourced_edges_topological_sort(g,seed,evf))} & \tcode{edge_descriptor<VId,true,E,EV>} \\
\hline
\end{tabular}}
\caption{topological\_sort View Functions}
\label{tab:topo_sort}
\end{center}
\end{table}

\phil{Is Topological Sort a view, an algorithm or both?}
