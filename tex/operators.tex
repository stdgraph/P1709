\chapter{Operators}
\section{Degree}
Degree of a vertex in graph theory refers to the number of edges that are incident to the vertex. Although the proposal provides customization point interface to access a degree of an individual vertex, a degree operator allows one to build a range of degrees of all vertices within a graph.
{\small
      \lstinputlisting{src/operators/degree.hpp}
}
\begin{itemdescr}
      \pnum\preconditions
      \begin{itemize}
            \item
                  \lstinline{graph} is an \lstinline{adjacency_list}, which can be directed or undirected. \muhammad{Support other graph types as well?}
      \end{itemize}
      \pnum\effects Output iterator \lstinline{degrees_per_vertex} contains the degree of each vertex in the graph, accessible through \lstinline{degrees_per_vertex[uid]}, where uid is the \lstinline{vertex_id}. The caller must assure \lstinline{size(degrees_per_vertex) >= size(vertices(graph))}.
      \pnum\complexity $\mathcal{O}(|V|)$ where $V$ is the number of vertices in the graph.
\end{itemdescr}

\section{Sort}
Sort operator comes in many different variants, allows to sort the input graph based on the source vertex, target vertex, the degree of the vertex or by the weight on the edges. Also relevant, \lstinline{topological_sort}, which returns a nonunique permutation of the vertices of a directed acyclic graph such that an edge from u to v implies that u appears before v in the topological sort order.

\subsection{Sort by Source Vertex}
{\small
      \lstinputlisting[firstline=2,lastline=5]{src/operators/sort.hpp}
}
\begin{itemdescr}
    \pnum\preconditions
    \begin{itemize}
          \item
                \lstinline{graph} is an \lstinline{adjacency_list}, which can be directed or undirected. The same API extends for \lstinline{edge_list}.
    \end{itemize}
    \pnum\effects \lstinline{graph}, which is now sorted by the source vertices in the graph.
\end{itemdescr}

\subsection{Sort by Target Vertex}
{\small
      \lstinputlisting[firstline=7,lastline=10]{src/operators/sort.hpp}
}
\begin{itemdescr}
    \pnum\preconditions
    \begin{itemize}
          \item
                \lstinline{graph} is an \lstinline{adjacency_list}, which can be directed or undirected. The same API extends for \lstinline{edge_list}.
    \end{itemize}
    \pnum\effects \lstinline{graph}, which is now sorted by the target vertices in the graph.
\end{itemdescr}

\subsection{Sort by Degree}
{\small
      \lstinputlisting[firstline=12,lastline=14]{src/operators/sort.hpp}
}
\begin{itemdescr}
    \pnum\preconditions
    \begin{itemize}
          \item
                \lstinline{graph} is an \lstinline{adjacency_list}, which can be directed or undirected. The same API extends for \lstinline{edge_list}.
    \end{itemize}
    \pnum\effects \lstinline{graph}, which is now sorted by the vertex degree.
\end{itemdescr}

\subsection{Sort by Edge Weight}
{\small
      \lstinputlisting[firstline=16,lastline=20]{src/operators/sort.hpp}
}
\begin{itemdescr}
    \pnum\preconditions
    \begin{itemize}
          \item
                \lstinline{graph} is an \lstinline{adjacency_list}, which can be directed or undirected. The same API extends for \lstinline{edge_list}.
          \item
                \lstinline{w} The edge value function.
    \end{itemize}
    \pnum\effects \lstinline{graph}, which is now sorted by the edge weight.
\end{itemdescr}

\section{Relabel}
Relabels the vertices of the graph according to a given mapping or to integers.

{\small
      \lstinputlisting[firstline=1,lastline=3]{src/operators/relabel.hpp}
}
\begin{itemdescr}
    \pnum\preconditions
    \begin{itemize}
          \item
                \lstinline{graph} is an \lstinline{adjacency_list}, which can be directed or undirected. The same API extends for \lstinline{edge_list}.
            \item
                \lstinline{mapping} is a \lstinline{std::map}, a dictionary with old labels as keys and new labels as values. If a partial mapping is provided (i.e. \lstinline{mapping.size() < size(vertices(graph))}), then only the specified vertices are relabeled.
    \end{itemize}
    \pnum\effects \lstinline{graph}: the vertices of the graph are relabeled starting based on the mapping provided.
\end{itemdescr}

{\small
      \lstinputlisting[firstline=5,lastline=7]{src/operators/relabel.hpp}
}
\begin{itemdescr}
    \pnum\preconditions
    \begin{itemize}
          \item
                \lstinline{graph} is an \lstinline{adjacency_list}, which can be directed or undirected. The same API extends for \lstinline{edge_list}.
    \end{itemize}
    \pnum\effects \lstinline{graph}: the vertices of the graph are relabeled starting from 0 to \lstinline{size(vertices(graph))}.
\end{itemdescr}

\section{Transpose}
\section{Join}