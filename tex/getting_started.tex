
% \chapter{Getting Started}
\section{Getting Started}

This paper is one of several interrelated papers for a proposed Graph Library for the Standard C++ Library. 
The Table \ref{tab:papers} describes all the related papers. 

Their order represents different layers of the library. Reading them in order will give the best overall picture, 
but you're welcome to read in any order for the area that interests you the most.

\begin{table}[h!]
    \begin{center}
    {\begin{tabular}{l l p{14cm}}
       \hline
       \textbf{Paper}     & \textbf{Status} & \textbf{Description}                                                                                                                                                                             \\
       \hline
       P1709              & Inactive       & Original proposal, now separated into the following papers. \\
       \hdashline
       P9901              & Active         & \textbf{Overview}, describing the big picture of what we are proposing. \\
       P9908              & Active         & \textbf{Background and Terminology} providing the theoretical background and terminology used across the other documents.\\
       P9902              & Active         & \textbf{Algorithms} covering the initial algorithms 
                                             as well as the ones we'd like to see in the future. \\
       %P9903              & Future         & \textbf{Operators} includes useful utility functions when
       %                                      working with graphs. \\
       P9904              & Active         & \textbf{Views} has helpful views for traversing a graph. \\
       P9905              & Active         & \textbf{Graph Container Interface} is the core interface used
                                             for uniformly accessing graph data structures by views and algorithms.
                                             It is also designed to easily adapt to existing graph data structures.\\
       P9906              & Active         & \textbf{Graph Container Implementations} describing use of standard containers
                                             and a new  high-performance \tcode{compressed_graph} container, based on a 
                                             Compressed Sparse Row matrix. \\
       %P9907              & Future         & \textbf{Adaptors} containing useful utilities to create adjacency lists from other data structures.\\
       \hline
    \end{tabular}}
      \caption{Graph Library Papers}
      \label{tab:papers}
    \end{center}
\end{table}

\textbf{Reading Guide:} 
\begin{itemize}
  \item If you're \textbf{new to the Graph Library}, we recommend starting with the \textit{Overview} paper (P9901).
  \item If you want to \textbf{understand the theoretical background} that underpins what we're doing, you should read the \textit{Background and Terminology} paper (P9908).
  \item If you want to \textbf{use the algorithms}, you should read the \textit{Algorithms} paper (P9902) and \textbf{Graph Container Implementations} paper (9906).
  \item If you want to \textbf{write algorithms}, you should read the \textit{Views} paper (P9905), \textit{Graph Container Interface} paper (P9905) and \textit{Graph Container Implementations} paper (P9906).
  \item If you want to \textbf{use your own graph container}, you should read the \textit{Graph Container Interface} paper (P9905) and \textit{Graph Container Implementations} paper (P9906).
\end{itemize}

  
%\clearpage

\section{Naming Conventions}

Table~\ref{tab:name_conv} shows the naming conventions used throughout the Graph Library documents.

\begin{table}[h!]
  \begin{center}
  {\begin{tabular}{l l l p{7cm}}
     \hline
     \textbf{Template}  &                                   & \textbf{Variable}    &                                                                                                                                                                                                  \\
     \textbf{Parameter} & \textbf{Type Alias}               & \textbf{Names}       & \textbf{Description}                                                                                                                                                                             \\
     \hline
     \tcode{G}          &                                   &                      & Graph                                                                                                                                                                                            \\
     & \tcode{graph_reference_t<G>}      & \tcode{g}            & Graph reference                                                                                                                                                                                                     \\
     \tcode{GV}         &                                   & \tcode{val}          & Graph Value, value or reference                                                                                                                                                                  \\
     \hline
     \tcode{EL}         &                                   & \tcode{el}           & Edge list                                                                                                                                                                                        \\
     \hline
     \tcode{V}          & \tcode{vertex_t<G>}               &                      & Vertex descriptor                                                                                                                                                                                \\
                        & \tcode{vertex_reference_t<G>}     & \tcode{u,v,x,y}      & Vertex descriptor reference. \tcode{u} is the source (or only) vertex. \tcode{v} is the target vertex.                                                                                           \\
     \tcode{VId}        & \tcode{vertex_id_t<G>}            & \tcode{uid,vid,seed} & Vertex id. \tcode{uid} is the source (or only) vertex id. \tcode{vid} is the target vertex id.                                                                                                   \\
     \tcode{VV}         & \tcode{vertex_value_t<G>}         & \tcode{val}          & Vertex Value, value or reference. This can be either the user-defined value on a vertex, or a value returned by a function object (e.g. \tcode{VVF}) that is related to the vertex.              \\
     \tcode{VR}         & \tcode{vertex_range_t<G>}         & \tcode{ur,vr}        & Vertex Range                                                                                                                                                                                     \\
     \tcode{VI}         & \tcode{vertex_iterator_t<G>}      & \tcode{ui,vi}        & Vertex Iterator. \tcode{ui} is the source (or only)                                                                                                                                              \\
                        &                                   & \tcode{first,last}   & vertex. \tcode{vi} is the target vertex.                                                                                                                                                         \\
     \tcode{VVF}        &                                   & \tcode{vvf}          & Vertex Value Function: vvf(u) $\rightarrow$ vertex value, or vvf(uid) $\rightarrow$ vertex value, depending on requirements of the consume algorithm or view.                                    \\
     \tcode{VProj}      &                                   & \tcode{vproj}        & Vertex info projection function: \tcode{vproj(x)} $\rightarrow$ \tcode{vertex_info<VId,VV>}.                                                                                                     \\
     \hdashline
                        & \tcode{partition_id_t<G>}         & \tcode{pid}          & Partition id.                                                                                                                                                                                    \\
                        &                                   & \tcode{P}            & Number of partitions.                                                                                                                                                                            \\
     \tcode{PVR}        & \tcode{partition_vertex_range_t<G>} & \tcode{pur,pvr}    & Partition vertex range.                                                                                                                                                                          \\
     \hline
     \tcode{E}          & \tcode{edge_t<G>}                 &                      & Edge descriptor                                                                                                                                                                                  \\
                        & \tcode{edge_reference_t<G>}       & \tcode{uv,vw}        & Edge descriptor reference. \tcode{uv} is an edge from vertices \tcode{u} to \tcode{v}. \tcode{vw} is an edge from vertices \tcode{v} to \tcode{w}.                                               \\
     \tcode{EV}         & \tcode{edge_value_t<G>}           & \tcode{val}          & Edge Value, value or reference. This can be either the user-defined value on an edge, or a value returned by a function object (e.g. \tcode{EVF}) that is related to the edge.                   \\
     \tcode{ER}         & \tcode{vertex_edge_range_t<G>}    &                      & Edge Range for edges of a vertex                                                                                                                                                                 \\
     \tcode{EI}         & \tcode{vertex_edge_iterator_t<G>} & \tcode{uvi,vwi}      & Edge Iterator for an edge of a vertex. \tcode{uvi} is an iterator for an edge from vertices \tcode{u} to \tcode{v}. \tcode{vwi} is an iterator for an edge from vertices \tcode{v} to \tcode{w}. \\
     \tcode{EVF}        &                                   & \tcode{evf}          & Edge Value Function: evf(uv) $\rightarrow$ edge value, or evf(eid) $\rightarrow$ edge value, depending on the requirements of the consuming algorithm or view.                                   \\
     \tcode{EProj}      &                                   & \tcode{eproj}        & Edge info projection function: \tcode{eproj(x)} $\rightarrow$ \tcode{edge_info<VId,Sourced,EV>}.                                                                                                 \\
     \hline
  \end{tabular}}
    \caption{Naming Conventions for Types and Variables}
    \label{tab:name_conv}
  \end{center}
\end{table}

