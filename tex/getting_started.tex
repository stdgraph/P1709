\chapter{Getting Started}

This paper is one of several interrelated proposals related to a Graph Library proposal that have been
broken out for easier consumption. The following table describes all the related papers.

\begin{table}[h!]
    \begin{center}
    {\begin{tabular}{l l p{14cm}}
       \hline
       \textbf{Paper}     & \textbf{Status} & \textbf{Description}                                                                                                                                                                             \\
       \hline
       P1709              & Inactive       & Original proposal, now broken into the following papers. \\
       \hdashline
       P9901              & Active         & \textbf{Overview and Introduction}, describing the big
                                             picture of what we are proposing and theortical basis underlying the proposal. \\
       P9902              & Active         & \textbf{Algorithms}, covering the initial algorithms 
                                             as well as the ones we'd like to see in the future. \\
       P9903              & Active         & \textbf{Operators} includes useful utility functions when
                                             working with graphs. \\
       P9904              & Active         & \textbf{Views} including helpful views for traversing a graph. \\
       P9905              & Active         & \textbf{Graph Container Inferface} is the core interface used
                                             for uniformly accessing graph data structures.\\
       P9906              & Active         & \textbf{Graph Container} describing the high-performance \tcode{compressed_graph} 
                                             container, based on a Compressed Sparse Row sparse matrix layout. \\
       P9907              & Active         & \textbf{Adaptors} containing useful utilities to convert graphs to different forms.\\
       \hline
    \end{tabular}}
      \caption{Graph Library Papers}
      \label{tab:papers}
    \end{center}
\end{table}
  
%\clearpage

\section{Naming Conventions}

Table~\ref{tab:name_conv} shows the naming conventions used throughout the Graph Library documents.

\begin{table}[h!]
  \begin{center}
  {\begin{tabular}{l l l p{7cm}}
     \hline
     \textbf{Template}  &                                   & \textbf{Variable}    &                                                                                                                                                                                                  \\
     \textbf{Parameter} & \textbf{Type Alias}               & \textbf{Names}       & \textbf{Description}                                                                                                                                                                             \\
     \hline
     \tcode{G}          &                                   &                      & Graph                                                                                                                                                                                            \\
     & \tcode{graph_reference_t<G>}      & \tcode{g}            & Graph reference                                                                                                                                                                                  \\
     \tcode{GV}         &                                   & \tcode{val}          & Graph Value, value or reference                                                                                                                                                                  \\
     \hline
     \tcode{EL}         &                                   & \tcode{el}           & Edge list                                                                                                                                                                                        \\
     \hline
     \tcode{V}          & \tcode{vertex_t<G>}               &                      & Vertex                                                                                                                                                                                           \\
     & \tcode{vertex_reference_t<G>}     & \tcode{u,v,x,y}      & Vertex reference. \tcode{u} is the source (or only) vertex. \tcode{v} is the target vertex.                                                                                                      \\
     \tcode{VId}        & \tcode{vertex_id_t<G>}            & \tcode{uid,vid,seed} & Vertex id. \tcode{uid} is the source (or only) vertex id. \tcode{vid} is the target vertex id.                                                                                                   \\
     \tcode{VV}         & \tcode{vertex_value_t<G>}         & \tcode{val}          & Vertex Value, value or reference. This can be either the user-defined value on a vertex, or a value returned by a function object (e.g. \tcode{VVF}) that is related to the vertex.              \\
     \tcode{VR}         & \tcode{vertex_range_t<G>}         & \tcode{ur,vr}        & Vertex Range                                                                                                                                                                                     \\
     \tcode{VI}         & \tcode{vertex_iterator_t<G>}      & \tcode{ui,vi}        & Vertex Iterator. \tcode{ui} is the source (or only) vertex.                                                                                                                                      \\
     &                                   & \tcode{first,last}   & \tcode{vi} is the target vertex.                                                                                                                                                                                    \\
     \tcode{VVF}        &                                   & \tcode{vvf}          & Vertex Value Function: vvf(u) $\rightarrow$ vertex value, or vvf(uid) $\rightarrow$ vertex value, depending on requirements of the consume algorithm or view.                                    \\
     \tcode{VProj}      &                                   & \tcode{vproj}        & Vertex descriptor projection function: \tcode{vproj(x)} $\rightarrow$ \tcode{vertex_descriptor<VId,VV>}.                                                                                         \\
     \hdashline
                        & \tcode{partition_id_t<G>}         & \tcode{pid}          & Partition id.                                                                                                                                                                                    \\
                        &                                   & \tcode{P}            & Number of partitions.                                                                                                                                                                            \\
     \tcode{PVR}        & \tcode{partition_vertex_range_t<G>} & \tcode{pur,pvr}    & Partition vertex range.                                                                                                                                                                          \\
     \hline
     \tcode{E}          & \tcode{edge_t<G>}                 &                      & Edge                                                                                                                                                                                             \\
     & \tcode{edge_reference_t<G>}       & \tcode{uv,vw}        & Edge reference. \tcode{uv} is an edge from vertices \tcode{u} to \tcode{v}. \tcode{vw} is an edge from vertices \tcode{v} to \tcode{w}.                                                                             \\
     \tcode{EId}        & \tcode{edge_id_t<G>}              & \tcode{eid,uvid}     & Edge id, a pair of vertex\_ids.                                                                                                                                                                  \\
     \tcode{EV}         & \tcode{edge_value_t<G>}           & \tcode{val}          & Edge Value, value or reference. This can be either the user-defined value on an edge, or a value returned by a function object (e.g. \tcode{EVF}) that is related to the edge.                   \\
     \tcode{ER}         & \tcode{vertex_edge_range_t<G>}    &                      & Edge Range for edges of a vertex                                                                                                                                                                 \\
     \tcode{EI}         & \tcode{vertex_edge_iterator_t<G>} & \tcode{uvi,vwi}      & Edge Iterator for an edge of a vertex. \tcode{uvi} is an iterator for an edge from vertices \tcode{u} to \tcode{v}. \tcode{vwi} is an iterator for an edge from vertices \tcode{v} to \tcode{w}. \\
     \tcode{EVF}        &                                   & \tcode{evf}          & Edge Value Function: evf(uv) $\rightarrow$ edge value, or evf(eid) $\rightarrow$ edge value, depending on the requirements of the consuming algorithm or view.                                   \\
     \tcode{EProj}      &                                   & \tcode{eproj}        & Edge descriptor projection function: \tcode{eproj(x)} $\rightarrow$ \tcode{edge_descriptor<VId,Sourced,EV>}.                                                                                     \\
     \hline
  \end{tabular}}
    \caption{Naming Conventions for Types and Variables}
    \label{tab:name_conv}
  \end{center}
\end{table}

