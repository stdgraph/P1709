
% \chapter{Getting Started}
\section{Getting Started}

This paper is one of several interrelated papers for a proposed Graph Library for the Standard C++ Library. 
The Table \ref{tab:papers} describes all the related papers. 

\begin{table}[h!]
    \begin{center}
    {\begin{tabular}{l l p{14cm}}
       \hline
       \textbf{Paper}     & \textbf{Status} & \textbf{Description}                                                                                                                                                                             \\
       \hline
       P1709              & Inactive       & Original proposal, now separated into the following papers. \\
       \hdashline
       \href{https://www.wg21.link/P3126}{P3126} & Active         & \textbf{Overview}, describes the big picture of what we are proposing. \\
       \href{https://www.wg21.link/P3127}{P3127} & Active         & \textbf{Background and Terminology} provides the motivation, theoretical background, and terminology used across the other documents.\\
       \href{https://www.wg21.link/P3128}{P3128} & Active         & \textbf{Algorithms} covers the initial algorithms 
                                             as well as the ones we'd like to see in the future. \\
       %P9903              & Future         & \textbf{Operators} includes useful utility functions when
       %                                      working with graphs. \\
       \href{https://www.wg21.link/P3129}{P3129} & Active         & \textbf{Views} has helpful views for traversing a graph. \\
       \href{https://www.wg21.link/P3130}{P3130} & Active         & \textbf{Graph Container Interface} is the core interface used
                                             for uniformly accessing graph data structures by views and algorithms.
                                             It is also designed to easily adapt to existing graph data structures.\\
       \href{https://www.wg21.link/P3131}{P3131} & Active         & \textbf{Graph Containers} describes a proposed high-performance \tcode{compressed_graph} container.
                                              It also discusses how to use containers in the standard library to define a graph, and how 
                                              to adapt existing graph data structures.\\
       %P9907              & Future         & \textbf{Adaptors} containing useful utilities to create adjacency lists from other data structures.\\
       \href{https://www.wg21.link/P3337}{P3337} & Soon           & \textbf{Comparison to other graph libraries} on performance and usage syntax. \\
       \hline
    \end{tabular}}
      \caption{Graph Library Papers}
      \label{tab:papers}
    \end{center}
\end{table}

Reading them in order will give the best overall picture.
If you're limited on time, you can use the following guide to focus on the papers that are most relevant to your needs.

\textbf{Reading Guide} 
\begin{itemize}
  \item If you're \textbf{new to the Graph Library}, we recommend starting with the \textit{Overview} (\href{https://www.wg21.link/P3126}{P3126}) paper to understand the focus and scope of our proposals.
        You'll also want to check out how it stacks up against other graph libraries in performance and usage syntax in the \textit{Comparison} (\href{https://www.wg21.link/P3337}{P3337}) paper.
  \item If you want to \textbf{understand the terminology and theoretical background} that underpins what we're doing, you should read the \textit{Background and Terminology} (\href{https://www.wg21.link/P3127}{P3127}) paper.
  \item If you want to \textbf{use the algorithms}, you should read the \textit{Algorithms} (\href{https://www.wg21.link/P3128}{P3128}) and \textit{Graph Containers} (\href{https://www.wg21.link/P3131}{P3131}) papers.
        You may also find the \textit{Views} (\href{https://www.wg21.link/P3129}{P3129}) and \textit{Graph Container Interface} (\href{https://www.wg21.link/P3130}{P3130}) papers helpful. 
  \item If you want to \textbf{write new algorithms}, you should read the \textit{Views} (\href{https://www.wg21.link/P3129}{P3129}), \textit{Graph Container Interface} (\href{https://www.wg21.link/P3130}{P3130}), and \textit{Graph Containers} (\href{https://www.wg21.link/P3131}{P3131}) papers.
        You'll also want to review existing implementations in the reference library for examples of how to write the algorithms.
  \item If you want to \textbf{use your own graph data structures}, you should read the \textit{Graph Container Interface} (\href{https://www.wg21.link/P3130}{P3130}) and \textit{Graph Containers} (\href{https://www.wg21.link/P3131}{P3131}) papers.
\end{itemize}

  
%\clearpage

\section{Naming Conventions}

Table~\ref{tab:name_conv} shows the naming conventions used throughout the Graph Library documents.

\begin{table}[h!]
  \begin{center}
  {\begin{tabular}{l l l p{7cm}}
     \hline
     \textbf{Template}  &                                   & \textbf{Variable}    &                                                                                                                                                                                                  \\
     \textbf{Parameter} & \textbf{Type Alias}               & \textbf{Names}       & \textbf{Description}                                                                                                                                                                             \\
     \hline
     \tcode{G}          &                                   &                      & Graph                                                                                                                                                                                            \\
     & \tcode{graph_reference_t<G>}      & \tcode{g}            & Graph reference                                                                                                                                                                                                     \\
     \tcode{GV}         &                                   & \tcode{val}          & Graph Value, value or reference                                                                                                                                                                  \\
     \hline
     \tcode{EL}         &                                   & \tcode{el}           & Edge list                                                                                                                                                                                        \\
     \hline
     \tcode{V}          & \tcode{vertex_t<G>}               &                      & Vertex descriptor                                                                                                                                                                                \\
                        & \tcode{vertex_reference_t<G>}     & \tcode{u,v,x,y}      & Vertex descriptor reference. \tcode{u} is the source (or only) vertex. \tcode{v} is the target vertex.                                                                                           \\
     \tcode{VId}        & \tcode{vertex_id_t<G>}            & \tcode{uid,vid,seed} & Vertex id. \tcode{uid} is the source (or only) vertex id. \tcode{vid} is the target vertex id.                                                                                                   \\
     \tcode{VV}         & \tcode{vertex_value_t<G>}         & \tcode{val}          & Vertex Value, value or reference. This can be either the user-defined value on a vertex, or a value returned by a function object (e.g. \tcode{VVF}) that is related to the vertex.              \\
     \tcode{VR}         & \tcode{vertex_range_t<G>}         & \tcode{ur,vr}        & Vertex Range                                                                                                                                                                                     \\
     \tcode{VI}         & \tcode{vertex_iterator_t<G>}      & \tcode{ui,vi}        & Vertex Iterator. \tcode{ui} is the source (or only)                                                                                                                                              \\
                        &                                   & \tcode{first,last}   & vertex. \tcode{vi} is the target vertex.                                                                                                                                                         \\
     \tcode{VVF}        &                                   & \tcode{vvf}          & Vertex Value Function: vvf(u) $\rightarrow$ vertex value, or vvf(uid) $\rightarrow$ vertex value, depending on requirements of the consume algorithm or view.                                    \\
     \tcode{VProj}      &                                   & \tcode{vproj}        & Vertex info projection function: \tcode{vproj(x)} $\rightarrow$ \tcode{vertex_info<VId,VV>}.                                                                                                     \\
     \hdashline
                        & \tcode{partition_id_t<G>}         & \tcode{pid}          & Partition id.                                                                                                                                                                                    \\
                        &                                   & \tcode{P}            & Number of partitions.                                                                                                                                                                            \\
     \tcode{PVR}        & \tcode{partition_vertex_range_t<G>} & \tcode{pur,pvr}    & Partition vertex range.                                                                                                                                                                          \\
     \hline
     \tcode{E}          & \tcode{edge_t<G>}                 &                      & Edge descriptor                                                                                                                                                                                  \\
                        & \tcode{edge_reference_t<G>}       & \tcode{uv,vw}        & Edge descriptor reference. \tcode{uv} is an edge from vertices \tcode{u} to \tcode{v}. \tcode{vw} is an edge from vertices \tcode{v} to \tcode{w}.                                               \\
     \tcode{EV}         & \tcode{edge_value_t<G>}           & \tcode{val}          & Edge Value, value or reference. This can be either the user-defined value on an edge, or a value returned by a function object (e.g. \tcode{EVF}) that is related to the edge.                   \\
     \tcode{ER}         & \tcode{vertex_edge_range_t<G>}    &                      & Edge Range for edges of a vertex                                                                                                                                                                 \\
     \tcode{EI}         & \tcode{vertex_edge_iterator_t<G>} & \tcode{uvi,vwi}      & Edge Iterator for an edge of a vertex. \tcode{uvi} is an iterator for an edge from vertices \tcode{u} to \tcode{v}. \tcode{vwi} is an iterator for an edge from vertices \tcode{v} to \tcode{w}. \\
     \tcode{EVF}        &                                   & \tcode{evf}          & Edge Value Function: evf(uv) $\rightarrow$ edge value, or evf(eid) $\rightarrow$ edge value, depending on the requirements of the consuming algorithm or view.                                   \\
     \tcode{EProj}      &                                   & \tcode{eproj}        & Edge info projection function: \tcode{eproj(x)} $\rightarrow$ \tcode{edge_info<VId,Sourced,EV>}.                                                                                                 \\
     \hline
  \end{tabular}}
    \caption{Naming Conventions for Types and Variables}
    \label{tab:name_conv}
  \end{center}
\end{table}

