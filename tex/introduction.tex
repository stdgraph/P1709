% algorithms: 
% views: 
% containers: adjacency matrix
% data model: incoming edges on a vertex

\chapter{Introduction}

\section{Example: Six Degrees of Kevin Bacon}

A classic example of the use of a graph algorithm is the game "The Six Degrees of Kevin Bacon."
The game is played by connecting actors to each other through movies they have appeared in together.
The goal is to find the smallest number of movies that connect a given actor to Kevin Bacon.
That number is called the Bacon number of the actor.  Kevin Bacon himself has a Bacon number of 0.
Since Kevin Bacon appeared with Tom Cruise in "A Few Good Men", Tom Cruise has a Bacon number of 1.

The following program computes the Bacon number for a small selection of actors.
{\small
\begin{lstlisting}
std::vector<std::string> actors { "Tom Cruise",       "Kevin Bacon",       "Hugo Weaving",
                                  "Carrie-Anne Moss", "Natalie Portman",   "Jack Nicholson",
                                  "Kelly McGillis",   "Harrison Ford",     "Sebastian Stan",
                                  "Mila Kunis",       "Michelle Pfeiffer", "Keanu Reeves",
                                  "Julia Roberts" };

std::vector<std::vector<int>> costar_adjacency_list {
    {1, 5, 6}, {7, 10, 0, 5, 12}, {4, 3, 11}, {2, 11}, {8, 9, 2, 12}, {0, 1}, {7, 0},
    {6, 1, 10}, {4, 9}, {4, 8}, {7, 1}, {2, 3}, {1, 4} };

int main() {
  std::vector<int> bacon_number(size(actors));

  for (auto&& [u, v] : bfs_edge_range(costar_adjacency_list, 1)) { // Vertex 1 -> Kevin Bacon
    bacon_number[v] = bacon_number[u] + 1;
  }

  for (int i = 0; i < size(actors); ++i) {
    std::cout << actors[i] << " has Bacon number " << bacon_number[i] << std::endl;
  }
}
\end{lstlisting}
}

In graph parlance, we are creating a graph where the vertices are actors and the edges are movies.
The number of movies that connect an actor to Kevin Bacon is the shortest path in the graph
from Kevin Bacon to that actor.  In the example above, we compute shortest paths from Kevin
Bacon to all other actors and print the results.


\section{Graph Background and Terminology}

We use commonly accepted terminology for graph data structures and algorithms and 
adopt the particular terminology used in the textbook by
Cormen, Leiserson, Rivest, and Stein (``CLRS'')~\cite{CLRS2022}.

\subsection{Theoretical Terminology}

A \emph{graph} $G$ is a set comprised of two other sets, the vertex set $V$ and the edge set $E$, 
typically expressed as $G=\{V,E\}$.  We can number the members of the vertex set
and write $V = \{ v_0, v_1, \ldots , v_{n-1} \}$.  Similarly, we can write $E = \{ e_0, e_1, \ldots, e_{m-1} \}$.  The number of elements in $V$ is denoted by $|V|$ and the number of elements in $E$ is denoted by $|E|$.  Edges in $E$ are pairs of vertices; for any $e_k \in E$, we write $e_k = ( v_i, v_j )$, where $v_i \in V$ and $v_j\in V$.  The edges in $E$ may be \emph{uordered}, in which case $(v_i, v_j) = (v_j, v_i)$ or \emph{ordered}, in which case $(v_i, v_j) \neq (v_j, v_i)$ (unless $i = j$).
A graph consisting of unordered edges is said to be \emph{undirected}; 
a graph consisting of ordered edges is said to be \emph{directed}.
Since there is a single set of vertices in this definition of $G$, that is, edges connect
vertices in $V$ to vertices in $V$, we say that
$G$ is a \emph{unipartite} graph.
If the graph further has no loops (edges connecting a vertex to itself)
and no multiple edges (edges connecting the same pair of vertices),
then $G$ is called a \emph{simple} graph.
A \emph{complete graph} is a simple graph $G$ where every vertex is connected
to every other vertex; a complete graph has
$n$ vertices and $n(n-1)/2$ edges.
A \emph{path} is a sequence of vertices $v_0, v_1, \ldots, v_{k-1}$ such that
there is an edge from $v_0$ to $v_1$, an edge from $v_1$ to $v_2$, and so on.
That is, a path is a set of edges $(v_i, v_{i+1}) \in E$ for all $i = 0, 1, \ldots, k-2$.

Graphs are often deipicted as a collection of nodes and links.  In this model, vertices are
represented graphically as nodes, which are
shown as circles (or some other shape, including points).  Edges are reprsented as links,
shown as lines connecting the nodes.

\andrew{We should have a picture here.}

Associated with every $v_i \in V$ is the integer $i$.  We call $i$ the \emph{index}
(or the \emph{vertex identifier}) of $v_i$.  The index of a vertex is a unique
identifier for that vertex.  The index of a vertex is not necessarily the same
as the position of the vertex in the vertex set $V$.  For example, in the graph
$G = \{ \{ 0, 1, 2, 3 \}, \{ (0, 1), (1, 2), (2, 3) \} \}$, the vertex $v_0$ has index 0,
$v_1$ has index 1, $v_2$ has index 2, and $v_3$ has index 3.  In the literature,
a vertex and its index are often used interchangeably.

A \emph{bipartite graph} is a graph $G$ with two disjoint sets of
vertices $U$ and $V$ such that every edge in $E$ connects a vertex
in $U$ to a vertex in $V$.  A bipartite graph is useful for modeling
connections between two different types of objects.  For example,
a bipartite graph can be used to model connections between
actors and movies that they have appeared in.  To obtain
an actor-actor co-star graph from an actor-movie graph, we
perform a \emph{join} operation on the graph and its \emph{transpose}, which is
the graph connecting movies to actors, that is the graph with the same
vertices as the original graph but with edge directions reversed, i.e., a graph
connecting movies to actors.

\subsubsection{Adjacency List}

An \emph{adjacency list} of $G$ is a data structure that represents a graph $G$ as an array
indexed by the vertex indices of the vertices of $G$.
The adjacency list of $G$ is denoted as $Adj(G)$.
For each vertex $v_i \in G$, the array entry $Adj(G)[i]$ is a list of
vertex indices of the vertices adjacent to $v_i$.
The adjacency list is the most common representation of a graph in the graph literature
and is used in the majority of graph algorithms as it has the important
property for traversal that (in the case of a unipartite graph)
an index $j$ stored neighbor list $Adj(G)[i]$ can
be used to index back into $Adj(G)$.  In this formulation, the adjacency list
does not store vertices per se (rather, it stores vertex indices), it captures the
structure of the graph.  This structure is the essence of the graph abstraction and
is the essential information needed for graph algorithms.
Even so, it is still possible to store vertex data in the adjacency list by storing
the vertex data in an auxiliary array indexed by vertex indices.


\subsubsection{Edge List}

An \emph{edge list} of $G$ is a data structure that contains the edges of $G$.  More precisely, as
with the adjacency list structure, the edge list stores pairs of vertex indices.  That is, for
every $(v_i, v_j) \in E$, the edge list contains the pair $(i, j)$.  The edge list is
a less common representation of a graph than the adjacency list, but it is useful for
some graph algorithms.  Moreover, it is a natural intermediate representation for
converting between data that has not yet been interpreted as a graph, e.g., a table,
and an adjacency list.  Moreover, certain tasks, such as sorting or relabeling vertices, are
more natural to express in terms of the edge list.


\subsubsection{Adjacency Matrix}

\andrew{Is this useful?  Should it be included in std::graph?
I personally don't think it is very useful, but the committee may disagree
since it is often used in the literature (typically introductory).}


\section{From Data to Graph}

\andrew{Illustrate the progression from data to a graph.}


\section{Implementations}


\subsection{Edge List: Array of Structs / Struct of Arrays}

\andrew{Edges can be stored as tuples (or tuple like) basically in parallel arrays (like a ranges::zip).}


\subsection{Vertices and Edges: Node Objects and Link Objects}

\andrew{ala Stanford Graph Base (et al).}


\subsection{Adjacency List: Container of Containers}

\andrew{An adjacency list can be represented as a container of containers (e.g., std::vector<std::list>).  Talk about range of ranges when we talk about library interfaces, requirements, concepts.  Note here that the structure of an adjacency list does not capture directedness -- directedness is a run-time property.}


\subsection{From Edge List to Adjacency List}

\andrew{Scan edge list and insert edgest into adjacency list.  Adjacency list must support insertion.}

\subsection{Edge List and Adjacency List: Compressed Edge List}

\andrew{Using a sort and group-by (or a sort, a run-length encoding, and a scan), we can compactify the edge-list reprentation and at the same time obtain an adjacency-list representation -- one that is memory and compute efficient.  Best of both worlds.  Has same basic structural principles as CSR / CSC matrices in linear algebra -- but much more general.}


\section{BiPartite Graphs}

So far, we have been considering graphs where the vertices, and both elements of an
edge, are members of a single set $|V|$.  A graph with a single vertex set is called a
\emph{unipartite} graph.  If the vertices in a graph can be partitioned into two
disjoint sets such that all of the edges in the graph only connect vertices from one
set of the vertices of the other set, the graph is called a \emph{bipartite} graph.

Although any (unipartite) graph can be analyzed to determine whether it can be partitioned into a bipartite graph, 
in a program such a partitioning would be a run-time property of aa graph.  Of particular interest for realizing a bipartite graph is when the graph is structurally bipartite, that is when we are explicitly given two different sets of vertices and the corresponding set of edges that connect vertices of the two sets.  This common---and important---use case arises when modeling relationships between different types of entities.  For example, we might use a structurally bipartite graph in which one vertex set represents customers and another vertex set represents products.  An edge between a customer and a product would be used to indicate that a customer has purchased a particular product.

\andrew{$G = \{U, V, E\}$ or even $G = \{U, V, E, F\}$}


\subsection{BiPartite Graph Models}


\subsection{BiPartite Graph Implementations}


\section{Relationship between Graphs and Sparse Matrices}



\section{Naming Conventions}


Table \ref{tab:name_conv} shows the naming conventions used throughout this document. 

\begin{table}[h!]
\begin{center}
{\begin{tabular}{l l l p{7cm}}
\hline
    \textbf{Template} & & \textbf{Variable} & \\
    \textbf{Parameter} & \textbf{Type Alias} & \textbf{Names} & \textbf{Description} \\
\hline
    \tcode{G} & & & Graph \\
    & \tcode{graph_reference_t<G>} & \tcode{g} & Graph reference \\
    \tcode{GV} &  & \tcode{val} & Graph Value, value or reference \\
\hline
    \tcode{V} & \tcode{vertex_t<G>} & & Vertex \\
    & \tcode{vertex_reference_t<G>} & \tcode{u,v,x,y} & Vertex reference. \tcode{u} is the source (or only) vertex. \tcode{v} is the target vertex. \\
    \tcode{VId} & \tcode{vertex_id_t<G>} & \tcode{uid,vid,seed} & Vertex id. \tcode{uid} is the source (or only) vertex id. \tcode{vid} is the target vertex id. \\
    \tcode{VV} &  \tcode{vertex_value_t<G>} & \tcode{val} & Vertex Value, value or reference. This can be either the user-defined value on a vertex, or a value returned by a function object (e.g. \tcode{VVF}) that is related to the vertex. \\
    \tcode{VR} &  \tcode{vertex_range_t<G>} & \tcode{ur,vr} & Vertex Range \\
    \tcode{VI} & \tcode{vertex_iterator_t<G>} & \tcode{ui,vi} & Vertex Iterator. \tcode{ui} is the source (or only) vertex. \\
                    &  & \tcode{first,last} & \tcode{vi} is the target vertex. \\
    \tcode{VVF} & & \tcode{vvf} & Vertex Value Function: vvf(u) $\rightarrow$ value \\
\hline
    \tcode{E} & \tcode{edge_t<G>} & & Edge \\
    & \tcode{edge_reference_t<G>} & \tcode{uv,vw} & Edge reference. \tcode{uv} is an edge from vertices \tcode{u} to \tcode{v}. \tcode{vw} is an edge from vertices \tcode{v} to \tcode{w}.  \\
    \tcode{EV} &  \tcode{edge_value_t<G>} & \tcode{val} & Edge Value, value or reference. This can be either the user-defined value on an edge, or a value returned by a function object (e.g. \tcode{EVF}) that is related to the edge. \\
    \tcode{ER} &  \tcode{vertex_edge_range_t<G>} & & Edge Range for edges of a vertex \\
    \tcode{EI} & \tcode{vertex_edge_iterator_t<G>} & \tcode{uvi,vwi} & Edge Iterator for an edge of a vertex. \tcode{uvi} is an iterator for an edge from vertices \tcode{u} to \tcode{v}. \tcode{vwi} is an iterator for an edge from vertices \tcode{v} to \tcode{w}. \\
    \tcode{EVF} & & \tcode{evf} & Edge Value Function: evf(uv) $\rightarrow$ value \\
\hline
\end{tabular}}
\caption{Naming Conventions for Types and Variables}
\label{tab:name_conv}
\end{center}
\end{table}

\begin{comment}
\section{Graph Definition}
 A \textit{graph} \cite{REF_graph} $G = (V, E)$ is a set of \textit{vertices} \cite{REF_graph} $V$, \textbf{points} in a space, and \textit{edges} \cite{REF_graph} $E$, \textbf{links} between these vertices. Edges may or may not be \textbf{oriented}, that is, \textit{directed} \cite{REF_graph} or \textit{undirected} \cite{REF_graph}, respectively. Moreover, edges may be \textit{weighted} \cite{REF_graph}, that is, assigned a value. Both \textbf{static} and \textbf{dynamic} implementations of a graph exist, specifically a (static) \textbf{matrix}, each having the typical advantages and disadvantages associated with static and dynamic data structures.

\subsection{Adjacency List}
%% adjacency graph, vertex, edge
%% incidence, neighbors (adjacency)
%% optional user-defined values for edge, vertex, graph

\subsection{Edge List}
% range with source_id & target_id members, and optional value member(s)
% edgelist view

\end{comment}

\section{Examples}

\andrew{I am not sure these belong here.  None of the library details have actually
  been explained so there will be alot of code that won't have been explained.  E.g.,
  all of the CPOs.  The motivating Kevin Bacon example is self-contained and simple
  enough to motivate the rest of the introduction. }

\subsection{Implementing a Graph Algorithm}

\subsection{Implementing a Graph Type}

