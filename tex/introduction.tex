% algorithms: 
% views: 
% containers: adjacency matrix
% data model: incoming edges on a vertex

\section{Introduction}

\andrew{Where should we put our motivating example (Kevin Bacon game)?}


\subsection{Graph Background and Terminology}

We use commonly accepted terminology for graph data structures and algorithms and 
adopt the particular terminology used in the textbook by
Cormen, Leiserson, Rivest, and Stein (``CLRS'')~\cite{CLRS2022}.

A \emph{graph} $G$ is a set comprised of two other sets, the vertex set $V$ and the edge set $E$, 
typically expressed as $G=\{V,E\}$.  We can number the members of the vertex set
and write $V = \{ v_0, v_1, \ldots , v_{n-1} \}$.  Similarly, we can write $E = \{ e_0, e_1, \ldots, e_{m-1} \}$.  The number of elements in $V$ is denoted by $|V|$ and the number of elements in $E$ is denoted by $|E|$.  Edges in $E$ are pairs of vertices; for any $e_k \in E$, we write $e_k = ( v_i, v_j )$, where $v_i \in V$ and $v_j\in V$.  The edges in $E$ may be \emph{uordered}, in which case $(v_i, v_j) = (v_j, v_i)$ or \emph{ordered}, in which case $(v_i, v_j) \neq (v_j, v_i)$ (unless $i = j$).
A graph consisting of unordered edges is said to be \emph{undirected}; 
a graph consisting of ordered edges is said to be \emph{directed}.  Since there is a single set of vertices in this definition of $G$, we say $G$ is a \emph{unipartite} graph.


\subsection{Graph Models}

\andrew{The idea here is to illustrate the progression from data to a graph.}


\subsubsection{Edge List}

\andrew{Essentially a direct realization of $G=\{V, E\}$.}


\subsubsection{Nodes and Links}

\andrew{A direct realization of vertices (circles) connected by edges (arrows).}


\subsubsection{Adjacency List}

\andrew{Representation necessary for useful graph algorithms.  Present essentials of definition in CLRS.}


\subsection{Implementations}


\subsubsection{Edge List: Array of Structs / Struct of Arrays}

\andrew{Edges can be stored as tuples (or tuple like) basically in parallel arrays (like a ranges::zip).}


\subsubsection{Vertices and Edges: Node Objects and Link Objects}

\andrew{ala Stanford Graph Base (et al).}


\subsubsection{Adjacency List: Container of Containers}

\andrew{An adjacency list can be represented as a container of containers (e.g., std::vector<std::list>).  Talk about range of ranges when we talk about library interfaces, requirements, concepts.}


\subsubsection{Edge List and Adjacency List: Compressed Edge List}

\andrew{Using a sort and group-by (or a sort, a run-length encoding, and a scan), we can compactify the edge-list reprentation and at the same time obtain an adjacency-list representation -- one that is memory and compute efficient.  Best of both worlds.  Has same basic structural principles as CSR / CSC matrices in linear algebra -- but much more general.}


\subsection{BiPartite Graphs}

So far, we have been considering graphs where the vertices, and both elements of an
edge, are members of a single set $|V|$.  A graph with a single vertex set is called a
\emph{unipartite} graph.  If the vertices in a graph can be partitioned into two
disjoint sets such that all of the edges in the graph only connect vertices from one
set of the vertices of the other set, the graph is called a \emph{bipartite} graph.

Although any (unipartite) graph can be analyzed to determine whether it can be partitioned into a bipartite graph, 
in a program such a partitioning would be a run-time property of aa graph.  Of particular interest for realizing a bipartite graph is when the graph is structurally bipartite, that is when we are explicitly given two different sets of vertices and the corresponding set of edges that connect vertices of the two sets.  This common---and important---use case arises when modeling relationships between different types of entities.  For example, we might use a structurally bipartite graph in which one vertex set represents customers and another vertex set represents products.  An edge between a customer and a product would be used to indicate that a customer has purchased a particular product.

\andrew{$G = \{U, V, E\}$ or even $G = \{U, V, E, F\}$}


\subsubsection{BiPartite Graph Models}


\subsubsection{BiPartite Graph Implementations}


\subsection{Relationship between Graphs and Sparse Matrices}



\subsection{Naming Conventions}


Table \ref{tab:name_conv} shows the naming conventions used throughout this document. 

\begin{table}[h!]
\begin{center}
{\begin{tabular}{l l l p{7cm}}
\hline
    \textbf{Template} & & \textbf{Variable} & \\
    \textbf{Parameter} & \textbf{Type Alias} & \textbf{Names} & \textbf{Description} \\
\hline
    \tcode{G} & & & Graph \\
    & \tcode{graph_reference_t<G>} & \tcode{g} & Graph reference \\
    \tcode{GV} &  & \tcode{val} & Graph Value, value or reference \\
\hline
    \tcode{V} & \tcode{vertex_t<G>} & & Vertex \\
    & \tcode{vertex_reference_t<G>} & \tcode{u,v,x,y} & Vertex reference. \tcode{u} is the source (or only) vertex. \tcode{v} is the target vertex. \\
    \tcode{VId} & \tcode{vertex_id_t<G>} & \tcode{uid,vid,seed} & Vertex id. \tcode{uid} is the source (or only) vertex id. \tcode{vid} is the target vertex id. \\
    \tcode{VV} &  \tcode{vertex_value_t<G>} & \tcode{val} & Vertex Value, value or reference. This can be either the user-defined value on a vertex, or a value returned by a function object (e.g. \tcode{VVF}) that is related to the vertex. \\
    \tcode{VR} &  \tcode{vertex_range_t<G>} & \tcode{ur,vr} & Vertex Range \\
    \tcode{VI} & \tcode{vertex_iterator_t<G>} & \tcode{ui,vi} & Vertex Iterator. \tcode{ui} is the source (or only) vertex. \\
                    &  & \tcode{first,last} & \tcode{vi} is the target vertex. \\
    \tcode{VVF} & & \tcode{vvf} & Vertex Value Function: vvf(u) $\rightarrow$ value \\
\hline
    \tcode{E} & \tcode{edge_t<G>} & & Edge \\
    & \tcode{edge_reference_t<G>} & \tcode{uv,vw} & Edge reference. \tcode{uv} is an edge from vertices \tcode{u} to \tcode{v}. \tcode{vw} is an edge from vertices \tcode{v} to \tcode{w}.  \\
    \tcode{EV} &  \tcode{edge_value_t<G>} & \tcode{val} & Edge Value, value or reference. This can be either the user-defined value on an edge, or a value returned by a function object (e.g. \tcode{EVF}) that is related to the edge. \\
    \tcode{ER} &  \tcode{vertex_edge_range_t<G>} & & Edge Range for edges of a vertex \\
    \tcode{EI} & \tcode{vertex_edge_iterator_t<G>} & \tcode{uvi,vwi} & Edge Iterator for an edge of a vertex. \tcode{uvi} is an iterator for an edge from vertices \tcode{u} to \tcode{v}. \tcode{vwi} is an iterator for an edge from vertices \tcode{v} to \tcode{w}. \\
    \tcode{EVF} & & \tcode{evf} & Edge Value Function: evf(uv) $\rightarrow$ value \\
\hline
\end{tabular}}
\caption{Naming Conventions for Types and Variables}
\label{tab:name_conv}
\end{center}
\end{table}

\begin{comment}
\subsection{Graph Definition}
 A \textit{graph} \cite{REF_graph} $G = (V, E)$ is a set of \textit{vertices} \cite{REF_graph} $V$, \textbf{points} in a space, and \textit{edges} \cite{REF_graph} $E$, \textbf{links} between these vertices. Edges may or may not be \textbf{oriented}, that is, \textit{directed} \cite{REF_graph} or \textit{undirected} \cite{REF_graph}, respectively. Moreover, edges may be \textit{weighted} \cite{REF_graph}, that is, assigned a value. Both \textbf{static} and \textbf{dynamic} implementations of a graph exist, specifically a (static) \textbf{matrix}, each having the typical advantages and disadvantages associated with static and dynamic data structures.

\subsubsection{Adjacency List}
%% adjacency graph, vertex, edge
%% incidence, neighbors (adjacency)
%% optional user-defined values for edge, vertex, graph

\subsubsection{Edge List}
% range with source_id & target_id members, and optional value member(s)
% edgelist view

\end{comment}

\subsection{Examples}

\andrew{I am not sure these belong here.  None of the library details have actually
  been explained so there will be alot of code that won't have been explained.  E.g.,
  all of the CPOs.  The motivating Kevin Bacon example is self-contained and simple
  enough to motivate the rest of the introduction. }

\subsubsection{Implementing a Graph Algorithm}

\subsubsection{Implementing a Graph Type}

