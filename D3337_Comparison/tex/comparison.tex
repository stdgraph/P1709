%% \chapter{Comparison}

\clearpage

\section{Syntax Comparison} \label{syntax}
We provide a usage syntax comparison of several graph algorithms
in Tier 1 of P3128 against the \textbf{boost::graph} equivalent.
We refer to the refrence implementation associated with this proposal
as \textbf{stdgraph}.
These algorithms are breadth-first search (BFS, Figure~\ref{fig:bfssyntax}),
connected components (CC, Figure~\ref{fig:ccsyntax}),
single sourced shortest paths (SSSP, Figure~\ref{fig:ssspsyntax}),
and triangle counting (TC)(\ref{fig:tcsyntax}).
We take these algorithms from the GAP Benchmark Suite~\cite{gapbs_2023}
which we discuss more in Section~\ref{performance}.

Unlike \textbf{boost::graph}, \textbf{stdgraph} does not specify edge directedness
as a graph property.
If a graph in \textbf{stdgraph} implemented by \textbf{container::compressed\_graph}
is undirected, then it will contain edges in both directions.
\textbf{boost::graph} has a \textbf{boost::graph::undirectedS} property
which can be used in the \textbf{boost::graph::adjacency\_matrix} class
to specify an unidrected graph, but
not in the \textbf{boost::graph::compressed\_sparse\_row\_graph} class.
Thus in Figures~\ref{fig:bfssyntax}-\ref{fig:tcsyntax}, the graph type always includes \textbf{boost::graph::directedS}.
Similarly to \textbf{stdgraph}, undirected graphs must contain the edges in both directions.

Intermediate data structures (i.e. edgelists) will be needed
to construct the compressed graph structures.
In order to focus on the differenes in algorithm syntax, we omit
code which populates the graph data structures.
In the following sections we address the syntax changes for each of
these algorithms.

\begin{figure}[ht]
\noindent\begin{minipage}{.499\textwidth}
{\small
      \lstinputlisting{D3337_Comparison/src/bgl_bfs.hpp}
}
\end{minipage}\hfill
\begin{minipage}{.499\textwidth}
{\small
      \lstinputlisting{D3337_Comparison/src/stdgraph_bfs.hpp}
}
\end{minipage}
\caption{Breadth\-First Search Syntax Comparison}
\label{fig:bfssyntax}
\end{figure}
\begin{figure}[ht]
\noindent\begin{minipage}{.499\textwidth}
{\small
      \lstinputlisting{D3337_Comparison/src/bgl_cc.hpp}
}
\end{minipage}\hfill
\begin{minipage}{.499\textwidth}
{\small
      \lstinputlisting{D3337_Comparison/src/stdgraph_cc.hpp}
}
\end{minipage}
\caption{Connected Components Syntax Comparison}
\label{fig:ccsyntax}
\end{figure}

\begin{figure}[ht]
\noindent\begin{minipage}{.499\textwidth}
{\small
      \lstinputlisting{D3337_Comparison/src/bgl_sssp.hpp}
}
\end{minipage}\hfill
\begin{minipage}{.499\textwidth}
{\small
      \lstinputlisting{D3337_Comparison/src/stdgraph_sssp.hpp}
}
\end{minipage}
\caption{Single Source Shortest Paths (Dijkstra) Syntax Comparison}
\label{fig:ssspsyntax}
\end{figure}

\begin{figure}[ht]
\noindent\begin{minipage}{.499\textwidth}
{\small
      \lstinputlisting{D3337_Comparison/src/bgl_tc.hpp}
}
\end{minipage}\hfill
\begin{minipage}{.499\textwidth}
{\small
      \lstinputlisting{D3337_Comparison/src/stdgraph_tc.hpp}
}
\end{minipage}
\caption{TC Syntax Comparison}
\label{fig:tcsyntax}
\end{figure}

\subsection{BFS}
BFS is often described as a graph algorithm, though a BFS traversal
by itself does not actually perform any task.
In reality, it is a data access pattern which specifies an order
vertices and edges should be processed by some higher level algorithm.
\textbf{boost::graph} provided a very customizable interface to this
data access pattern through the use of visitors which allows users
to customize function calls during BFS events.
For example discover\_vertex is called when a vertex is encountered for the
first time; examine\_vertex is called when a vertex is popped from the queue;
examine\_edge is called on each edge of a vertex when it is discovered, etc.

This capability is very powerful but often cumbersome if the BFS traversal
simply requires vertex and edge access upon visiting.
For this reason stdgraph provides a simple, range-based-for loop BFS traversal
called a view.
Figure~\ref{fig:bfssyntax} compares the most simple \textbf{boost::graph}
BFS visitor against the range-based-for loop implementation.
The authors of this proposal acknowledge that some power users still want
the full customization provided by visitors,
and we plan to add them to this proposal.

\subsection{CC}
There is very little difference in the connected component interfaces.

\subsection{SSSP}
Of the four algorithms discussed here, only SSSP makes use of some edge property, in this case distance.
Along with the input edge property, the algorithm also associates with
every vertex a distance from the start vertex, and a predecessor
vertex to store the shortest path.
In Figure~\ref{fig:ssspsyntax} we see that \textbf{boost::graph} requires
property maps to lookup edge and vertex properties.
These property maps are tightly coupled with the graph data strucutres.
We propose properties be stored external to the graph.
For edge properties we provide a weight lambda function to the algorithm
to lookup distance from the \textbf{edge\_reference\_t}.

\subsection{TC}
\textbf{boost::graph} does not contain a global triangle counting
similar to the one proposed by stdgraph.
Instead we must iterate through the vertices counting the number of triangles
on every vertex, and adjust for overcounting at the end.







\section{Performance Comparison} \label{performance}
To evaluate the performance of this proposed library, we compare its reference implementation
(stdgraph) against BGL and NWGraph on a subset of the GAP Benchmark Suite\cite{gapbs_2023}.
This comparison includes four of the five GAP algorithms that are in the tier 1 algorithm list of this proposal:
triangle counting (TC), weak connected components (CC), breadth-first search (BFS),
and single-source shortest paths (SSSP).
Table~\ref{tab:gap_graphs} summarizes the graphs specified by the GAP benchmark.
These graphs were chosen to be large but still fit on shared memory machines and have edge counts in the billions.
We compare to BGL because it the commonly used sequential C++ graph library as described above.
NWGraph was implemented with many of the ideas of this proposal in mind, and we expect very similar performance
between NWGraph and this reference implementation.

\begin{table}[h!]
\centering
\begin{tabular}{c c c c c c c}
Name & Description & \#Vertices & \#Edges & Degree & (Un)directed & References \\
     &             & (M)        & (M)     & Distribution & & \\\hline
road & USA road network & 23.9 & 57.7 & bounded & undirected & \\\hline
Twitter & Twitter follower links & 61.6 & 1,468.4 & power & directed & \\\hline
web & Web crawl of .sk domain & 50.6 & 1,930.3 & power & directed &\\\hline
kron & Synthetic graph & 134.2 & 2,111.6 & power & undirected & \\\hline
urand & Uniform random graph & 134.2 & 2,147.5 & normal & undirected & \\\hline
\end{tabular}
\caption{Summary of GAP Benchmark Graphs}
\label{tab:gap_graphs}
\end{table}

The NWGraph authors published a similar comparison to BGL\cite{REF_nwgraph_library} in which they
demonstrated performance improvement of NWGraph over BGL.
To simplify experimental setup, we rerun these new experiments using the same machine used in\cite{REF_nwgraph_library},
(compute nodes consisting of two Intel® Xeon® Gold 6230 processors, each with 20 physical cores running at 2.1 GHz,
and 188GB of memory per processor).
NWGraph and stdgraph were compiled with gcc 13.2 using -Ofast -march=native compilation flags.

Even though NWGraph contains an implmentation of Dijkstra, the SSSP results in \cite{REF_nwgraph_library}
were based on delta-stepping. For this comparison, stdgraph and NWgraph both use Dijkstra.
The NWGraph and stdgraph implementation of CC is based on the Afforest \cite{sutton2018optimizing} algorithm.
While BFS and SSSP implementations are very similar for NWGraph and stdgraph, the latter contains
support for event-based visitors, and it is immportant to make sure this does not incur a performance penalty.
Table~\ref{tab:performance_numbers} summarizes our GAP benchmark results for stdgraph compared to BGL and NWGraph.

\begin{table}[h!]
\centering
\begin{tabular}{ c c c c c c c }
Algorithm & Library & road & twitter & kron & web & urand \\
\hline
\multirow{3}{*}{BFS} & BGL & 1.09s & 12.11s & 54.80s & 5.52s & 73.26s \\
& NWGraph & 0.91s & 11.25s & 38.86s & 2.37s & 64.63s \\
& stdgraph & 1.39s & 8.54s & 16.34s & 3.52s & 62.75s \\
\hline
\multirow{3}{*}{CC} & BGL & 1.36s & 21.96s & 81.18s & 6.64s & 134.23s \\
& NWGraph & 1.05s & 3.77s & 10.16s & 3.04s & 36.59s \\
& stdgraph & 0.78s & 2.81s & 8.37s & 2.23s & 33.75s \\
\hline
\multirow{3}{*}{SSSP} & BGL & 4.03s & 47.89s & 167.20s & 28.29s & OOM \\
& NWGraph & 3.63s & 109.37s & 344.12s & 35.58s & 400.23s \\
& stdgraph & 4.22s & 79.75s & 211.37s & 33.87s & 493.15s \\
\hline
\multirow{3}{*}{TC} & BGL & 1.34s & >24H & >24H & >24H & 4425.54s \\
& NWGraph & 0.41s & 1327.63s & 6840.38s & 131.47s & 387.53s \\
& stdgraph & 0.17s & 459.08s & 2357.95s & 50.04s & 191.36s \\
\hline
\end{tabular}
\caption{GAP Benchmark Performance: Time for GAP benchmark algorithms is shown for Boost Graph Library, NWGraph, and this proposal's reference implementation (stdgraph)}
\label{tab:performance_numbers}
\end{table}
