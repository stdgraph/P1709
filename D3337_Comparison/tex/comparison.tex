%% \chapter{Comparison}
\clearpage

\section{Syntax Comparison} \label{syntax}
We refer to the reference implementation associated with this proposal as 
\stdgraph. In this section, we provide a usage syntax comparison of several 
graph algorithms in Tier 1 of P3128 against the equivalent implementations in 
\bgl~\cite{BGL} and the more recent \nwgraph~\cite{REF_nwgraph_paper,REF_nwgraph_library}.
These algorithms are breadth-first search (BFS, Figure~\ref{fig:bfssyntax}),
connected components (CC, Figure~\ref{fig:ccsyntax}),
single source shortest paths (SSSP, Figure~\ref{fig:ssspsyntax}),
and triangle counting (TC, Figure~\ref{fig:tcsyntax}).
We take these algorithms from the GAP Benchmark Suite~\cite{beamer2015gap}.
We defer to later sections any discussion of
underlying implementation details and resulting performance.

Unlike \bgl, \stdgraph does not specify edge direction as a graph property.
If a graph in \stdgraph implemented by \tcode{container::compressed\_graph}
is undirected, then it will contain distinct edges in both directions.
\bgl has a \tcode{boost::graph::undirectedS} property
which can be used in the \tcode{boost::graph::adjacency\_matrix} class
to specify an undirected graph, but
not in the \tcode{boost::graph::compressed\_sparse\_row\_graph} class.
Thus in Figures~\ref{fig:bfssyntax}-\ref{fig:tcsyntax}, the \bgl graph type 
always includes \tcode{boost::graph::directedS}.
Similar to \stdgraph, undirected graphs must contain the edges in both directions.

Intermediate data structures (e.g., edge lists) will be needed
to construct the compressed graph structures.
In order to focus on the differences in algorithm syntax, we omit
code which populates the graph data structures.
In the following subsections, we address the syntax differences for each of
these algorithms.

\begin{figure}[ht]
\noindent\begin{minipage}{.499\textwidth}
{\small
      \lstinputlisting{D3337_Comparison/src/bgl_bfs.hpp}
}
\end{minipage}\hfill
\begin{minipage}{.499\textwidth}
{\small
      \lstinputlisting{D3337_Comparison/src/stdgraph_bfs.hpp}
}
\end{minipage}
\caption{Breadth-First Search Syntax Comparison}
\label{fig:bfssyntax}
\end{figure}
\begin{figure}[ht]
\noindent\begin{minipage}{.499\textwidth}
{\small
      \lstinputlisting{D3337_Comparison/src/bgl_cc.hpp}
}
\end{minipage}\hfill
\begin{minipage}{.499\textwidth}
{\small
      \lstinputlisting{D3337_Comparison/src/stdgraph_cc.hpp}
}
\end{minipage}
\caption{Connected Components Syntax Comparison}
\label{fig:ccsyntax}
\end{figure}

\begin{figure}[ht]
\noindent\begin{minipage}{.499\textwidth}
{\small
      \lstinputlisting{D3337_Comparison/src/bgl_sssp.hpp}
}
\end{minipage}\hfill
\begin{minipage}{.499\textwidth}
{\small
      \lstinputlisting{D3337_Comparison/src/stdgraph_sssp.hpp}
}
\end{minipage}
\caption{Single Source Shortest Paths Syntax Comparison}
\label{fig:ssspsyntax}
\end{figure}

\begin{figure}[ht]
\noindent\begin{minipage}{.499\textwidth}
{\small
      \lstinputlisting{D3337_Comparison/src/bgl_tc.hpp}
}
\end{minipage}\hfill
\begin{minipage}{.499\textwidth}
{\small
      \lstinputlisting{D3337_Comparison/src/stdgraph_tc.hpp}
}
\end{minipage}
\caption{Triangle Counting Syntax Comparison}
\label{fig:tcsyntax}
\end{figure}

\subsection{Breadth-First Search}
BFS is often described as a graph algorithm, though a BFS traversal
by itself does not actually perform any task.
In reality, it is a data access pattern which specifies an order
vertices and edges should be processed by some higher level algorithm.
\bgl provided a very customizable interface to this
data access pattern through the use of visitors which allows users
to customize function calls during BFS events.
For example discover\_vertex is called when a vertex is encountered for the
first time; examine\_vertex is called when a vertex is popped from the queue;
examine\_edge is called on each edge of a vertex when it is discovered, etc.

This capability is very powerful but often cumbersome if the BFS traversal
simply requires vertex and edge access upon visiting.
For this reason \stdgraph provides a simple, range-based-for loop BFS traversal
called a view.
Figure~\ref{fig:bfssyntax} compares the most simple \bgl
BFS visitor against the range-based-for loop implementation.
The authors of this proposal acknowledge that some power users still want
the full customization provided by visitors,
and we plan to add them to this proposal.

\subsection{Connected Components}
There is very little difference in the connected component interfaces.

\subsection{Single Source Shortest Paths}
Of the four algorithms discussed here, only SSSP makes use of some
edge property, in this case distance.
Along with the input edge property, the algorithm also associates with
every vertex a distance from the start vertex, and a predecessor
vertex to store the shortest path.
In Figure~\ref{fig:ssspsyntax} we see that \bgl requires
property maps to lookup edge and vertex properties.
These property maps are tightly coupled with the graph data structures.
We propose properties be stored external to the graph.
For edge properties we provide a weight lambda function to the algorithm
to lookup distance from the \tcode{edge\_reference\_t}.

\subsection{Triangle Counting}
\bgl does not contain a global triangle counting
similar to the one proposed by \stdgraph.
Instead we must iterate through the vertices counting the number of triangles
on every vertex, and adjust for overcounting at the end.

\clearpage

\section{Performance Comparison} \label{performance}
\subsection{Experimental Setup}
To evaluate the performance of this proposed library, we compare its reference implementation
(\stdgraph) against \bgl and \nwgraph on a subset of the GAP Benchmark Suite\cite{gapbs_2023}.
This comparison includes four of the five GAP algorithms that are in the tier 1 algorithm list of this proposal:
triangle counting (TC), connected components (CC), breadth-first search (BFS),
and single-source shortest paths (SSSP).
Table~\ref{tab:gap_graphs} summarizes the graphs specified by the GAP benchmark.
These graphs were chosen to be large but still fit on shared memory machines and have edge counts in the billions.
We compare to BGL because it the commonly used sequential C++ graph library as described above.
\nwgraph was implemented with many of the ideas of this proposal in mind, and we expect very similar performance
between \nwgraph and this reference implementation.

\begin{table}[h!]
\centering
\begin{tabular}{c c c c c c c}
Name & Description & \#Vertices & \#Edges & Degree & (Un)directed & References \\
     &             & (M)        & (M)     & Distribution & & \\\hline
road & USA road network & 23.9 & 57.7 & bounded & undirected & \cite{dimacs9th}\\\hline
Twitter & Twitter follower links & 61.6 & 1,468.4 & power & directed & \cite{Twitter}\\\hline
web & Web crawl of .sk domain & 50.6 & 1,930.3 & power & directed & \cite{LAW1}\\\hline
kron & Synthetic graph & 134.2 & 2,111.6 & power & undirected & \cite{Graph500} \\\hline
urand & Uniform random graph & 134.2 & 2,147.5 & normal & undirected & \cite{Erdos}\\\hline
\end{tabular}
\caption{Summary of GAP Benchmark Graphs}
\label{tab:gap_graphs}
\end{table}

The \nwgraph authors published a similar comparison to BGL\cite{REF_nwgraph_library} in which they
demonstrated performance improvement of \nwgraph over BGL.
To simplify experimental setup, we rerun these new experiments using the same machine used in\cite{REF_nwgraph_library},
(compute nodes consisting of two Intel® Xeon® Gold 6230 processors, each with 20 physical cores running at 2.1 GHz,
and 188GB of memory per processor).
\nwgraph and \stdgraph were compiled with gcc 13.2 using -Ofast -march=native compilation flags.

Even though \nwgraph contains an implementation of Dijkstra, the SSSP results in \cite{REF_nwgraph_library}
were based on delta-stepping. For this comparison, \stdgraph and \nwgraph both use Dijkstra.
The \nwgraph implementations also used a version of SSSP which did not compute
a predecessor map, only providing the final distances.
\stdgraph provides SSSP without predecessors called $dijkstra\_shortest\_distances$ which is similar to the Dijkstra in
Figure~\ref{fig:ssspsyntax} with the predecessor argument omitted.
\bgl can also compute shortest distances only by omitting the predecessor map.
We use the shortest distance version for these experiments.

The \nwgraph and \stdgraph implementation of CC is based on the Afforest \cite{sutton2018optimizing} algorithm.
While BFS and SSSP implementations are very similar for \nwgraph and \stdgraph, the latter contains
support for event-based visitors.
If this functionality is not required it should be optimized out and not
incur a performance penalty,
but we seek to verify this experimentally.
\nwgraph and \stdgraph contain similar implementations of triangle
counting which perform a set intersection of the neighbor list of vertices
$u$ and $v$, only if $v$ is a neighbor of $u$.
By first performing a lexicographic sort of the vertex ids of the adjacency
structure, the set intersection is limited to neighbors with vertex ids greater
than $u$ and $v$, or equivalently the upper triangular portion of the adjacency
matrix.
Table~\ref{tab:performance_numbers} summarizes our GAP benchmark results for \stdgraph compared to \bgl and \nwgraph.

\begin{table}[h!]
\centering
\begin{tabular}{ c c c c c c c }
Algorithm & Library & road & twitter & kron & web & urand \\
\hline
\multirow{3}{*}{BFS} & \bgl & 1.09s & 12.11s & 54.80s & 5.52s & 73.26s \\
& \nwgraph & 0.91s & 11.25s & 38.86s & 2.37s & 64.63s \\
& \stdgraph & 1.39s & 8.54s & 16.34s & 3.52s & 62.75s \\
\hline
\multirow{3}{*}{CC} & BGL & 1.36s & 21.96s & 81.18s & 6.64s & 134.23s \\
& \nwgraph & 1.05s & 3.77s & 10.16s & 3.04s & 36.59s \\
& \stdgraph & 0.78s & 2.81s & 8.37s & 2.23s & 33.75s \\
\hline
\multirow{3}{*}{SSSP} & BGL & 4.03s & 47.89s & 167.20s & 28.29s & OOM \\
& \nwgraph & 3.63s & 109.37s & 344.12s & 35.58s & 400.23s \\
& \stdgraph & 4.22s & 79.75s & 211.37s & 33.87s & 493.15s \\
\hline
\multirow{3}{*}{TC} & BGL & 1.34s & >24H & >24H & >24H & 4425.54s \\
& \nwgraph & 0.41s & 1327.63s & 6840.38s & 131.47s & 387.53s \\
& \stdgraph & 0.17s & 459.08s & 2357.95s & 50.04s & 191.36s \\
\hline
\end{tabular}
\caption{GAP Benchmark Performance: Time for GAP benchmark algorithms is shown for \bgl, \nwgraph, \stdgraph}
\label{tab:performance_numbers}
\end{table}

\subsection{Experimental Analysis}
BFS results are consistent between the three implementations,
except for the kron graph where \stdgraph is 2.4x faster
than \nwgraph and 3.4x faster than \bgl.

CC results are consistent between \nwgraph and \stdgraph, which
are both much faster than \bgl on twitter, kron, and urand.
This is reasonable as \bgl is using a simple breadth-first
search based CC algorithm while the other two implementations use the
Afforest algorithm.
Of the four algorithms, CC shows the closest agreement between \nwgraph
and \stdgraph.

SSSP results are more mixed, with differing performance on twitter and kron.
Interestingly of the algorithms we profile, this is the only one where
\bgl is often faster than the other implementations,
faster than \stdgraph by 1.7x on twitter and 1.3x on kron, though
failing by running out of memory on urand.

TC performance from the na\"ive \bgl implementation
is far slower than the adjacency matrix set intersection used by \nwgraph
and \stdgraph.
Since the same triangle is counted 6 times in \bgl,
we expect at least that much of a slowdown, but in fact the slowdown
is often much worse.
However the TC results are concerning because the \stdgraph
performance is around 2x that of \nwgraph.
We plan to review the implementation details to discover the cause of
this discrepancy.

\section{Memory Allocation}
Unlike existing STL algorithms, the graph algorithms we propose here
will often require their own memory allocations.
Table~\ref{tab:internalmem} records the internal memory allocations
required for our implementations of the GAP Benchmark algorithms
where relevant.
It is important to note that the memory usage is not prescribed
by the algorithm interface in P3128, and is ultimately up to the
library implementer.
Some memory use, such as the queues in BFS and SSSP, will
probably be common to most implementations.
However, the color map in BFS and the reindex map in CC
(used to ensure the resulting component indices are contiguous)
could potentially be avoided.

\begin{table}[h!]
\centering
\begin{tabular}{| c | c | c |}
\hline
Algorithm & Required Member Data & Max Size \\\hline
BFS  & queue          & $O(|V|)$ \\
     & color map      & V \\\hline
CC   & reindex map    & $O(|components|)$ \\\hline
SSSP & priority queue & $O(|E|)$\\\hline
TC   & None          & N\/A\\
\hline
\end{tabular}
\caption{Memory Allocations of GAP Benchmark Algorithm Implementations}
\label{tab:internalmem}
\end{table}
