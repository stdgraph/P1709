\section{Revision History}

\subsection*{\paperno r0}

\begin{itemize}
      \item Split from P1709r5. Added \textit{Getting Started} section.
      \item Added A*, Best-first search and Adamic-Adar Index to Tier 2 algorithms based on input.
      \item Removed allocator parameters for consistency with existing algorithms. It was observed that \tcode{stable_sort} 
            allocates memory, but does not take an allocator parameter.
      \item Removed exception throwing from algorithms to support free-standing C++. The caller will need to
            follow the preconditions to avoid undefined behavior. The other option considered was to return
            an error code.
      \item Identify all \tcode{concept} definitions as "For exposition only" until we have consensus of whether they 
            belong in the standard or not.
\end{itemize}

\subsection*{\paperno r1}
\begin{itemize}
      \item Create new \textit{Traversal} section and move Breadth-First Search and Topological Sort algorithms to it.
            Also added new Depth-First Search algorithm to it.
      \item Revise the Dijkstra and Bellman-Ford shortest-path and shortest-distance algorithms
            \begin{itemize}
                  \item Add a visitor parameter to allow the caller to
                        customize the behavior of the algorithm without having to modify the algorithm itself.
                        The visitor functions are member functions on a user-defined class that must match the concept for the event.
                        The visitor events mimic those used in the Boost Graph Library for each algorithm.
                  \item Remove overloads that excluded Compare and Combine functions because they don't add much value,
                        and to keep the proposal small.
                  \item Add overloads for multiple sources. This is particularly important for Bellman-Ford to avoid
                        repeated calls to the algorithm that would make an already slow algorithm even slower.
                  \item Change "invalid distance" to "infinite distance" to reflect how the value is used 
                        in the algorithm.
            \end{itemize}
      \item Add the ability to detect and find the negative weight cycle in the Bellman-Ford algorithm.
\end{itemize}
