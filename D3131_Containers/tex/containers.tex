
\clearpage
%% \chapter{Graph Container Implementations}

\section{compressed\_graph Graph Container}
\tcode{compressed_graph} is a graph container being proposed for the standard library. It is a high-performance data structure that 
uses \href{https://en.wikipedia.org/wiki/Sparse_matrix#Compressed_sparse_row_\%28CSR\%2C_CRS_or_Yale_format\%29}{Compressed Sparse Row} 
(CSR) format to store its vertices, edges and associated values. Once constructed, vertices and edges cannot be added or deleted 
but values on vertices and edges can be modified.

There are a number of features added beyond the typical CSR implmentation:
\begin{itemize}
    \item   \textbf{User-defined values} Values can be associated with edge, vertices and and the graph itself by defining the 
            \tcode{EV}, \tcode{VV}, and \tcode{GV} template arguments respectively. If the type is void, no memory overhead is 
            incurred.
    \item   \textbf{Index type sizes} The size of the integral indexes into the internal row and column structures can be controlled
            by the \tcode{VId} and \tcode{EIndex} template arguments respectively to give a balance between storage and performance.
    \item   \textbf{Multi-partite graphs} The vertices can optionally be partitioned into multiple partitions by passing the 
            starting vertex id of each partition in the \tcode{part_start_ids} parameter in the constructors.
\end{itemize}

The following listings shows the prototype for the \tcode{compressed_graph} when the graph value type \tcode{GV} is non-\tcode{void} and a 
class template specialization when it is \tcode{void}.

Only the constuctors shown for \tcode{compressed_graph} are public. All other types or functions are only accessible through the types 
and functions in the Graph Container Interface.

\begin{table}[h]
    \setcellgapes{3pt}
    \makegapedcells
    \centering
    \begin{tabular}{|P{0.38\textwidth}|P{0.35\textwidth}|P{0.25\textwidth}|}
    \hline
    \textbf{vertex\_id assignment:} Contiguous & \textbf{\tcode{has_edge(g)}} $O(1)$ & \textbf{Append vertices?} No \\
    \textbf{Vertices range:} Contiguous & \textbf{\tcode{num_edges(g)}} $O(1)$ & \textbf{Append edges?} No \\
    \textbf{Edge range:} Contiguous & \textbf{\tcode{partition_id(g,uid)}} $O(log(P+1))$ & \textbf{Partions?} Yes\\
    %O(log(P+1)) is the complexity of std::upper_bound; +1 is for the number of vertices added at the end of partition_start_ids
    \hline
    \end{tabular}
    %\caption{Jaccard Coefficient Summary}
    \label{tab:compressed_graph_summary}
\end{table}
Note: $P$ is the number of partitions. It has a minimum of 1 for a unipartite graph, and is exepected to be small for all
reasonable cases.

\phil{tilde in destructor is giving problems in latex}

\phil{What is allowed/required for EV, VV, GV? Default constructible, ...? Create concept, or describe in mandates or preconditions. Use tuple value as starting point.}

\subsection{compressed\_graph when \tcode{GV} is not \tcode{void}}
{\small
      \lstinputlisting{D3131_Containers/src/compressed_graph.hpp}
}

\begin{itemdescr}
    \pnum\mandates
        \begin{itemize}
            \item The \tcode{EProj} template argument must be a projection that returns a value of \tcode{copyable_edge<VId, true, EV>} type
                  given a value of \tcode{erng}. If the value type of \tcode{Erng} is already a \tcode{copyable_edge<VId, true, EV>} type, 
                  then \tcode{EProj} can be \tcode{identity}.
            \item The \tcode{VProj} template argument must be a projection that returns a value of \tcode{copyable_vertex<VId, VV>} type,
                  given a value of \tcode{vrng}. If the value type of \tcode{Vrng} is already a \tcode{copyable_vertex<VId, VV>} type, 
                  then \tcode{VProj} can be \tcode{identity}.
        \end{itemize}
    \pnum\preconditions
        \begin{itemize}
            \item The \tcode{VId} template argument must be able to store a value of |V|+1, where |V| is the
                  number of vertices in the graph. The size of this type impacts the size of the \textit{edges}.
            \item The \tcode{EIndex} template argument must be able to store a value of |E|+1, where |E| is the
                  number of edges in the graph. The size of this type impact the size of the \textit{vertices}.
            \item The \tcode{EProj} and \tcode{VProj} template arguments must be valid projections.
            \item The \tcode{part_start_ids} range includes the starting vertex id for each partition. If it is empty, then the graph is single-partite
                    and the number of partitions is 1. If it is not empty, then the number of partitions is the size of the range, where the first element
                    must be 0 and all remaining elements must be in ascending order, and not exceed the number of vertices in the graph. Any
                    violation of these conditions results in undefined behavior.
        \end{itemize}
    \pnum\effects
        \begin{itemize}
            \item When \tcode{EV}, \tcode{VV}, or \tcode{GV} are \tcode{void}, no extra memory overhead is incurred for it.
            \item The \tcode{VId} and \tcode{EIndex} template arguments impact the internal storage requirements and performance.
                  The default of \tcode{uint32_t} is sufficient for most graphs and provides a good balance between storage and performance.
        \end{itemize}
    %\pnum\result
    %\pnum\returns \lstinline{void} \\
    %\pnum\throws \tcode{out_of_range} is thrown when \tcode{source} is not in the range \tcode{0 <= source < num_vertices(graph)}.  \\
    %\pnum\complexity \\
    \pnum\remarks 
        \begin{itemize}
            \item The default allocator type of \tcode{allocator<VId>} is rebound for different internal containers.
            \item The complexity for \tcode{num_edges(g)} and \tcode{has_edge(g)} is $O(1)$.
        \end{itemize}
%\pnum\errors
\end{itemdescr}

\subsection{compressed\_graph specialization when \tcode{GV} is \tcode{void}}
When \tcode{GV} is void the number of constructors decreases significantly as shown in the following listing.
It is a strict subset so there is no need to repeat the mandates, preconditions, effects, and remarks.

{\small
      \lstinputlisting{D3131_Containers/src/compressed_graph_gvoid.hpp}
}

\section{Using Existing Graph Data Structures}
Reasonable defaults have been defined for the GCI functions to minimize the amount of work
needed to adapt an existing graph data structure to be used by the views and algorithms.

There are two cases supported. The first is for the use of standard containers to define the graph and the other
is for a broader set of more complicated implementations.

\subsection{Using Standard Containers for the Graph Data Structure}

For example this we'll use \tcode{G = vector<forward_list<tuple<int,double>>>} to define the graph, where \tcode{g} 
is an instance of \tcode{G}. \tcode{tuple<int,double>} defines the target\_id and weight property respectively. We
can write loops to go through the vertices, and edges within each vertex, as follows.

\begin{lstlisting}
using G = vector<forward_list<tuple<int,double>>>;
auto weight = [&g](edge_t& uv) { return get<1>(uv); }

G g;
load_graph(g, ...); // load some data

// Using GCI functions
for(auto&& [uid, u] : vertices(g)) {
  for(auto&& [vid, uv]: edges(g,u)) {
    auto w = weight(uv);
    // do something...
  }
}
\end{lstlisting}

Note that no function override was required and is a special case when the outer range is a \tcode{random_access_range} and
and inner inner range is a \tcode{forward_range}, and the value type of the inner range is either \tcode{integral}
 or \tcode{tuple<integral, ...>}. This extends to any range type. For instance, boost::containers can be used
 just as easily as std containers. 

\begin{table}[h!]
    \begin{center}
    \resizebox{\textwidth}{!}
    {\begin{tabular}{l l p{1.5cm} L{7.0cm}}
    \hline
        \textbf{Function or Value} & \textbf{Concrete Type} \\
    \hline
        \tcode{vertices(g)} & \tcode{vector<forward_list<tuple<int,double>>>} (when \tcode{random_access_range<G>}) \\
        \tcode{u} & \tcode{forward_list<tuple<int,double>>} & \\
        \tcode{edges(g,u)} & \tcode{forward_list<tuple<int,double>>} (when \tcode{random_access_range<vertex_range_t<G>>}) \\
        \tcode{uv} & \tcode{tuple<int,double>} & \\
        \tcode{edge_value(g,uv)} & \tcode{tuple<int,double>} (when \tcode{random_access_range<vertex_range_t<G>>}) \\
        \tcode{target_id(g,uv)} & \tcode{integral}, when \tcode{uv} is either \tcode{integral}  or \tcode{tuple<integral,...>} \\
    \hline
    \end{tabular}}
    \caption{Types When Using Standard Containers}
    \label{tab:simple_graph}
    \end{center}
\end{table}


\subsection{Using Other Graph Data Structures}
For other graph data structures more function overrides are required. Table \ref{tab:cmn_cpo_overrides} shows the 
common function overrides anticipated for most cases, keeping in mind that all functions can be overridden.
When they are defined they must be in the same namespace as the data structures.

\begin{table}[h!]
    \begin{center}
    %\resizebox{\textwidth}{!}
    {\begin{tabular}{l l p{1.5cm} L{7.0cm}}
    \hline
        \textbf{Function} & \textbf{Comment} \\
    \hline
        \tcode{vertices(g)} & \\
        \tcode{edges(g,u)} & \\
        \tcode{target_id(g,uv)} & \\
        \tcode{edge_value(g,uv)} & If edges have value(s) in the graph \\
        \tcode{vertex_value(g,u)} & If vertices have value(s) in the graph \\
        \tcode{graph_value(g)} & If the graph has value(s) \\
    \hline
        \multicolumn{2}{c}{When edges have the optional source\_id on an edge} \\
    \hdashline
        \tcode{source_id(g,uv)} & \\
    \hline
        \multicolumn{2}{c}{When the graph supports multiple partitions} \\
    \hdashline
        \tcode{num_partitions(g)} & \\
        \tcode{partition_id(g,u)} & \\
        \tcode{vertices(g,u,pid)} & \\
    \hline
    \end{tabular}}
    \caption{Common CPO Function Overrides}
    \label{tab:cmn_cpo_overrides}
    \end{center}
\end{table}
