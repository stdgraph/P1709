
\clearpage
%% \chapter{Graph Container Implementations}

\section{compressed\_graph Graph Container}
\tcode{compressed_graph} is a graph container being proposed for the standard library. It is a high-performance data structure that 
uses \href{https://en.wikipedia.org/wiki/Sparse_matrix#Compressed_sparse_row_\%28CSR\%2C_CRS_or_Yale_format\%29}{Compressed Sparse Row} 
(CSR) format to store its vertices, edges and associated values. Once constructed, vertices and edges cannot be added or deleted 
but values on vertices and edges can be modified.

There are a number of features added beyond the typical CSR implmentation:
\begin{itemize}
    \item   \textbf{User-defined values} The typical CSR implementation stores values on edges (columns) by defining the \tcode{EV}
            template paraemter. \tcode{compressed_graph} extends that to also allow values on vertices (rows) and the graph itself 
            by defining the \tcode{VV} and \tcode{GV} template arguments respectively. If a type is void, no memory overhead is incurred.
    \item   \textbf{Index type sizes} The size of the integral indexes into the internal vertex (row) and edge (column) structures can be 
            controlled by the \tcode{VId} and \tcode{EIndex} template arguments respectively to give a balance between capacity, memory 
            usage and performance.
    \item   \textbf{Multi-partite graphs} The vertices can optionally be partitioned into multiple partitions by passing the 
            starting vertex id of each partition in the \tcode{partition_start_ids} argument in the constructors. If no partitions are
            specified, the graph is single-partite.
\end{itemize}

The listings in the following sections show the prototypes for the \tcode{compressed_graph} when the graph value type \tcode{GV} is 
non-\tcode{void} (section \ref{compressed_full}) and a class template specialization when it is \tcode{void} (section \ref{compressed_specialization}).

Only the constuctors and destructor shown for \tcode{compressed_graph} are public. All other types and functions related to the graph
are only accessible through the types and functions in the Graph Container Interface.

\begin{table}[h]
    \setcellgapes{3pt}
    \makegapedcells
    \centering
    \begin{tabular}{|P{0.38\textwidth}|P{0.35\textwidth}|P{0.25\textwidth}|}
    \hline
    \textbf{vertex\_id assignment:} Contiguous & \textbf{\tcode{has_edge(g)}} $O(1)$ & \textbf{Append vertices?} No \\
    \textbf{Vertices range:} Contiguous & \textbf{\tcode{num_edges(g)}} $O(1)$ & \textbf{Append edges?} No \\
    \textbf{Edge range:} Contiguous & \textbf{\tcode{partition_id(g,uid)}} $O(log(P+1))$ & \textbf{Partions?} Yes\\
    & & \textbf{is\_directed?} No \\
    %O(log(P+1)) is the complexity of std::upper_bound; +1 is for the number of vertices added at the end of partition_start_ids
    \hline
    \end{tabular}
    %\caption{Jaccard Coefficient Summary}
    \label{tab:compressed_graph_summary}
\end{table}
$P$ is the number of partitions and is exepected to be small, e.g. $P = 2$ for bipartite and $P \leq 10$ for typical 
multi-partite graphs.

The \tcode{is_directed} trait is not supported. If \tcode{compressed_graph} is intended to be used for an undirected
graph, then the edge pairs must be included for both directions, (uid,vid) and (vid,uid), when constructing the graph.

\phil{Add \tcode{operator[](vertex_id_t<G>)}?}

\subsection{compressed\_graph when \tcode{GV} is not \tcode{void}} \label{compressed_full}
{\small
      \lstinputlisting{D3131_Containers/src/compressed_graph.hpp}
}

\subsection{compressed\_graph specialization when \tcode{GV} is \tcode{void}} \label{compressed_specialization}
When \tcode{GV} is void the number of constructors decreases significantly as shown in the following listing.

{\small
      \lstinputlisting{D3131_Containers/src/compressed_graph_gvoid.hpp}
}

\subsection{compressed\_graph description}

\phil{Is it possible to support movable EV and VV types?}

\begin{itemdescr}
    \pnum\mandates
        \begin{itemize}
            \item The \tcode{EV} template argument for an edge value must be a copyable type or \tcode{void.}
            \item The \tcode{VV} template argument for a vertex value must be a copyable type or \tcode{void.}
            \item When the \tcode{GV} template argument for a graph value is not \tcode{void} it can be movable or copyable. It must have a 
                  default constructor if it is not passed in a \tcode{compressed_graph} constructor.
            \item The \tcode{EProj} template argument must be a projection that returns a value of \tcode{copyable_edge<VId, true, EV>} type
                  given a value of \tcode{erng}. If the value type of \tcode{ERng} is already a \tcode{copyable_edge<VId, true, EV>} type, 
                  then \tcode{EProj} can be \tcode{identity}.
            \item The \tcode{VProj} template argument must be a projection that returns a value of \tcode{copyable_vertex<VId, VV>} type,
                  given a value of \tcode{vrng}. If the value type of \tcode{Vrng} is already a \tcode{copyable_vertex<VId, VV>} type, 
                  then \tcode{VProj} can be \tcode{identity}.
        \end{itemize}
    \pnum\preconditions
        \begin{itemize}
            \item The \tcode{VId} template argument must be able to store a value of |V|+1, where |V| is the
                  number of vertices in the graph. The size of this type impacts the size of the \textit{edges}.
            \item The \tcode{EIndex} template argument must be able to store a value of |E|+1, where |E| is the
                  number of edges in the graph. The size of this type impact the size of the \textit{vertices}.
            \item The \tcode{EProj} and \tcode{VProj} template arguments must be valid projections.
            \item The \tcode{partition_start_ids} range includes the starting vertex id for each partition. If it is empty, then the graph is single-partite
                    and the number of partitions is 1. If it is not empty, then the number of partitions is the size of the range, where the first element
                    must be 0 and all elements are in ascending order. A vertex id in the range must not exceed the number of 
                    vertices in the graph. Any violation of these conditions results in undefined behavior.
        \end{itemize}
\phil{If duplicate partition\_start\_ids exist they create an empty partition with no vertices.}

    \pnum\effects
        \begin{itemize}
            \item When \tcode{EV}, \tcode{VV}, or \tcode{GV} are \tcode{void}, no extra memory overhead is incurred for that type.
         \end{itemize}
    %\pnum\result
    %\pnum\returns \lstinline{void} \\
    %\pnum\throws \tcode{out_of_range} is thrown when \tcode{source} is not in the range \tcode{0 <= source < num_vertices(graph)}.  \\
    %\pnum\complexity \\
    \pnum\remarks 
        \begin{itemize}
            \item  The \tcode{VId} and \tcode{EIndex} template arguments impact the capacity, internal storage requirements and performance.
                   The default of \tcode{uint32_t} is sufficient for most graphs and provides a good balance between storage and performance.

                   The memory requirements are roughly, 
                        \[|V| \times (sizeof(EIndex)+sizeof(VV)) + |E| \times (sizeof(VId) + sizeof(EV)) + sizeof(GV)\]
                   where $|V|$ is the number of vertices and $|E|$ is the number of edges in the graph. $sizeof$ \tcode{void} is 0
                   when considering $sizeof$ for \tcode{VV}, \tcode{EV}, and \tcode{GV}. Alignment and overhead for internal vectors are 
                   not included in this calculation.
            \item The allocator passed to constructors is rebound for different types used by different internal containers.
        \end{itemize}
%\pnum\errors
\end{itemdescr}

\section{Using Existing Data Structures}
Reasonable defaults have been defined for the adjacency list and edgelist using data structures and types
in the standard library, with some adaptation for externally defined containers, out of the box that require
no function overrides.

\subsection{Adjacency List Data Structures}

\subsubsection{Using Standard Containers for an Adjacency List}
When the graph is defined using standard containers, the GCI functions can be used without any function overrides.

For example this we'll use \tcode{G = vector<forward_list<tuple<int,double>>>} to define the graph, where \tcode{g} 
is an instance of \tcode{G}. \tcode{tuple<int,double>} defines the target\_id and weight property respectively. We
can write loops to go through the vertices, and edges within each vertex, as follows.

\begin{lstlisting}
using G = vector<forward_list<tuple<int,double>>>;
auto weight = [&g](edge_t& uv) { return get<1>(uv); }

G g;
load_graph(g, ...); // load some data

// Using GCI functions
for(auto&& [uid, u] : vertices(g)) {
  for(auto&& [vid, uv]: edges(g,u)) {
    auto w = weight(uv);
    // do something...
  }
}
\end{lstlisting}

\tcode{edge_t<G>} is defined as \tcode{tuple<int,double>} in the example; the only requirement is that the first element
in the \tcode{tuple} is integral and is used as the target\_id. The \tcode{edge_value} function is not defined, as it is
assumed that algorithms will take a lambda to extract the value from the edge, if needed.

If all you need is the target\_id without any values, you can use \tcode{G = vector<forward_list<int>>}. The \tcode{int}
is used as the target\_id. Again, the only requirement is that it be \tcode{integral}.

Note that no function override was required. More formally, the two patterns recognized are \tcode{random_access_range<forward_range<integral>>}
and \tcode{random_access_range<forward_range<tuple<integral,...>>>}. This extends to any range type. For instance, boost::containers can be used
just as easily as std containers.

Table \ref{tab:simple_graph} shows how the types are defined for the example above.
\begin{table}[h!]
    \begin{center}
    \resizebox{\textwidth}{!}
    {\begin{tabular}{l l p{1.5cm} L{7.0cm}}
    \hline
        \textbf{Function or Value} & \textbf{Concrete Type} \\
    \hline
        \tcode{vertices(g)} & \tcode{vector<forward_list<tuple<int,double>>>} (when \tcode{random_access_range<G>}) \\
        \tcode{u} & \tcode{forward_list<tuple<int,double>>} & \\
        \tcode{edges(g,u)} & \tcode{forward_list<tuple<int,double>>} (when \tcode{random_access_range<vertex_range_t<G>>}) \\
        \tcode{uv} & \tcode{tuple<int,double>} & \\
        \tcode{edge_value(g,uv)} & \tcode{tuple<int,double>} (when \tcode{random_access_range<vertex_range_t<G>>}) \\
        \tcode{target_id(g,uv)} & \tcode{integral}, when \tcode{uv} is either \tcode{integral}  or \tcode{tuple<integral,...>} \\
    \hline
    \end{tabular}}
    \caption{Types When Using Standard Containers}
    \label{tab:simple_graph}
    \end{center}
\end{table}


\subsubsection{Using Other Graph Data Structures}
For other graph data structures function overrides are required. Table \ref{tab:cmn_cpo_overrides} shows the 
common function overrides anticipated for most cases, keeping in mind that all functions can be overridden
if the default implementation is not suitable. 
When they are defined they must be in the same namespace as the data structures.

\begin{table}[h!]
    \begin{center}
    %\resizebox{\textwidth}{!}
    {\begin{tabular}{l l p{1.5cm} L{7.0cm}}
    \hline
        \textbf{Function} & \textbf{Comment} \\
    \hline
        \tcode{vertices(g)} & \\
        \tcode{edges(g,u)} & \\
        \tcode{target_id(g,uv)} & \\
        \tcode{edge_value(g,uv)} & If edges have value(s) in the graph \\
        \tcode{vertex_value(g,u)} & If vertices have value(s) in the graph \\
        \tcode{graph_value(g)} & If the graph has value(s) \\
    \hline
        \multicolumn{2}{c}{When edges have the optional source\_id on an edge} \\
    \hdashline
        \tcode{source_id(g,uv)} & \\
    \hline
        \multicolumn{2}{c}{When the graph supports multiple partitions} \\
    \hdashline
        \tcode{num_partitions(g)} & \\
        \tcode{partition_id(g,u)} & \\
        \tcode{vertices(g,u,pid)} & \\
    \hline
    \end{tabular}}
    \caption{Common CPO Function Overrides}
    \label{tab:cmn_cpo_overrides}
    \end{center}
\end{table}

\subsection{Edgelist Data Structures}
\subsubsection{Using Standard Containers for an Edgelist}
Like the adjacency list, an edgelist can be defined using standard containers and types without 
requiring any function overrides.

\begin{lstlisting}
    using EL = vector<tuple<int, int, double>>;
    using E  = std::ranges::range_value_t<EL>;
    EL el{{1, 2, 11.1}, {1, 4, 22.2}, {2, 3, 3.33}, {2, 4, 4.44}};
    for (auto&& e : el) {
      int    sid = source_id(e);
      int    tid = target_id(e);
      double val = edge_value(e);
    }
\end{lstlisting}

An alternative is to use the \tcode{edge_descriptor} used in this proposal. Notice that the only difference
is the definition of the edgelist type. All other code is identical to the previous example.
\begin{lstlisting}
    using EL = vector<edge_descriptor<int, true, void, double>>;
    using E  = std::ranges::range_value_t<EL>;
    EL el{{1, 2, 11.1}, {1, 4, 22.2}, {2, 3, 3.33}, {2, 4, 4.44}};
    for (auto&& e : el) {
      int    sid = source_id(e);
      int    tid = target_id(e);
      double val = edge_value(e);
    }
\end{lstlisting}

While these examples show the optional edge\_value that is a \tcode{double}, it can be omitted if the edges 
do not have values.

Type alias are in the namespace \tcode{std::graph::edgelist} to avoid conflicts with adjacency\_list
types.

\subsubsection{Using Other Graph Data Structures}
If you have different edge type not covered by the standard types, you can override the
\tcode{source_id(e)}, \tcode{target_id(e)} and \tcode{edge_value(E)} functions for that type.
The functions must be in the same namespace as the edge data structure you want to use.

\phil{Include a section on representing directed vs. undirected graphs? (e.g. using a pair of symmetric edges for undirected)}
