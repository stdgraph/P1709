\section{Revision History}

\subsection*{\paperno r0}

\begin{itemize}
      \item Split from P1709r5. Added \textit{Getting Started} section.
      \item Add default implementation for \tcode{target_id(g,uv)} when the graph type matches the 
            pattern \tcode{random_access_range<forward_range<integral>>} or \tcode{random_access_range<forward_range<tuple<integral,...>>>};
            \tcode{vertex_id_t<G>} also defaults to the \tcode{integral} type given.
      \item Revised concept definitions, adding \tcode{sourced_targeted_edge} and \tcode{target_edge_range}, and replaced
            summary table with code for clarity. Also assured that all combinations of adjacency list concepts for
            \textit{basic}, \textit{sourced} and \textit{index} exist.
      \item Move text for graph data structures created from std containers from Graph Container Interface to 
            Container Implementation paper.
      \item Identify all \tcode{concept} definitions as "For exposition only" until we have consensus of whether they 
            belong in the standard or not.
\end{itemize}

\subsection*{\paperno r1}

\begin{itemize}
      \item Add \tcode{num_edges(g)} and \tcode{has_edge(g)} functions. Split function table into 3 tables for graph,
            vertex and edge functions because it was getting too big.
      \item Removed the Load Graph Data section with it's load functions from \href{https://www.wg21.link/P3130}{P3130 Graph Container Interface}
            because it unnecessarily complicates the interface with constructors for graph data structures. To complement this, constructors have 
            been added for \tcode{compressed_graph} in \href{https://www.wg21.link/P3131}{P3131 Graph Containers}.
      \item Revised partition functions after implementation in \tcode{compressed_graph} to reflect usage, including: 
            renaming \tcode{partition_count(g)} 
            to \tcode{num_partitions(g)} to match other names used, changed \tcode{partition_id(g,u)} to \tcode{partition_id(g,uid)}
            because vertices may not exist when the function is called, and removing \tcode{edges(g,u,pid)} because it can easily be 
            implemented as a filter using ranges functionality when target vertices can be in different partitions.
\end{itemize}

\subsection*{\paperno r2}

\begin{itemize}
      \item Add the edgelist as an abstract data structure as a peer to the adjacency list. 
            This causes a reorganization of this paper and the addition of a new section for the edgelist.
\end{itemize}
