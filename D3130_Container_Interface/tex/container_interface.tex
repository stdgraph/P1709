
\clearpage
%% \chapter{Graph Container Interface}
\section{Graph Container Interface}

The Graph Container Interface (GCI) defines the primitive concepts, traits, types and functions used to define and access an adacency lists (aka graph)
and edgelists, no matter their internal design and organization. For instance, an adjacency list can be a vector of lists from standard containers, 
CSR-based graph and adjacency matrix. Likewise, an edgelist can be a range of edges from a standard container or externally defined edge types, 
provided they have a source\_id, target\_id and optional edge\_value. 

If there is a desire to use the algorithms against externally defined data structures, the GCI exposes is functions as customization points
to be overridden as needed. Likewise, externally defined algorithms can be used to operate on other data structures that meet the GCI requirements.
This achieves the same goals as the STL, where algorithms can be used on any container that meets the requirements of the algorithm.

The GCI is designed to support a wider scope of graph containers than required by the views and algorithms in this 
proposal. This enables for future growth of the graph data model (e.g. incoming edges on a vertex), or as a framework for graph implementations 
outside of the standard. For instance, existing implementations may have requirements that cause them to define features with looser constraints, 
such as sparse vertex\_ids, non-integral vertex\_ids, or storing vertices in associative bi-directional containers (e.g. std::map or std::unordered\_map). 

Such features require specialized implementations for views and algorithms. The performance for such algorithms will be sub-optimal, but may be
preferrable to run them on the existing container rather than loading the graph into a high-performance graph container and then running the 
algorithm on it, where the loading time can far outweigh the time to run the sub-optimal algorithm. To achieve this, care has been taken to 
make sure that the use of concepts chosen is appropriate for algorithm, view and container.

All algorithms in this and related proposals require that adjacency list vertices are stored in random access containers and that \tcode{vertex_id_t<G>} 
is integral. Future designs may relax these requirements, but for now they are required.

\section{Adjacency List Interface}

\subsection{Concepts}

This section describes the concepts to describe the adjacency lists used for graphs in the Graph Library. There are 
a couple of qualifiers that are used in concept names.
\begin{itemize}
    \item \textbf{index} where the vertex range is random-access and the vertex id is integral.
    \item \textbf{sourced} where an edge has a source id.
\end{itemize}

\emph{While we belive the use of concepts is appropriate for graphs as a range-of-ranges, we are marking them as "For exposition only" 
until we have consensus of whether they belong in the standard or not.}

\subsubsection{Edge Concepts}
The types of edges that can occur in a graph are described with the edges concepts.
{\small
     \lstinputlisting{D3130_Container_Interface/src/concepts_edges.hpp}
}

Return types are not validated in order to provide flexibility, and because it offers little value. Let's look at the options using 
\tcode{target_id(g,uv)} as an example.

\begin{lstlisting}
    { target_id(g,uv) } -> integral;
\end{lstlisting}
This may seem obvious on first glance to some, but doing so limits us to integral ids. Graphs can use non-integral vertex id types for 
vertices stored in a \tcode{map} or \tcode{unordered_map}. 
It can be more efficient to simply run an algorithm on the existing graph rather than to copy it into a “high performance” graph data 
structure just to run the algorithm because the copying operation can far outweigh the cost of running the algorithm on the native data 
structures, even when those data structures offer $\mathcal{O}(log(n))$ lookup on vertices. While we're not proposing algorithms that 
can do this today, the library needs to keep the door open to such algorithms in the future as well as supporting such algorithms 
outside the standard library.

\begin{lstlisting}
    { target_id(g,uv) } -> vertex_id_t<G>;
\end{lstlisting}
This is better than the previous example. It doesn’t require an integral vertex id and maintains the integrity of the expected type. 
A problem with this is that if it fails, the error reported will be something like “doesn’t meet concept requirements” which is 
obscure and takes time by the user to understand and resolve.

\begin{lstlisting}
    target_id(g,uv);
\end{lstlisting}
The final option allows the compiler to report a regular error or warning if the returned value isn't what's expected in the 
context it's used because the types are included in the error message, making it easier to understand what the problem is.
Additionally, functions aren't distiguished by their return type, so there's little value in attempting to check it in this case.

There is precedent for this design choice of not validating the return type, as can be seen in the
\href{https://en.cppreference.com/w/cpp/ranges/sized_range}{\tcode{sized_range}}concept.

\paragraph{Edge Range Concepts}

There is a single edge range concept.
{\small
     \lstinputlisting{D3130_Container_Interface/src/concepts_target_edge_range.hpp}
}

\subsubsection{Vertex Concepts}
The \tcode{vertex_range} concept is the general definition used for adjacency lists while \tcode{index_vertex_range} is used for
high performance graphs where vertices typically stored in a \tcode{vector}.
{\small
     \lstinputlisting{D3130_Container_Interface/src/concepts_vertex_range.hpp}
}

\subsubsection{Adjacency List Concepts}
The adjacency list concepts bring together the vertex and edge concepts used for core graph concepts. 
All algorithms initially proposed for the Graph Library use the \tcode{index_adjacency_list}. Future
proposals may use the \tcode{adjacency_list} concept which allows for vertices in associative containers.
{\small
     \lstinputlisting{D3130_Container_Interface/src/concepts_adj_list.hpp}
}

\phil{Verify we're using \tcode{adjacency_list} in the View and algorithm definitions.}

\subsection{Traits}
Table \ref{tab:graph_traits} summarizes the type traits in the Graph Container Interface, allowing views and algorithms to query the graph's characteristics.

\begin{table}[h!]
\begin{center}
%\resizebox{\textwidth}{!}
{\begin{tabular}{l l L{7.0cm}}
\hline
    \textbf{Trait} & \textbf{Type} & \textbf{Comment} \\
\hline
    \tcode{has_degree<G>} & concept & Is the \tcode{degree(g,u)} function available? \\
    \tcode{has_find_vertex<G>} & concept & Are the \tcode{find_vertex(g,_)} functions available? \\
    \tcode{has_find_vertex_edge<G>} & concept & Are the \tcode{find_vertex_edge(g,_)} functions available?\\
    \tcode{has_contains_edge<G>} & concept & Is the \tcode{contains_edge(g,uid,vid)} function available?\\
\hline
    \tcode{define_unordered_edge<G> : false_type} & struct & Specialize to derive from \tcode{true_type} for a graph with unordered edges \\
    \tcode{unordered_edge<G>} & concept & \\
\hline
    \tcode{ordered_edge<G>} & concept & \\
\hline
    \tcode{define_adjacency_matrix<G> : false_type} & struct & Specialize for graph implementation to derive from \tcode{true_type} for edges stored as a square 2-dimensional array \\
    \tcode{is_adjacency_matrix<G>} & struct & \\
    \tcode{is_adjacency_matrix_v<G>} & type alias & \\
    \tcode{adjacency_matrix<G>} & concept & \\
\hline
\end{tabular}}
\caption{Graph Container Interface Type Traits}
\label{tab:graph_traits}
\end{center}
\end{table}


\subsection{Types}
Table \ref{tab:graph_type} summarizes the type aliases in the Graph Container Interface. These are the types used to define the objects 
in a graph container, no matter its internal design and organization. Thus, it is designed to be able to reflect all forms of adjacency 
graphs including a vector of lists, compressed\_graph and adjacency matrix.

The type aliases are defined by either a function specialization for the underlying graph container, or a refinement of one of those types (e.g. an iterator of a range). Table \ref{tab:graph_func} describes the functions in more detail.

\tcode{graph_value(g)}, \tcode{vertex_value(g,u)} and \tcode{edge_value(g,uv)} can be optionally implemented, depending on whether the graph container supports values on the graph, vertex and edge types.

\begin{table}[h!]
\begin{center}
\resizebox{\textwidth}{!}
{\begin{tabular}{l l L{1.5cm}}
\hline
    \textbf{Type Alias} & \textbf{Definition} & \textbf{Comment} \\
\hline
    \tcode{graph_reference_t<G>} & \tcode{add_lvalue_reference<G>} & \\
    \tcode{graph_value_t<G>} & \tcode{decltype(graph_value(g))} & optional \\
\hline
    \tcode{vertex_range_t<G>} & \tcode{decltype(vertices(g))} & \\    
    \tcode{vertex_iterator_t<G>} & \tcode{iterator_t<vertex_range_t<G>>} & \\    
    \tcode{vertex_t<G>} & \tcode{range_value_t<vertex_range_t<G>>} & \\    
    \tcode{vertex_reference_t<G>} & \tcode{range_reference_t<vertex_range_t<G>>} & \\    
    \tcode{vertex_id_t<G>} & \tcode{decltype(vertex_id(g,u))} & \\
    \tcode{vertex_value_t<G>} & \tcode{decltype(vertex_value(g,u))} & optional \\
\hline
    \tcode{vertex_edge_range_t<G>} & \tcode{decltype(edges(g,u))} & \\    
    \tcode{vertex_edge_iterator_t<G>} & \tcode{iterator_t<vertex_edge_range_t<G>>} & \\    
    \tcode{edge_t<G>} & \tcode{range_value_t<vertex_edge_range_t<G>>} & \\    
    \tcode{edge_reference_t<G>} & \tcode{range_reference_t<vertex_edge_range_t<G>>} & \\    
    \tcode{edge_value_t<G>} & \tcode{decltype(edge_value(g,uv))} & optional \\
\hline
    \tcode{partition_id_t<G>} & \tcode{decltype(partition_id(g,u))} & optional \\
    \tcode{partition_vertex_range_t<G>} & \tcode{vertices(g,pid)} & optional \\    
\hline
\end{tabular}}
\caption{Graph Container Interface Type Aliases}
\label{tab:graph_type}
\end{center}
\end{table}

There is no contiguous requirement for \tcode{vertex_id} from one partition to the next, though
in practice they will often be assigned contiguously. Gaps in \tcode{vertex_id}s between partitions should be allowed.

\subsection{Classes and Structs}

The \tcode{graph_error} exception class is available, inherited from \tcode{runtime_error}. While any function may use it, it 
is only anticipated to be used by the \tcode{load} functions at this time.
No additional functionality is added beyond that provided by \tcode{runtime_error}.

\emph{While we belive the use of concepts is appropriate for graphs as a range-of-ranges, we are marking them as "For exposition only" 
until we have consensus of whether they belong in the standard or not.}

\subsection{Functions}

Tables \ref{tab:graph_func}, \ref{tab:vertex_func} and \ref{tab:edge_func} summarize the primitive functions in the Graph Container Interface.
used to access an adacency graph, no matter its internal design and organization. Thus, it is designed 
to be able to reflect all forms of adjacency graphs including a vector of lists, CSR-based graph and adjacency matrix.


\begin{table}[h!]
\begin{center}
\resizebox{\textwidth}{!}
{\begin{tabular}{l l p{2.3cm} L{6.2cm}}
\hline
    \textbf{Function} & \textbf{Return Type} & \textbf{Complexity} & \textbf{Default Implementation} \\
\hline
    \tcode{graph_value(g)} & \tcode{graph_value_t<G>} & constant & n/a, optional \\
    \tcode{vertices(g)} & \tcode{vertex_range_t<G>} & constant & \tcode{g} if \tcode{random_access_range<G>}, n/a otherwise \\
    \tcode{num_vertices(g)} & \tcode{integral} & constant & \tcode{size(vertices(g))} \\
    \tcode{num_edges(g)} & \tcode{integral} & |E| & \tcode{n=0; for(u: vertices(g)) n+=distance(edges(g,u)); return n;} \\
    \tcode{has_edge(g)} & \tcode{bool} & |V| & \tcode{for(u: vertices(g)) if !empty(edges(g,u)) return true; return false;} \\
\hdashline
    \tcode{num_partitions(g)} & \tcode{integral} & constant & 1 \\
    \tcode{vertices(g,pid)} & \tcode{partition_vertex_range_t<G>} & constant & \tcode{vertices(g)} \\
    \tcode{num_vertices(g,pid)} & \tcode{integral} & constant & \tcode{size(vertices(g))}  \\
\hline
\end{tabular}}
\caption{Graph Functions}
\label{tab:graph_func}
\end{center}
\end{table}
The complexity shown above for \tcode{num_edges(g)} and \tcode{has_edge(g)} is for the default implementation. 
Specific graph implementations may have better characteristics.

The only vertex function that requires a vertex id \tcode{(uid)} is \tcode{find_vertex(g,uid)}. The other 
functions that use it are convenience functions that imply a call to \tcode{find_vertex(g,uid)} to get the 
vertex descriptor before a call to the overloaded function that takes a descriptor is made.

The complexity shown above for \tcode{vertices(g,pid)} and \tcode{num_vertices(g,pid)} is for the default
implementation. Specific graph implementations may have different characteristics.

\begin{table}[h!]
\begin{center}
\resizebox{\textwidth}{!}
{\begin{tabular}{l l p{1.5cm} L{7.0cm}}
\hline
    \textbf{Function} & \textbf{Return Type} & \textbf{Complexity} & \textbf{Default Implementation} \\
\hline
    \tcode{find_vertex(g,uid)} & \tcode{vertex_iterator_t<G>} & constant & \tcode{begin(vertices(g)) + uid} \\
    & & & if \tcode{random_access_range<vertex_range_t<G>>}  \\
    \tcode{vertex_id(g,u)} & \tcode{vetex_id_t<G>} & constant & (see Determining the vertex\_id type below) \\
    & & & Override to define a different \tcode{vertex_id_t<G>} type (e.g. int32\_t). \\
    \tcode{vertex_value(g,u)} & \tcode{vertex_value_t<G>} & constant & n/a, optional \\
    \tcode{vertex_value(g,uid)} & \tcode{vertex_value_t<G>} & constant & \tcode{vertex_value(g,*find_vertex(g,uid))}, optional \\
    \tcode{degree(g,u)} & \tcode{integral} & constant & \tcode{size(edges(g,u))} if \tcode{sized_range<vertex_edge_range_t<G>>} \\
    \tcode{degree(g,uid)} & \tcode{integral} & constant & \tcode{size(edges(g,uid))} if \tcode{sized_range<vertex_edge_range_t<G>>} \\
    \tcode{edges(g,u)} & \tcode{vertex_edge_range_t<G>} & constant & \tcode{u} if \tcode{forward range<vertex_t<G>>}, n/a otherwise \\
    \tcode{edges(g,uid)} & \tcode{vertex_edge_range_t<G>} & constant & \tcode{edges(g,*find_vertex(g,uid))} \\
\hdashline
    \tcode{partition_id(g,u)} & \tcode{partition_id_t<G>} & constant &  \\
    \tcode{partition_id(g,uid)} & \tcode{partition_id_t<G>} & constant &  \\
\hline
\hline
\end{tabular}}
\caption{Vertex Functions}
\label{tab:vertex_func}
\end{center}
\end{table}

The default implementation for the \tcode{degree} functions assumes that \tcode{vertex_edge_range_t<G>} is a sized range
to have constant complexity. If the underlying container has a non-linear \tcode{size(R)} function, the \tcode{degree} functions will 
also be non-linear. This is expected to be an uncommon case.

\begin{table}[h!]
\begin{center}
\resizebox{\textwidth}{!}
{\begin{tabular}{l l p{1.5cm} L{7.0cm}}
\hline
    \textbf{Function} & \textbf{Return Type} & \textbf{Complexity} & \textbf{Default Implementation} \\
\hline
    \tcode{target_id(g,uv)} & \tcode{vertex_id_t<G>} & constant & (see below) \\
    \tcode{target(g,uv)} & \tcode{vertex_t<G>} & constant & \tcode{*(begin(vertices(g)) + target_id(g, uv))} if \tcode{random_access_range<vertex_range_t<G>> \&\& integral<target_id(g,uv)>} \\
    \tcode{edge_value(g,uv)} & \tcode{edge_value_t<G>} & constant & \tcode{uv} if \tcode{forward_range<vertex_t<G>>}, n/a otherwise, optional \\
    \tcode{find_vertex_edge(g,u,vid)} & \tcode{vertex_edge_t<G>} & linear & \tcode{find(edges(g,u), [](uv) {target_id(g,uv)==vid;\})}} \\
    \tcode{find_vertex_edge(g,uid,vid)} & \tcode{vertex_edge_t<G>} & linear & \tcode{find_vertex_edge(g,*find_vertex(g,uid),vid)} \\
    \tcode{contains_edge(g,uid,vid)} & \tcode{bool} & constant & \tcode{uid < size(vertices(g)) \&\& vid < size(vertices(g))} if \tcode{is_adjacency_matrix_v<G>}.\\
    & & linear & \tcode{find_vertex_edge(g,uid) != end(edges(g,uid))} otherwise. \\
\hdashline
    \multicolumn{4}{c}{The following are only available when the optional \tcode{source_id(g,uv)} is defined for the edge} \\
\hdashline
    \tcode{source_id(g,uv)} & \tcode{vertex_id_t<G>} & constant & n/a, optional \\
    \tcode{source(g,uv)} & \tcode{vertex_t<G>} & constant & \tcode{*(begin(vertices(g)) + source_id(g,uv))} if \tcode{random_access_range<vertex_range_t<G>> \&\& integral<target_id(g,uv)>} \\
\hline
\end{tabular}}
\caption{Edge Functions}
\label{tab:edge_func}
\end{center}
\end{table}

When the graph matches the pattern \tcode{random_access_range<forward_range<integral>>} or \tcode{random_access_range<forward_range<tuple<integral,...>>>},
the default implementation for \tcode{target_id(g,uv)} will return the \tcode{integral}. Additionally, if the caller does not override 
\tcode{vertex_id(g,u)}, the \tcode{integral} value will define the \tcode{vertex_id_t<G>} type.

Functions that have n/a for their Default Implementation must be defined by the author of a Graph Container implementation. 

Value functions (\tcode{graph_value(g)}, \tcode{vertex_value(g,u)} and \tcode{edge_value(g,uv)}) can be optionally implemented, 
depending on whether the graph container supports values on the graph, vertex and edge types. They return a single value and can 
be scaler, struct, class, union, or tuple. These are abstract types used by the GVF, VVF and EVF function objects to retrieve
values used by algorithms. As such it's valid to return the "enclosing" owning class (graph, vertex or edge), or some other
embedded value in those objects.

\tcode{find_vertex(g,uid)} is constant complexity because all algorithms in this proposal require that \tcode{vertex_range_t<G>} is a random access range. 

If the concept requirements for the default implementation aren't met by the graph container the function will need to be overridden.

\subsection{Determining the vertex\_id and its type}
To determine the type for \tcode{vertex_id_t<G>} the following steps are taken, in order, to determine its type.
\begin{enumerate}
    \item Use the type returned by \tcode{vertex_id(g,u)} when overridden for a graph.
    \item When the graph matches the pattern \tcode{random_access_range<forward_range<integral>>} or \tcode{random_access_range<forward_range<tuple<integral,...>>>},
          use the \tcode{integral} type specified, which is assumed to be the target\_id on an edge.
    \item Use \tcode{size_t} in all other cases.
\end{enumerate}

\tcode{vertex_id_t<G>} is defined by the type returned by \tcode{vertex_id(g)} and it defaults to the difference\_type of the underlying container used for vertices (e.g int64\_t for 64-bit systems). 
This is sufficient for all situations. However, there are often space and performance advantages if a smaller type is used, such as int32\_t or even int16\_t. It is recommended to consider overriding 
this function for optimal results, assuring that it is also large enough for the number of possible vertices and edges in the application. It will also need to be overridden if the implementation doesn't 
expose the vertices as a range.

\tcode{vertex_id(g,u)} is evaluated in the context of a descriptor using the following rules:
\begin{enumerate}
    \item Use the value returned by \tcode{vertex_id(g,u)} when overridden for a graph.
    \item Use the index value on the descriptor.
\end{enumerate}

\subsection{Vertex and Edge Descriptor Views}
The ranges returned by \tcode{vertices(g)} and \tcode{edges(g,u)} are views of their respective underlying 
container in the adjacency list. The value\_type of the view is a descriptor, which refers to an object in 
the underlying range.

Descriptors are opaque, abstract objects that represent vertices and edges in an adjacency list. They are 
particularly useful because they abstract away the implementation details of the adjacency list, allowing 
a user to work with different graph types in a consistent manner. 

A benefit of using descriptors is that it reduces the number of functions and concepts needed for the GCI
compared to previous designs. Without them, an iterface would require additional functions and concepts, 
and algorithms would need to be specialized for vertices stored in random-access containers compared to
associative containers.

Practically, a descriptor is either an integral index or an iterator, depending on the underlying container of vertices
or edges. For example, a descriptor for a vertex in a \tcode{vector} is an index, while the descriptor for an edge in a
\tcode{list} is an iterator. Looking forward to the future, beyond this proposal, a descriptor for a vertex in a 
\tcode{map} or \tcode{unordered\_map} would use an iterator; the door is opened for future expansion with minimal 
impact to the GCI.

The vertex and edge descriptors are defined as the \tcode{vertex_t<G>} and \tcode{edge_t<G>} types, respectively. 
The following are characteristics of descriptors:
\begin{itemize}
    \item Equality comparison: Descriptors can be compared for equality using the \tcode{==} and \tcode{!=} operators.
    \item Ordering: Descriptors can be ordered using the \tcode{<}, \tcode{<=}, \tcode{>}, and \tcode{>=} operators, if supported 
          by the iterators in underlying container being used.
    \item Copy and assignment: Descriptors can be copied and assigned, ensuring they can be used in standard algorithms and containers.
    \item Default construction: Descriptors can be default-constructed, though the resulting value are not guaranteed to represent 
          a valid vertex or edge.
\end{itemize}

In addition, a descriptor has an \tcode{inner_value()} member function that returns a reference to the underlying value
in the underlying container. This is only needed for overriding the customization points when you're adapting your own
graph container to the GCI.
 
The only vertex function that requires a vertex id \tcode{(uid)}is \tcode{find_vertex(g,uid).} All other functions
that accept vertex id are convenience functions that imply a call to \tcode{find_vertex(g,uid)}to get the vertex 
descriptor before a call.

The following are the descriptor views used by \tcode{vertices(g)} and \tcode{edges(g,u)}. \tcode{descriptor_subrange_view} 
is used when edges for multiple vertices are stored in a single container, such as a CSR or adjacency matrix
(e.g. \tcode{compressed\_graph}).
{\small
      \lstinputlisting{D3130_Container_Interface/src/descriptor_view.hpp}
}

\subsection{Unipartite, Bipartite and Multipartite Graph Representation}

\tcode{num_partitions(g)} returns the number of partitions, or partiteness, of the graph. It has a range of 1 to n, where 1 identifies 
a unipartite graph, 2 is a bipartite graph, and a value of 2 or more can be considered a multipartite graph. 

If a graph data structure doesn't support partitions then it is unipartite with one partition and partite functions will reflect that. 
For instance, \tcode{num_partitions(g)} returns a value of 1, and \tcode{vertices(g,0)} (vertices in the first partition) will return a range that includes all vertices in the graph.

A partition identifies a type of a vertex, where the vertex value types are assumed to be uniform in each partition. This creates a dilemma because 
the existing \tcode{vertex_value(g,u)} returns a single type based template parameter for the vertex value type. Supporting 
multiple types can be addressed in different ways using C++ features. The key to remember is that the actual value used by algorithms is done
by calling a function object that retrieves the value to be used. That function is specific to the graph data structure, using the partition to 
determine how to get the appropriate value.
\begin{itemize}
    \item
          \tcode{std::variant}: The lambda returns the appropriate variant value based on the partition.
    \item
          Base class pointer: The lambda can call a member function to return the value based on the partition.
    \item
          \tcode{void*}: The lambda can cast the pointer to a concrete type based on the partition, and
          then return the appropriate value.
\end{itemize}

\tcode{edges(g,uid,pid)} and \tcode{edges(g,u,pid)} filter the edges where the target is in the partition \tcode{pid} passed.
This isn't needed for bipartite graphs.

\section{Edgelist Interface}
An edgelist is a range of values where we can get the source\_id and target\_id, and an optional edge\_value. It is similar to edges in an
adjacency list or edges in the incidence view, but is a distinct range of values that are separate from the others.

Like the adjacency list, the edgelist has default implementations that use the standard library for simple implmentations out of the box.
It's also able to easily adapt to externally defined edge types by overriding the \tcode{source_id(e)}, \tcode{target_id(e)} and 
\tcode{edge_value(e)} functions.

\subsection{Namespace}
The concepts and types for the edgelist are defined in the \tcode{std::graph::edgelist} namespace to avoid conflicts with the adjacency list.

\phil{It would be nice to include them in std::graph, but I'm having difficulting figuring out how to do that based on adjlist and edgelist concepts.}


\subsection{Concepts}
The concepts for edgelists follow the same naming conventions as the adjacency lists. 
% The "basic" qualifier has been dropped WRT integral indexes. There's no reason to have a distinction between the two.

{\small
     \lstinputlisting{D3130_Container_Interface/src/edgelist_concepts.hpp}
}
\subsection{Traits}
Table \ref{tab:edgelist_traits} summarizes the type traits in the Edgelist Interface, allowing views and algorithms to query the graph's characteristics.

\begin{table}[h!]
\begin{center}
%\resizebox{\textwidth}{!}
{\begin{tabular}{l l L{7.0cm}}
\hline
    \textbf{Trait} & \textbf{Type} & \textbf{Comment} \\
\hline
    \tcode{is_directed<EL> : false_type} & struct & When specialized for an edgelist to derive \\
    \tcode{is_directed_v<EL>}          & & from \tcode{true_type}, it may be used during graph 
                                                     construction to add a second edge with source\_id and target\_id reversed. \\
\hline
\end{tabular}}
\caption{Graph Container Interface Type Traits}
\label{tab:edgelist_traits}
\end{center}
\end{table}

%{\small
%     \lstinputlisting{D3130_Container_Interface/src/edgelist_typetraits.hpp}
%}
\subsection{Types}
Table \ref{tab:edgelist_type} summarizes the type aliases in the Edgelist Interface.

The type aliases are defined by either a function specialization for the edgelist implementation, or a refinement of one of 
those types (e.g. an iterator of a range). Table \ref{tab:edgelist_func} describes the functions in more detail.

\tcode{edge_value(g,uv)} can be optionally implemented, depending on whether or not the edgelist has values on the edge types.

\begin{table}[h!]
\begin{center}
%\resizebox{\textwidth}{!}
{\begin{tabular}{l l L{1.5cm}}
\hline
    \textbf{Type Alias} & \textbf{Definition} & \textbf{Comment} \\
\hline
    \tcode{edge_range_t<EL>} & \tcode{EL} & \\    
    \tcode{edge_iterator_t<EL>} & \tcode{iterator_t<edge_range_t<EL>>} & \\    
    \tcode{edge_t<EL>} & \tcode{range_value_t<edge_range_t<EL>>} & \\       
    \tcode{edge_reference_t<EL>} & \tcode{range_reference_t<edge_range_t<EL>>} & \\    
    \tcode{edge_value_t<EL>} & \tcode{decltype(edge_value(e))} & optional \\
\hline
    \tcode{vertex_id_t<EL>} & \tcode{decltype(target_id(e))} & \\
\hline
\end{tabular}}
\caption{Edgelist Interface Type Aliases}
\label{tab:edgelist_type}
\end{center}
\end{table}

%{\small
%     \lstinputlisting{D3130_Container_Interface/src/edgelist_types.hpp}
%}
\subsection{Functions}
Table \ref{tab:edgelist_func} shows the functions available in the Edgelist Interface. Unlike the adjacency list, \tcode{source_id(e)}
is always available.


\begin{table}[h!]
    \begin{center}
    \resizebox{\textwidth}{!}
    {\begin{tabular}{l l p{1.7cm} L{7.0cm}}
    \hline
        \textbf{Function} & \textbf{Return Type} & \textbf{Complexity} & \textbf{Default Implementation} \\
    \hline
        \tcode{target_id(e)} & \tcode{vertex_id_t<EL>} & constant & (see below) \\
        \tcode{source_id(e)} & \tcode{vertex_id_t<EL>} & constant & (see below) \\
        \tcode{edge_value(e)} & \tcode{edge_value_t<EL>} & constant & optional, see below \\
    \hdashline
        \tcode{contains_edge(el,uid,vid)} & \tcode{bool}     & linear & \tcode{find_if(el, [](edge_reference_t<EL> e) \{} \\
                                          &                  &        & \hspace{5mm} \tcode{return source_id(e)==uid} \\
        \tcode{num_edges(el)}             & \tcode{integral} & constant & \tcode{size(el)} \\
        \tcode{has_edge(el)}              & \tcode{bool}     & constant & \tcode{num_edges(el)>0} \\
    \hline
    \end{tabular}}
    \caption{Edgelist Interface Functions}
    \label{tab:edgelist_func}
    \end{center}
\end{table}

% \tcode{find_if(el, )}
% [](edge_reference_t<EL>) \{return source_id(e)==uid && target_id(e)==vid\}

\subsection{Determining the source\_id, target\_id and edge\_value types}
Special patterns are recognized for edges based on the \tcode{tuple} and \tcode{edge_info} types. When they are used the \tcode{source_id(e)}, 
\tcode{target_id(e)} and \tcode{edge_value} functions will be defined automatically.

The \tcode{tuple} patterns are
\begin{itemize}
    \item \tcode{tuple<integral,integral>} for \tcode{source_id(e)} and \tcode{target_id(e)} respectively.
    \item \tcode{tuple<integral,integral,scalar>} for \tcode{source_id(e)}, \tcode{target_id(e)} and \tcode{edge_value(e)} respectively.
\end{itemize}

The \tcode{edge_info} patterns are
\begin{itemize}
    \item \tcode{edge_info<integral,true,void,void>} with \tcode{source_id(e)} and \tcode{target_id(e)}.
    \item \tcode{edge_info<integral,true,void,scaler>} with \tcode{source_id(e)}, \tcode{target_id(e)} and \tcode{edge_value(e)}.
\end{itemize}

In all other cases the functions will need to be overridden for the edge type.

\section{Using Existing Data Structures}
Reasonable defaults have been defined for the adjacency list and edgelist functions to minimize the amount of work
needed to adapt existing data structures to be used by the views and algorithms.

Useful defaults have been created using types and containers in the standard library, with the ability
to override them for external data structures. This is described in more detail in the paper for Graph Library 
Containers.
