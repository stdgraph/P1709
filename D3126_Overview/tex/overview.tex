
%% \chapter{Overview}
\section{Overview}

Graphs, used in ML and other \textbf{scientific} domains, as well as \textbf{industrial} and \textbf{general} programming, 
do \textbf{not} presently exist in the C++ standard. In ML, a graph forms the underlying structure of an \textbf{artificial neural network} (ANN). 
In a \textbf{game}, a graph can be used to represent the \textbf{map} of a game world. In \textbf{business} environments, graphs arise as 
\textbf{entity relationship diagrams} (ERD) or \textbf{data flow diagrams} (DFD). In the realm of \textbf{social media}, a graph represents a 
\textbf{social network}.

All documents, taken as a whole for a Graph Library, proposes the addition of \textbf{graph algorithms, operators, views, adaptors}, the 
\textbf{graph container interface}, and a \textbf{graph container implementation} to the C++ library to support \textbf{machine learning} (ML), 
as well as other applications. ML is a large and growing field, both in the \textbf{research community} and \textbf{industry}, that has 
received a great deal of attention in recent years. This documents presents an \textbf{interface} of the proposed algorithms, operators, adaptors,
views, graph functions, and containers.

\section{Goals and Priorities}

Because graphs and their algorithms cover a broad range of capabilities and implementations, we have defined a focused set of 
goals and priorities that will provide an initial set of useful functionality, as well as a sound foundation for future work.

\begin{itemize}
      \item Provide a firm theoretical foundation for the library.
      \item Follow the separation of algorithms, ranges, views, and containers established by the standard library.
      \item Include a rich enough set of algorithms for the library to be useful.
            \begin{itemize}
                  \item The syntax for an algorithm's implementation should be simple, expressive, and easy to understand. 
                  \item The ability to write high-performance algorithms should not be compromised.
                  \item Algorithms can expect vertices to be in a random access range with an integral vertex id initially.
            \end{itemize}
      \item Include views for common traversals of a graph's vertices and edges that is concise and consistant without having to use a lower level interface.
            \begin{itemize}
                  \item Simple views for vertexlist, incidence edges on a vertex, neighbors of a vertex, and edges of a graph.
                  \item Complex views for depth-first search, breath-first search, and topological sort.
            \end{itemize}
      \item A Graph Container Interface, used by Views and Algorithms, that provides a consistent interface for different graph data structures. The interface
            includes concepts, types, traits and functions and provides a similar role to the Ranges library for standard containers.
            \begin{itemize}
                  \item Descriptors for a consistent data model for vertex, edge and neighbor by views and edge lists.

                  \item Adjacency list, an outer range of vertices with an inner range of outgoing edges on each vertex.
                        \begin{itemize}
                              \item Be able to use the algorithms and views with existing graph data structures using customization points.
                              \item Support for optional user-defined value types on an edge, vertex, and/or the graph itself.
                              \item Support bipartite and multipartite graphs.
                        \end{itemize}

                  \item Edge list, which is a range of edge descriptors.
                        \begin{itemize}
                              \item From an \tcode{edgelist} view.
                              \item From a user-defined range that uses \tcode{std::ranges::transform_view}.
                        \end{itemize}
            \end{itemize}
      \item Provide one or more graph containers that can be used with the algorithms.
            \begin{itemize}
                  \item A high-performance \tcode{compressed_graph} container, based on the Compressed Sparse Row matrix.
                  \item The ability to create simple graph container from standard containers, e.g. \tcode{vector<vector<int>>}.
            \end{itemize}
\end{itemize}

The design should not hinder the ability to extend the functionality to support expanded functionality identified in the future roadmap
that follows.

\subsection{Future Roadmap}

The following are areas we'd like to see in future proposals, after the initial proposals are accepted. We endeavor to investigate
these to assure the existing design will support them.
\begin{itemize}
      \item Additional graph algorithms. The Graph Algorithms paper identifies tiers of algorithms we'd like to see added in the future, 
            including parallel algorithms.
      \item Support for sparse vertex ids, implying the use of bi-directional containers such as \tcode{map} and \tcode{unordered_map} for vertices.
      \item Bi-directional graphs, where vertices have incoming and outgoing edges.
      \item Non-integral vertex ids.
      \item Constexpr graphs, where vertices and edges are stored in \tcode{std::array} or other constexpr-friendly container.
\end{itemize}


\section{Examples}

%\andrew{Where do examples really belong?  In P2300 they are up front here, but I think there is too much forward referencing for that.}

The following code demonstrates how a simple graph can be created as a range of ranges, using the standard containers. 

\phil{Duplicated in Introduction. OK?}

{\small
  \lstinputlisting[firstline=26,lastline=48]{D3126_Overview/src/bacon.cpp}
}

\tcode{target_id(g,uv)} defines the required function to get a target\_id for an edge in the graph \tcode{G}. Other functions can also
be overridden to allow a developer to adapt their own graph data structures to the library.

\section{What this proposal is  \textbf{not}}

The Graph Library proposal limits itself to adjacency graphs and edgelists only. An adjacency graph is an outer range of vertices with an inner range of outgoing
edges on each vertex. An edgelist is a view of edges, which is either all the edges in the adjacency graph or a projection of a user-defined range.

Parallel versions of the algorithms are not included for several reasons. The executors proposal in P2300r5 \cite{REF_P2300r5} is expected to introduce new 
and better ways to do parallel algorithms beyond that used in the parallel STL algorithms and we would like to wait for finalization of that proposal before 
committing to parallel implementations. Secondly, many graph algorithms don't benefit from parallel implementations so there is less need to offer an implementation. 
Lastly, it will help limit the size of this proposal which is already looking to be large without it. It is expected that future proposals will be submitted for parallel graph algorithms. 

Hypergraphs are not supported.

\section{Impact on the Standard}
This proposal is a pure \textbf{library} extension.

\section{Interaction wtih Other Papers}
Other than the papers identified as part of the Graph Libary, there is no interaction with other proposals to the standard.

\section{Implementation Experience}
The github \href{https://github.com/stdgraph}{github.com/stdgraph} repository contains an implementation for this proposal.

\section{Usage Experience}
There is no current use of the library. There are plans to begin using it in 2024 in a commercial setting.

\section{Deployment Experience}
There is no current deployment experience of the library. There are plans for this to follow the usage experience.

\section{Performance Considerations}
The algorithms are being ported from NWGraph to the \href{https://github.com/stdgraph}{github.com/stdgraph} implementation used for this proposal. 
Performance analysis from those algorithms can be found in the peer-reviewed papers for NWGraph~\cite{REF_nwgraph_paper,gapbs_2023}.

\section{Prior Art}
\textbf{boost::graph} has been an important C++ graph implementation since 2001. It was developed with the goal of providing
a modern (at the time) generic library that addressed all the needs of a graph library user. It is still a viable library used today, attesting to the value it brings.

However, boost::graph was written using C++98 in an ``expert-friendly'' style, adding many abstractions and using sophisticated tempate metaprogramming, making it difficult to use by a casual developer.

(Andrew is a co-author of boost::graph.)

\medskip

\textbf{NWGraph} (\cite{REF_nwgraph_library} and \cite{REF_nwgraph_paper}) was published in 2022
by Lumsdaine et al, bringing additional experience gained since creating boost::graph, to create a modern graph library using C++20 for its implementation 
that was more accessible to the average developer. % with the latest algorithms.

While NWGraph made important strides to introduce the idea of the graph as a range-of-ranges and implemented many important algorithms,
there are some areas it didn't address that come a practical use in the field. For instance, it didn't have a well-defined API for graph
data structures that could be applied to existing graphs, and there wasn't a uniform approach to properties.

This proposal takes the best of NWGraph, with previous work done for P1709 to define a Graph Container Interface, to provide a library that
embraces performance, ease-of-use, and the ability to use the algorithms and views on externally defined graph containers.

\medskip

\textbf{GraphBLAS} TBD
%The following is generated by CoPilot and may, or may not, be relevant::
%is a standard for graph algorithms that is used in many graph libraries. It is a set of linear algebra operations that can be used to
%implement graph algorithms. It is not a graph library itself, but it is used in many graph libraries to implement graph algorithms.

\section{Alternatives}
We're unaware of any other library that meets the same requirements and uses concepts and ranges from C++20.

%Later revisions should include this:
%\section{Changes Library Evolution previously requested}

\section{Feature Test Macro}
The \tcode{__cpp_lib_graph} feature test macro is recommended to represent all features in this proposal including algorithms, views, concepts, traits, types, functions, and graph container(s).

\section{Freestanding}
We believe this library can be used in a freestanding C++ implementation.

\section{Namespaces}
Graph containers and their views and algorithms are not interchangeable with existing containers and algorithms.
Additionally, there are some domain-specific terms that may clash with existing or future names, such as 
\tcode{degree} and \tcode{partition_id}.
For these reasons, we recommend their own namespaces. The following assumption is used in this proposal.
\begin{itemize}
\item[]\tcode{std::graph} and \tcode{std::graph::views}
\end{itemize}

\noindent
Alternative locations include the following:
\begin{itemize}
\item[]\tcode{std::ranges} and \tcode{std::ranges::views}
\item[]\tcode{std::ranges/graph} and \tcode{std::ranges::graph::views}
\end{itemize}
The advantage of these two options are that there would be no requirement to use the ranges:: prefix for things
in the std::ranges namespace, a common occurance.

\section{Notes and Considerations}
There are some interesting observations that can be made about graphs and how they compare and contrast to the 
standard library that may not be obvious.
\begin{itemize}
      \item The adjacency list, the primary data structure for this proposal, is a compound data structure of a
            range of ranges. This introduces a new form of container beyond a simple range.
      \item There is more than one possible value type, one each for edge, vertex, and graph. Each is optional.
            This is in contrast to existing practice where the value type is the distinguishing difference between
            different containers, such as for \tcode{set} and \tcode{map}.
      \item Algorithms will often use views, though they can use the GCI functions when needed.
      \item Algorithms and Views often need to allocate memory internally to achieve their purpose. This is a departure from
            common practice in the standard.
\end{itemize}

There are other observations we've also discovered along the way that may not be obvious.
\begin{itemize}
      \item Storing vertices in a \tcode{map} (bi-directional range) requires a different style of programming 
            algorithms, compared to being kept in a \tcode{vector} (random access range). When using a \tcode{vector},
            \tcode{edges(g,uid)} would normally be used without much thought. Using that with a \tcode{map} would
            incur a $\mathcal{O}(\log(V))$ cost. Instead, it will use vertex id once to get the vertex reference 
            and then use \tcode{edges(g,uv)}. This is expected to result in overloading of existing algorithms based on the
            range type of a container, distinguished with concepts.
\end{itemize}

The addition of concepts to the standard library is a serious consideration because, once added, they cannot 
be removed. We believe that graphs as a range-of-ranges merits the addition new concepts but we recognize that it
may be a controversial decision. Toward that end, we will continue to include them to help clarify the examples given
and are assumed to be "For exposition only" as suggested implementation until clear decisions to include them, or not, 
is made.

\section{Issues Status}
This sections lists the known and open issues for the Graph Library proposal across all papers. They are organized
by the paper they are associated with.

\subsection{Open Design Issues}
\begin{itemize}
      \item \href{https://www.wg21.link/P3130}{P3130 Graph Container Interface}
            \begin{enumerate}
                  \item Complete the definition of the edgelist concepts, types and CPO functions. This is distinct from the existing edgelist view.
            \end{enumerate}
      \item \textbf{Pxxxx: Graph Operators} (paper not yet submitted)
            \begin{enumerate}
                  \item Complete the paper for additional utility functions including degree, sort, relabel, transpose and join.
            \end{enumerate}
\end{itemize}

\subsection{Open Reported Issues}
\begin{itemize}
      \item General
            \begin{enumerate}
                  \item Build on \tcode{mdspan} and try to standardize (or at least understand) what might reasonably be called an unstructured span
                        \begin{itemize}
                              \item The statement assumes vertices are in a random-access range and prevents the use of bi-directional ranges like std::map, which 
                                    could be used for sparse vertex ids. The existing design should be able to adept easily to \tcode{mdspan,} so there's no reason to pursue this.
                              \item I don't think I expressed myself very well here. I completely take your point about not assuming that vertices are in a random-access range. 
                                    But what I'm trying to get at is as follows. Suppose someone standardizes unstructured span, as a natural extension of mdspan. What could we 
                                    learn from its api that may be relevant for graphs? In both cases, we will presumably have a method which allows iteration over the ith partition 
                                    (or edges of a given node, for graphs). Consistency of the stl may mean we want these to have the same look/feel.
                        \end{itemize}
            \end{enumerate}
      \item \href{https://www.wg21.link/P3126}{P3126 Overview}
            \begin{enumerate}
                  \item GraphBLAS is not included as part of the prior art.
                        \begin{itemize}
                              \item We will add a reference to GraphBLAS.
                        \end{itemize}
            \end{enumerate}
      \item \href{https://www.wg21.link/P3127}{P3127 Background and Terminology}
            \begin{enumerate}
                  \item P1709 has lots of details which I think to be irrelevant. (P1709 is the original proposal that was split into multiple papers)
                        \begin{itemize}
                              \item Clarification: I don't find the discussion about adjacency matrices helpful, but rather a distraction. 
                                    It's not that it shouldn't be there in some form, but at the moment it has a prominence which I don't think is 
                                    commensurate with its importance to the paper, perhaps exacerbated by the fact that the paper lacks many salient details (see next point).
                        \end{itemize}
                  \item It is very hard to follow
                        \begin{itemize}
                              \item Clarification: As it stands, the paper lacks a discussion of the authors' standpoint on graph terminology, defining features 
                                    (e.g. self loops, multi-edges) and the sort of trade-offs you get by allowing/not allowing them. Put another way, I think the 
                                    paper would be easier to follow if there's a technical narrative that reveals the way the authors are thinking about this huge area. 

                                    I like the style of the motivation in R5; if this could be greatly extended to include the mathematical background that Andrew is 
                                    working on, this would be really helpful. And beyond the mathematical background, as discussion of the computational tradeoffs for 
                                    both graph implementations and the associated algorithms, given certain choice, would be great to have.
                              \item We plan on extending the paper to include a more rigorous mathematical description.
                        \end{itemize}
                  \item We need to add a mathematical perspective to the paper.
                        \begin{itemize}
                              \item P3127 includes some of this. We plan on extending it to include a more rigorous mathematical description.
                        \end{itemize}
                  \item There needs to be a proper discussion about whether the paper's definition of graph is what some authors call a multigraph 
                        and whether it does/doesn't include loops.
                        \begin{itemize}
                              \item The current version of P3128 Algorithms has a summary table for each algorithm that includes Complexity, Directed?, Multi-edge?, 
                                    Cycles?, Self-loops?, and Throws?. We still need to make a pass through the algorithms to assure the values are correct.
                              \item The summary tables for the algorithms are necessary but not sufficient: 
                                    \begin{itemize}
                                          \item There needs to be a discussion of these aspects for graph implementations themselves. Various graph operations may 
                                                be more efficient if the graph structure is more constrained. However, not allowing e.g. multiple edges between pairs 
                                                of nodes prohibits representing many useful systems. There are trade-offs and these need to be discussed.
                                          \item A justification of the choices made for the algorithms may be helpful.
                              \end{itemize}
                              
                        \end{itemize}
                  \item The electrical circuit example has issues in P3127, section 6.1.
                        \begin{itemize}
                              \item We acknowledge this and will remove it, or replace it with a better example.
                              \item I think it's very valuable to include electrical circuits in addition to a simpler example. As we've discussed, electrical 
                                    circuits are surprisingly subtle to represent using graphs, but I think users of a graph library should rightly expect that 
                                    it can be elegantly done. I think signs of a good design for std::graph is that people can do this. So I think electrical 
                                    circuits should stay in, in all their glory, but complemented by something less subtle. 
                        \end{itemize}
      \end{enumerate}
      \item \href{https://www.wg21.link/P3128}{P3128 Graph Algorithms}
            \begin{enumerate}
                  \item A concern is that the DFS and BFS functionality isn't flexible enough, especially when compared to boost::graph's visitors.
                        \begin{itemize}
                              \item We agree having a more general and flexible BFS and DFS would be valuable. We are investigating the merits 
                                    of implementations based on coroutines and boost::graph-like visitors.
                              \item The purpose of the standard library is to adopt standard practice. If we propose the use of coroutines, it is different 
                                    from the use of visitors by boost::graph (written around 2000) which is the closest to a defacto standard available. 
                                    It will be helpful to justify using something different ahead of time to smooth the process.
                        \end{itemize}
            \end{enumerate}
      \item \textbf{Pyyyy: Comparison to boost::graph} (paper not yet submitted)
            \begin{enumerate}
                  \item My comment about the structure of the paper changing was a reference to previous comparisons with boost::graph. 
                        I'm sure these were in an earlier version, or am I misremembering?
                  \begin{itemize}
                        \item We never had any comparisons to boost::graph. 
                        \item We are planning on adding a new paper to compare it to graph-v2 in regards to syntax and performance.
                  \end{itemize}
            \end{enumerate}
\end{itemize}

\subsection{Resolved Issues}
\begin{itemize}
      \item \href{https://www.wg21.link/P3130}{P3130 Graph Container Interface}
            \begin{enumerate}
                  \item I'm not convinced by the load API.
                        \begin{itemize}
                              \item We agree because the use of both load functions and constructors creates ambiguity and complexity when both are defined.
                                    Even though constructors weren't in the paper it wasn't clear wheter they should be included or not.
                                    We have removed the load functions and added constructors for \tcode{compressed_graph} to simplify the interface.
                                    An optional \tcode{partition_start_id} range has also been added to the constructors to support partitions.
                        \end{itemize}
            \end{enumerate}
\end{itemize}
