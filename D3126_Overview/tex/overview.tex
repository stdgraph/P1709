
%% \chapter{Overview}
\section{Overview}

The C++ Standard Library (nee the Standard Template Library, or STL), is a well-structured collection of generic
software components, systematically organized based on fundamental properties of their underlying problem
domain.  In the case of the Standard Library, this domain comprises that of one-dimensional containers,
specified by iterator pairs (equiv.\ ranges).  Of course, the contents of any one-dimensional container can
be another container, to the Standard Library, those contents are opaque.

Since the introduction of the STL and the generic programming paradigm that is its
intellectual foundation, there has been recognition of the need to extend the
standard library to support hierarchical containers (containers of containers).
(See, for example, Matt Austern's paper ``Segmented Iterators and Hierarchical Algorithms,
or WG21/N3700
``Hierarchical Data Structures and Related Concepts for the C++ Standard Library'').
C++ does have some notion of multidimensional data structures (multi-dimensional ranges,
\lstinline{std::mdspan}), but they are not
containers in the sense of the standard library.  Moreover, multi-dimensional arrays
are not ``interesting'' hierarchical containers, in that, even if viewed as a
container of containers, the container one level down from a given container is
still a multi-dimensional array.

Graphs are an important abstraction in Computer Science and that arise in numerous problem domains.  They have
attracted significant attention recently with the increasing importance of their use in data science, machine
learning, and artificial intelligence.
%
Moreover, graphs represent perhaps the most basic form of an interesting (non-trivial) hierarchical data
structure.  For example, although the ``adjacency list'' representation of a graph only comprises two
levels, it is richer than a two-dimensional container.  Most significantly, the ``inner container''
of the adjacency list cannot be opaque; traversal of a graph is of necessity an interleaved traversal
of the inner and outer containers.


\section{Goals and Priorities}

The main goal of this library is to provide a self-consistent and systematic library of software components for
graph computations, based on well-defined graph representations. it includes generic 
algorithms, named requirements, views, and utilities.
views.  
%
It is informed by the authors' past experience with the Boost Graph Library (BGL), the NWGraph library, the C++
GraphBLAS, and the library proposed in P1709.  The proposed library seeks to leverage the existing Standard
Library while introducing as little new machinery as possible.
%
Because of the broad scope of graph data structures and algorithms, we have defined a focused set of goals and
priorities to deliver an initial set of useful and practical functionality that will also establish a solid
foundation for future development.


\textbf{Design Principles.}  We propose a lightweight library of software components for graph computing that is
self-consistent and systematically defined.  The foundation of the library is its collection of generic traversal
patterns and algorithms.  Although concepts will be used ``under the hood,'' the API for the library's
algorithms and traversals will be specified using named requirements.  As little new machinery as possible is
introduced; to the greatest extent possible, graph library components are built on top of existing Standard
Library components.

\textbf{Named Requirements.}  The named requirements in the library reflect the needs of graph algorithms and
are organized and named according to well-accepted practice in the underlying problem domain of graphs and graph
algorithms.  The named requirements are intended to be minimal in order to enable the largest variety of
concrete data structures to satisfy them.

\textbf{Graph Representation.}  To be precise, and in keeping with common practice, the named requirements, the
underlying concepts, and the algorithms in the library are defined in terms of representations of graphs.  
What we mean by a graph representations is developed in detail in the companion document D3127.

\textbf{Leverage the Standard Library.}
With the basis of the library being the representation of graphs in the form of hierarchical containers, and
with the intention for the library to be included as part of the C++ standard library, we leverage standard
library components to the greatest extent possible.  Notably, compositions of Standard Library containers
satisfy the graph representation requirements, sufficient for their use with the proposed graph library.

\textbf{Graph Library Components.}  The library will comprise algorithms and views, along with domain-specific
named requirements.\andrew{This is still in flux.  I imagine a diagram and a table of some kind.}

% https://www.boost.org/doc/libs/1_81_0/libs/graph/doc/graph_concepts.html


\subsection{Future Roadmap}

\noindent
\andrew{Aren't we doing most of these?}
\noindent

The following areas are opportunities for future proposals, after the initial proposals are accepted. We endeavor to investigate
them (without introducing additional proposals) to ensure the currently proposed design will support them.
\begin{itemize}
      \item Additional graph algorithms. The Graph Algorithms proposals (D3128) identifies tiers of algorithms that we suggest be added in a staged fashion (including parallel algorithms).
      \item Support for sparse vertex ids, implying the use of bi-directional containers such as \tcode{map} and \tcode{unordered_map} for vertices.
      \item Bi-directional graphs, where vertices have incoming and outgoing edges.
      \item Constexpr graphs, where vertices and edges are stored in \tcode{std::array} or other constexpr-friendly container.
      \item Parallel graph algorithms.
\end{itemize}


% \section{Examples}
%% %\andrew{Where do examples really belong?  In P2300 they are up front here, but I think there is too much forward referencing for that.}
%% The following code demonstrates how a simple graph can be created as a range of ranges, using the standard containers. 
%% % \phil{Duplicated in Introduction. OK?}
%% {\small
%%   \lstinputlisting[firstline=26,lastline=48]{D3126_Overview/src/bacon.cpp}
%% }
%% \tcode{target_id(g,uv)} defines the required function to get a target\_id for an edge in the graph \tcode{G}. Other functions can also
%% be overridden to allow a developer to adapt their own graph data structures to the library.

\section{Example: Six Degrees of Kevin Bacon}
\label{sec:bacon}

A classic example of the use of a graph algorithm is the game ``The Six Degrees of Kevin Bacon.''
The game is played by connecting actors to each other through movies they have appeared in together.
The goal is to find the smallest number of movies that connect a given actor to Kevin Bacon.
That number is called the ``Bacon number'' of the actor. Kevin Bacon himself has a Bacon number of 0.
Since Kevin Bacon appeared with Tom Cruise in ``A Few Good Men'', Tom Cruise has a Bacon number of 1.

The following program computes the Bacon number for a small selection of actors.

% \phil{Duplicated in Overview's Examples. OK?}
{\small
  \lstinputlisting[firstline=26,lastline=48]{D3126_Overview/src/bacon.cpp}
}


\noindent
Output:
\begin{lstlisting}
Tom Cruise has Bacon number 1
Kevin Bacon has Bacon number 0
Hugo Weaving has Bacon number 3
Carrie-Anne Moss has Bacon number 4
Natalie Portman has Bacon number 2
Jack Nicholson has Bacon number 1
Kelly McGillis has Bacon number 2
Harrison Ford has Bacon number 1
Sebastian Stan has Bacon number 3
Mila Kunis has Bacon number 3
Michelle Pfeiffer has Bacon number 1
Keanu Reeves has Bacon number 4
Julia Roberts has Bacon number 1    
\end{lstlisting}  


In graph parlance, we are creating a graph where the vertices are actors and the edges are movies.
The number of movies that connect an actor to Kevin Bacon is the shortest path in the graph
from Kevin Bacon to that actor. In the example above, we compute shortest paths from Kevin
Bacon to all other actors and print the results.
Note, however, that actor-actor relationships are not how data about actors
is available in the wild (from IMDB, for example).  Rather, two available types of data are actor-movie and movie-actor relationships.  See Section~\ref{sec:bipartite} below.

\section{What this proposal is  \textbf{not}}

The Graph Library proposal limits itself to adjacency graphs and edgelists only. An adjacency graph is an outer range of vertices with an inner range of outgoing
edges on each vertex. An edgelist is a view of edges on an adjacency list, or a range of edge types.

Parallel graph algorithms are not included in this proposal for several reasons.
\begin{itemize}
      \item Parallelism is not beneficial for some algorithms, such as for depth-first search.
      \item There is no clear industry standard for a parallel version of some algorithms.
      \item The parallel algorithm is a different algorithm altogether, such as \textit{Delta-Stepping} for shortest paths. 
            Omitting them helps to limit the size of this proposal that is already large.
      \item A richer set of parallelization mechanisms is required because of the irregular and hierarchical nature of graph 
            data structures. Deferring this to a future proposal constrains the complexity and size of this initial proposal.
\end{itemize}
We feel that providing a broader set of algorithms to address different interests is the better choice. 
We anticipate that proposals will be submitted for parallel graph algorithms in the future. 

Hypergraphs are not supported.

\section{Impact on the Standard}
This proposal is a pure \textbf{library} extension.

\section{Interaction wtih Other Papers}
The entirety of our proposal for graph algorithms and data structures comprises multiple companion papers: D3127 (Terminology), D3128 (Algorithms), D3129 (Views), D3130 (Container Interface), D3131 (Containers), D9903 (Operators), and D9907 (Adaptors).
Other than these papers, there are no interactions with other proposals to the standard.

\section{Implementation Experience}
The github \href{https://github.com/stdgraph}{github.com/stdgraph} repository contains a reference implementation for this proposal.

\section{Usage Experience}
There is no current use of the library outside of the proposers. There are plans to begin using it in 2025 in commercial, academic, and research settings.

\section{Deployment Experience}
There is no current deployment experience of the library. Deployment experience will be gathered in conjunction with use.

\section{Performance Considerations}
The algorithms are being ported from NWGraph to the \href{https://github.com/stdgraph}{github.com/stdgraph} implementation used for this proposal. 
Performance analysis from those algorithms can be found in the peer-reviewed papers for NWGraph~\cite{REF_nwgraph_paper,gapbs_2023}.

\section{Prior Art}

\textbf{boost::graph} has been an important C++ graph implementation since 2001. It was developed with the goal of providing
a modern (at the time) generic library that addressed all the needs of a graph library user. It is still a viable library used today, attesting to the value it brings.

However, boost::graph was written using C++98 in an ``expert-friendly'' style, adding many abstractions and using sophisticated tempate metaprogramming, making it difficult to use by a casual developer.  Particular pain-points described in ad-hoc discussions with users  include: property maps, parameter-passing, visitors.

\medskip

\textbf{NWGraph} (\cite{REF_nwgraph_library} and \cite{REF_nwgraph_paper}) was published in 2022
by Lumsdaine et al, bringing additional experience gained since creating boost::graph, to create a modern graph library using C++20 for its implementation 
that was more accessible to the average developer. % with the latest algorithms.

While NWGraph made important strides to introduce the idea of an adjacency list as a range-of-ranges and implemented many important algorithms,
there are some areas it didn't address that come a practical use in the field. For instance, it didn't have a well-defined API for graph
data structures that could be applied to existing graphs, and there wasn't a uniform approach to properties.

This proposal takes the best of NWGraph, with previous work done for P1709 to define a Graph Container Interface, to provide a library that
embraces performance, ease-of-use, and the ability to use the algorithms and views on externally defined graph containers.

\medskip

\textbf{GraphBLAS}
%The following is generated by CoPilot and may, or may not, be relevant::
%is a standard for graph algorithms that is used in many graph libraries. It is a set of linear algebra operations that can be used to
%implement graph algorithms. It is not a graph library itself, but it is used in many graph libraries to implement graph algorithms.
Graph algorithms are traditionally developed, and then implemented, using explicit loops
over a graph data structure---sometimes referred to as ``pointer chasing.''
An alternative formulation of graph algorithms leverages the close inherent relationship between
graphs and sparse matrices to formulate graph algorithms as sequences of higher-level operations:
sparse matrix multiplication (and other similar operations) over a semiring~\cite{kepner-gilbert}.

The GraphBLAS is an ad-hoc
standardization effort to develop a set of kernel operations for supporting classical
graph algorithms.  As an API specification, the GraphBLAS is not a a graph library per se, 
but rather is intended to be
used to implement graph algorithms (much as the linear algebra BLAS are used  to implement
linear algebra libraries such as LAPACK).

A C language binding that specifically implements the API is available as part of SuiteSparse.
However, the resulting library relies on its own (opaque) data structures for representing graphs
and would not be inter-operable with modern C++ approaches to library and application design.
There have been early attempts at native C++ realizations of GraphBLAS, e.g., the
GraphBLAS Template Library (GBTL).
% Scott -- is there a reference to your and Ben's current efforts?

(NB: Andrew is a co-author of boost::graph; 
Scott and Andrew were participants in GraphBLAS standardization
and co-authors of GBTL; Andrew, Scott, and Phil are 
co-authors of NWGraph.)

\section{Alternatives}
Although the prior efforts have served, and do serve, important roles,
they do not meet the needs or expectations of modern C++ development.
We are currently
unaware of any existing graph
library that meets the same requirements and uses concepts and ranges from C++20.

%Later revisions should include this:
%\section{Changes Library Evolution previously requested}

\section{Feature Test Macro}
The \tcode{__cpp_lib_graph} feature test macro is recommended to represent all features in this proposal including algorithms, views, concepts, traits, types, functions, and graph container(s).

\section{Freestanding}
We are unable to support freestanding implementations in this proposal because many of the algorithms and views 
require a \tcode{stack} or \tcode{queue}, which are not available in a freestanding environment. 
Additionally, \tcode{stack} and \tcode{queue} require memory allocation which could throw a \tcode{bad_alloc} 
exception.

\section{Namespaces}
Graph containers and their views and algorithms are not interchangeable with existing containers and algorithms.
Additionally, there are some domain-specific terms that may clash with existing or future names, such as 
\tcode{degree} and \tcode{partition_id}.
For these reasons, we recommend their own namespaces. The following assumption is used in this proposal.
\begin{itemize}
\item[]\tcode{std::graph}, \tcode{std::graph::views} and \tcode{std::graph::edgelist}
\end{itemize}

\noindent
Alternative locations include the following:
\begin{itemize}
\item[]\tcode{std::ranges}, \tcode{std::ranges::views}, and \tcode{std::ranges::edgelist}
\item[]\tcode{std::ranges/graph}, \tcode{std::ranges::graph::views} and \tcode{std::ranges::graph::edgelist}
\end{itemize}
The advantage of these two options are that there would be no requirement to use the ranges:: prefix for things
in the std::ranges namespace, a common occurance.

\section{Notes and Considerations}
There are some interesting observations that can be made about graphs and how they compare and contrast to the 
standard library that may not be obvious.
\begin{itemize}
      \item The adjacency list, the primary data structure for this proposal, is a compound data structure of a
            range of ranges. This introduces a new form of container beyond a simple range.
      \item There is more than one possible value type, one each for edge, vertex, and graph. Each is optional.
            This is in contrast to existing practice where the value type is the distinguishing difference between
            different containers, such as for \tcode{set} and \tcode{map}.
      \item Algorithms will often use views, though they can use the GCI functions when needed.
      \item Algorithms and Views often need to allocate memory internally to achieve their purpose. This is a departure from
            common practice in the standard.
\end{itemize}

There are other observations we've also discovered along the way that may not be obvious.
\begin{itemize}
      \item Storing vertices in a \tcode{map} (bi-directional range) requires a different style of programming 
            algorithms, compared to being kept in a \tcode{vector} (random access range). When using a \tcode{vector},
            \tcode{edges(g,uid)} would normally be used without much thought. Using that with a \tcode{map} would
            incur a $\mathcal{O}(\log(V))$ cost. Instead, it will use vertex id once to get the vertex reference 
            and then use \tcode{edges(g,uv)}. This is expected to result in overloading of existing algorithms based on the
            range type of a container, distinguished with concepts.
\end{itemize}

The addition of concepts to the standard library is a serious consideration because, once added, they cannot 
be removed. We believe that adjacency lists as a range-of-ranges merits the addition new concepts but we recognize that it
may be a controversial decision. Toward that end, we will continue to include them to help clarify the examples given
and are assumed to be "For exposition only" as suggested implementation until a clear decision to include them, or not, 
is made.

\section{Issues Status}
This sections lists the known and open issues for the Graph Library proposal across all papers. They are organized
by the paper they are associated with.

\subsection{Open Design Issues}

\subsection{Open Reported Issues}
\begin{itemize}
      \item \href{https://www.wg21.link/P3127}{P3127 Background and Terminology}
            \begin{enumerate}
                  \item P1709 has lots of details which I think to be irrelevant. (P1709 is the original proposal that was split into multiple papers)
                        \begin{itemize}
                              \item Clarification: I don't find the discussion about adjacency matrices helpful, but rather a distraction. 
                                    It's not that it shouldn't be there in some form, but at the moment it has a prominence which I don't think is 
                                    commensurate with its importance to the paper, perhaps exacerbated by the fact that the paper lacks many salient details (see next point).
                        \end{itemize}
                  \item It is very hard to follow
                        \begin{itemize}
                              \item Clarification: As it stands, the paper lacks a discussion of the authors' standpoint on graph terminology, defining features 
                                    (e.g. self loops, multi-edges) and the sort of trade-offs you get by allowing/not allowing them. Put another way, I think the 
                                    paper would be easier to follow if there's a technical narrative that reveals the way the authors are thinking about this huge area. 

                                    I like the style of the motivation in P1709R5; if this could be greatly extended to include the mathematical background that Andrew is 
                                    working on, this would be really helpful. And beyond the mathematical background, as discussion of the computational tradeoffs for 
                                    both graph implementations and the associated algorithms, given certain choice, would be great to have.
                              \item This paper includes much of the content from P1709R5 for motivation. Andrew will be extending the paper to include a more rigorous 
                                    mathematical description.
                        \end{itemize}
                  \item We need to add a mathematical perspective to the paper.
                        \begin{itemize}
                              \item P3127 includes some of this. We plan on extending it to include a more rigorous mathematical description.
                        \end{itemize}
                  \item There needs to be a proper discussion about whether the paper's definition of graph is what some authors call a multigraph 
                        and whether it does/doesn't include loops.
      \end{enumerate}
      \item \href{https://www.wg21.link/P3128}{P3128 Graph Algorithms}
            \begin{enumerate}
                  \item A concern is that the DFS and BFS functionality isn't flexible enough, especially when compared to boost::graph's visitors.
                        \begin{itemize}
                              \item We agree having a more general and flexible BFS and DFS would be valuable. We have investigated the use
                                    of coroutines and boost::graph-like visitors and found that the coroutine implementation has better
                                    syntax but is 20\% slower than the visitor implementation. That kind of performance penalty is not 
                                    acceptable, making the visitor implementation the clear choice.
                              \item Visitors have been added to Dijkstra's and Belman-Ford shortest paths algorithms and will be added 
                                    to BFS, DFS and Topological Sort algorithms in the next revision of P3128.
                              \item We are also considering adding similar functionality to the views in P3129.
                        \end{itemize}
                  \item The summary tables for the algorithms are necessary but not sufficient: 
                        \begin{itemize}
                              \item There needs to be a discussion of these aspects for graph implementations themselves. Various graph operations may 
                                    be more efficient if the graph structure is more constrained. However, not allowing e.g. multiple edges between pairs 
                                    of nodes prohibits representing many useful systems. There are trade-offs and these need to be discussed.
                                    \begin{itemize}
                                          \item Some work has been done, but more is needed.
                                    \end{itemize}
                              \item A justification of the choices made for the algorithms may be helpful.
                  \end{itemize}
            \end{enumerate}
      \item \href{https://www.wg21.link/P3337}{P3337 Comparison to Other Graph Libraries} (Unpublished)
            \begin{enumerate}
                  \item My comment about the structure of the paper changing was a reference to previous comparisons with boost::graph. 
                        I'm sure these were in an earlier version, or am I misremembering?
                  \begin{itemize}
                        \item We never had any comparisons to boost::graph. 
                        \item A draft was reviewed in the March 2025 SG19 meeting. It will be published after it is updated with new
                              benchmark numbers when the descriptors functionality is fully implemented.
                  \end{itemize}
            \end{enumerate}
\end{itemize}

\subsection{Resolved Issues}
\begin{itemize}
      \item P3126 General Design
            \begin{enumerate}
                  \item Build on \tcode{mdspan} and try to standardize (or at least understand) what might reasonably be called an unstructured span
                              
                  Suppose someone standardizes unstructured span, as a natural extension of mdspan. What could we 
                  learn from its api that may be relevant for graphs? In both cases, we will presumably have a method which allows iteration over the ith partition 
                  (or edges of a given node, for graphs). Consistency of the stl may mean we want these to have the same look/feel.
                  \begin{itemize}
                        \item This is a very different direction and beyond the scope of this proposal. This will not be pursued
                              and we invite others to submit a separate proposal.
                  \end{itemize}
            \item Complete the unpublished \textbf{Graph Operators} proposal, which adds utility functions including degree, sort, relabel, transpose and join.
                  \begin{itemize}
                        \item While useful, the funcitons are not critical to offer a complete library. This is deferred until the other papers 
                              have been voted out of SG19.
                  \end{itemize}
            \end{enumerate}
      \item \href{https://www.wg21.link/P3126}{P3126 Overview}
            \begin{enumerate}
                  \item GraphBLAS is not included as part of the prior art.
                        \begin{itemize}
                              \item Added in P3126r1.
                        \end{itemize}
                  \item The electrical circuit example has issues in P3127, section 6.1.
                        \begin{itemize}
                              \item Removed in P3126r3.
                        \end{itemize}
            \end{enumerate}
      \item \href{https://www.wg21.link/P3130}{P3130 Graph Container Interface}
            \begin{enumerate}
                  \item I'm not convinced by the load API.
                        \begin{itemize}
                              \item We agree because the use of both load functions and constructors creates ambiguity and complexity when both are defined.
                                    Even though constructors weren't in the paper it wasn't clear whether they should be included or not.
                                    We have removed the load functions and added constructors for \tcode{compressed_graph} to simplify the interface.
                        \end{itemize}
                        \item Complete the definition of the edgelist concepts, types and CPO functions. This is distinct from the existing edgelist view.
                  \end{enumerate}
\end{itemize}
