
%% \chapter{Graph Container Implementations}

\section{Using Existing Graph Data Structures}
Reasonable defaults have been defined for the GCI functions to minimize the amount of work
needed to adapt an existing graph data structure to be used by the views and algorithms.

There are two cases supported. The first is for the use of standard containers to define the graph and the other
is for a broader set of more complicated implementations.

\subsection{Using Standard Containers for the Graph Data Structure}

For example this we'll use \tcode{G = vector<forward_list<tuple<int,double>>>} to define the graph, where \tcode{g} 
is an instance of \tcode{G}. \tcode{tuple<int,double>} defines the target\_id and weight property respectively. We
can write loops to go through the vertices, and edges within each vertex, as follows.

\begin{lstlisting}
using G = vector<forward_list<tuple<int,double>>>;
auto weight = [&g](edge_t& uv) { return get<1>(uv); }

G g;
load_graph(g, ...); // load some data

// Using GCI functions
for(auto&& [uid, u] : vertices(g)) {
  for(auto&& [vid, uv]: edges(g,u)) {
    auto w = weight(uv);
    // do something...
  }
}
\end{lstlisting}

Note that no function override was required and is a special case when the outer range is a \tcode{random_access_range} and
and inner inner range is a \tcode{forward_range}, and the value type of the inner range is either \tcode{integral}
 or \tcode{tuple<integral, ...>}. This extends to any range type. For instance, boost::containers can be used
 just as easily as std containers. 

\begin{table}[h!]
    \begin{center}
    \resizebox{\textwidth}{!}
    {\begin{tabular}{l l p{1.5cm} L{7.0cm}}
    \hline
        \textbf{Function or Value} & \textbf{Concrete Type} \\
    \hline
        \tcode{vertices(g)} & \tcode{vector<forward_list<tuple<int,double>>>} (when \tcode{random_access_range<G>}) \\
        \tcode{u} & \tcode{forward_list<tuple<int,double>>} & \\
        \tcode{edges(g,u)} & \tcode{forward_list<tuple<int,double>>} (when \tcode{random_access_range<vertex_range_t<G>>}) \\
        \tcode{uv} & \tcode{tuple<int,double>} & \\
        \tcode{edge_value(g,uv)} & \tcode{tuple<int,double>} (when \tcode{random_access_range<vertex_range_t<G>>}) \\
        \tcode{target_id(g,uv)} & \tcode{integral}, when \tcode{uv} is either \tcode{integral}  or \tcode{tuple<integral,...>} \\
    \hline
    \end{tabular}}
    \caption{Types When Using Standard Containers}
    \label{tab:simple_graph}
    \end{center}
\end{table}


\subsection{Using Other Graph Data Structures}
For other graph data structures more function overrides are required. Table \ref{tab:cmn_cpo_overrides} shows the 
common function overrides anticipated for most cases, keeping in mind that all functions can be overridden.
When they are defined they must be in the same namespace as the data structures.

\begin{table}[h!]
    \begin{center}
    %\resizebox{\textwidth}{!}
    {\begin{tabular}{l l p{1.5cm} L{7.0cm}}
    \hline
        \textbf{Function} & \textbf{Comment} \\
    \hline
        \tcode{vertices(g)} & \\
        \tcode{edges(g,u)} & \\
        \tcode{target_id(g,uv)} & \\
        \tcode{edge_value(g,uv)} & If edges have value(s) in the graph \\
        \tcode{vertex_value(g,u)} & If vertices have value(s) in the graph \\
        \tcode{graph_value(g)} & If the graph has value(s) \\
    \hline
        \multicolumn{2}{c}{When edges have the optional source\_id on an edge} \\
    \hdashline
        \tcode{source_id(g,uv)} & \\
    \hline
        \multicolumn{2}{c}{When the graph supports multiple partitions} \\
    \hdashline
        \tcode{partition_count(g)} & \\
        \tcode{partition_id(g,u)} & \\
        \tcode{vertices(g,u,pid)} & \\
    \hline
    \end{tabular}}
    \caption{Common CPO Function Overrides}
    \label{tab:cmn_cpo_overrides}
    \end{center}
\end{table}

\section{compressed\_graph}
The \tcode{compressed_graph} that is being proposed for the standard library. It is a high-performance graph container that 
uses \href{https://en.wikipedia.org/wiki/Sparse_matrix#Compressed_sparse_row_\%28CSR\%2C_CRS_or_Yale_format\%29}{Compressed Sparse Row} 
format to store its vertices, edges and associated values. Once constructed, vertices and edges cannot be added or deleted but values 
on vertices and edges can be modified.

The following listing shows the prototype for the \tcode{compressed_graph}. Only the members shown for \tcode{compressed_graph} are public. 
No other member functions or types are exposed as part of the standard. All other types are only accessible through the types and functions 
in the Graph Container Interface. Multiple partitions (multi-partite) can be defined by passing the number of partitions in a constructor.

When a value type template argument (EV, VV, GV) is void then no extra overhead is incurred for it. 
The selection of the VId template argument impacts the inter storage requirements. If you have a small 
graph where the number of vertices is less than 256, and the number of edges is less than 256, 
then a \tcode{uint8_t} would be sufficient.

\begin{table}[h]
    \setcellgapes{3pt}
    \makegapedcells
    \centering
    \begin{tabular}{|P{0.35\textwidth}|P{0.30\textwidth}|P{0.35\textwidth}|}
    \hline
    \textbf{Implements \tcode{load_graph}?} Yes & \textbf{Can append vertices?} No & \textbf{vertex\_id assignment:} Contiguous\\
    \textbf{Implements \tcode{load_vertices}?} Yes & \textbf{Can append edges?} No & \textbf{Vertices range:} Contiguous \\
    \textbf{Implements \tcode{load_edges}?} Yes &  & \textbf{Edge range:} Contiguous \\
    \textbf{Implements \tcode{load_partition}?} Yes &  & \\
    \hline
    \end{tabular}
    %\caption{Jaccard Coefficient Summary}
    \label{tab:compressed_graph_summary}
\end{table}
Vertices and edges cannot be appended to an existing partition, but they can be added to a new partition.

\begin{lstlisting}
template <class    EV     = void,     // Edge Value type
          class    VV     = void,     // Vertex Value type
          class    GV     = void,     // Graph Value type
          integral VId    = uint32_t, // vertex id type
          integral EIndex = uint32_t, // edge index type
          class    Alloc  = allocator<uint32_t>> // for internal containers
class compressed_graph {
public:
    compressed_graph();
    compressed_graph(size_t num_partitions); // multi-partite
    compressed_graph(const compressed_graph&);
    compressed_graph(compressed_graph&&);
    {tilde}compressed_graph();

    compressed_graph& operator=(const compressed_graph&);
    compressed_graph& operator=(compressed_graph&&);
}
\end{lstlisting}

\phil{Additional members to define?}

\phil{tilde in destructor is giving problems in latex}

